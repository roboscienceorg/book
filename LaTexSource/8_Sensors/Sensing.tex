\hypertarget{sensing}{%
\section{Sensing}\label{sensing}}

Sensors are a key tool to perceiving the environment. Our eyes, ears,
nose, tongue, and skin are all sensors giving us details about our
surroundings. Whether the environment is known or unknown, a robot
requires sensors to perceive it as well. A roboticist needs to
understand how a sensor functions and what its limitations are in order
to use it to its full capacity. These limitations could include noise,
bandwidth, data errors, and many other issues that must be accounted for
in order to get accurate results. Understanding the physical principles
of the sensors available is the key to understanding, modeling, and
utilizing this information.

A sensor can be any device that converts energy into a usable signal.
Sensors fall into two classes: \texttt{passive\ sensors} and
\texttt{active\ sensors}. \textbf{Passive Sensors} use energy from the
environment to power the measurement. A bump or temperature sensor are
examples of passive sensing. \textbf{Active sensors} inject energy into
the environment in a particular manner and measure the reaction. Active
sensing can often get better results, but at an increased complexity,
cost, and power requirements. It could also require the modification of
the surrounding environment, such as the placement of beacons or tags.
Laser and ultrasonic ranging are examples of active sensing.

We can further classify sensors by which part of the robot's
environment, \texttt{properioceptive} or \texttt{exterioceptive}, they
are sensing. \textbf{Properioceptive Sensors} measure the internal state
of the system (robot), such as motor speeds, wheel loads, turn angles,
battery status, temperature and other aspects that are internal to the
machine. \textbf{Exterioceptive Sensors} measure information from the
robot's external environment; external to the robot, such as distances
to objects, GPS, ambient light and temperature, magnetic fields,
accelerations, etc.

Another way to classify sensors is by the data they return. Sensors can
return information in \textbf{analog} or \textbf{digital} form.
Typically an \texttt{analog\ sensor} will vary voltage (maybe current)
as the measured quantity changes value. Normal application is to feed
that signal into a device known as an ADC or analog to digital
converter. The vast array of microcontrollers on the market offer built
in ADC lines. For example Arduino boards can take in analog signals. The
voltage level is sampled and converted to a numerical value (hence
digital). This means you might need some glue electronics to convert
your sensor's voltage range to the full range of the ADC inputs. The ADC
will be listed at having a certain number of bits (say 12 bit). This
gives the resolution. For a 12 bit device, it means that the sampling
will break the signal into \(2^{12}\) discrete values or 4096 different
levels. More bits means better resolution. But it takes more hardware
and memory inside so there is a tradeoff. A \texttt{digital\ sensor} is
a sensor that returns the measurement already in digitized or packet
form and there is no need for an analog to digital conversion.

Digital sensors are often analog devices with built in ADC chips. To
save packaging space they will very often communicate via a bus and not
have the 12 pins required for a 12 bit sampling. So beyond the ADC, they
will have some other device inside to send the signal out on some type
of serial line protocol (uart, i2c, spi, ...). Digital devices can be
more accurate than analog devices but not always. It depends on the
number of bits used and other factors in fabrication.

The desired qualities of a sensor are high accuracy and resolution, wide
range of measurement, low delay times, stability of measurement with
respect to the environment (no temperature drift or magnetic field
interference for example), and above all very low cost. Sensors will
talk to computers using two standard methods: polling and interrupts.
For polling, the computer periodically reads the value on the correct
register. Simplistically this is implemented via a loop in the software
with a delay although better approaches involve setting a cpu timer and
using an interrupt. The other approach is to have the sensor generate
the interrupt and the cpu's interrupt handler will read the value on the
register.

\hypertarget{sensor-metrics}{%
\subsection{Sensor Metrics}\label{sensor-metrics}}

There is a significant range in the quality of the sensed data. Some
sensors may be very narrow in the range of sensing (e.g. range or angle
of perception) while other are very wide. Sensors have noise which can
vary depending on the sensor and the environment. Sensors will measure
some quantity over some range, there are maximum and minimum values for
inputs. The dynamic range is the ratio of the upper limit to the lower
limit. Ranges can be very large and so the decibel is normally used. The
formula for expressing the ratio depends on whether the sensed quantity
is related to power or a field. Use of the 20 instead of the 10 is based
on the standard use of the ratio of the squares and so we have a factor
of 2 which comes out front.

\begin{description}
\item[Power]
\[L_p = 10\log_{10} \left( \frac{P}{P_0}\right) \mbox{dB}\]
\item[Field]
\[L_p = 20\log_{10} \left( \frac{F}{F_0}\right) \mbox{dB}\]
\end{description}

Note that the ratio makes the decibel a unitless quantity.

Compute the dynamic range in dB for a measurement from 20mW to 50kW.

\[L_p = 10\log_{10} \left( \frac{50000}{.02}\right) \mbox{dB} = 63.979 \mbox{dB}\]

Compute the dynamic range in dB for a measurement from 0.1V to 12V.

\[L_p = 20\log_{10} \left( \frac{12}{.1}\right) \mbox{dB} =  41.584\mbox{dB}\]

We have been using some terms that describe the sensor data and it is
worth reviewing these terms.

\begin{description}
\item[Resolution]
In the world of digital sensors, this is often described as the number
of bits used. It is the smallest change in the sensed value that can be
measured. It is what you see in your science courses on measurement
precision.
\item[Accuracy]
It is how close the reported or measured value is to the actual value.
\item[Range]
Or measurement range. It is the range of input values the sensor can
detect.
\item[Repeatability]
This describes the changes in the measured parameter over multiple
measurements with a fixed value.
\item[Frequency]
Some sensors produce new values at some clock rate which is given by
frequency.
\item[Response time]
The time delay between the measurement and the output value. Sometimes
this will be used as the time delay between when the cpu requests a
measurement and when the measurement is the available to the cpu.
\item[Linearity]
The signal output is a linear function of the input.
\item[Sensitivity]
The ratio of measured value to sensor output value.
\end{description}

All sensing involves measurement errors. There are standard ways to
measure the error. Assume that \(x\) is the true value and \(z\) is the
measured value. The \emph{absolute error} is given by \(| x - z|\). The
\emph{relative error} is given by \(| 1 - z/x|\). The reason we might
choose relative error over absolute error is based on scale. For
example, which of the pairs would you say is a better estimate:
\((x,z) = (0.1, 0.2)\) and \((x,z) = (100, 102)\)? The absolute error
for the first is 0.1 and for the second is 2. Two is larger than 0.1.
But intuitively we see that going from 100 to 102 is closer at the scale
of 100. The relative error shows this with the first being a relative
error of \(|1 - 0.2/0.1| = 2\). The relative error on the second one is
\(|1-102/100| = 0.02\). This fits with our intuition about the errors.
Relative error removes the scale and can be reported as a percentage
which is called the percentage error, \(100|1-z/x|\). The accuracy of a
measurement is given by 100 -percentage error.

\hypertarget{sensor-interaction}{%
\subsection{Sensor Interaction}\label{sensor-interaction}}

When working with embedded devices, there are two common methods that
data is read from a sensor: \texttt{polling} and \texttt{interrupts}.
Polling is a simple technique to read the sensor at some fixed
frequency. This is easily done by using a loop with a delay. Interrupts
are an approach that requires hardware support. When the sensor has new
information, it triggers a read operation. This can be explicitly by
pulling a IO line low or implicitly when the data is written on an IO
location that triggers the interrupt.

Polling is simple to implement. It does place the CPU in a loop and can
lock of the cpu as well as use up cpu time. Having multiple devices
polled can affect how fast the loop can operate. Polling also has the
hardwware reading the sensor when no new data is available.

Interrupts require some basic interrupt hardware support and a way that
the sensor can trigger the interrupt. Once an interrupt is triggered,
the cpu runs a special routine (function) called an interrupt handler.
Typically this routine grabs and processes the sensor data. Interrupt
approaches do not tie up the CPU like polling can. It can also be more
responsive if the polling loop is running at a low frequency.
