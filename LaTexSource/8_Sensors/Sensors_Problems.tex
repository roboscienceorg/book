\hypertarget{problems}{%
\section{Problems}\label{problems}}

\begin{enumerate}
\item
  Assume that you are working in a large event center which has beacons
  located around the facility. Estimate the location of a robot,
  \((a,b,c)\), if the \((x,y,z)\) location of the beacon and the
  distance from the beacon to the robot, \(d\), are given in the table
  below.

  \begin{longtable}[]{@{}llll@{}}
  \toprule
  \begin{minipage}[b]{0.07\columnwidth}\raggedright
  x\strut
  \end{minipage} & \begin{minipage}[b]{0.07\columnwidth}\raggedright
  y\strut
  \end{minipage} & \begin{minipage}[b]{0.07\columnwidth}\raggedright
  z\strut
  \end{minipage} & \begin{minipage}[b]{0.07\columnwidth}\raggedright
  d\strut
  \end{minipage}\tabularnewline
  \midrule
  \endhead
  \begin{minipage}[t]{0.07\columnwidth}\raggedright
  884\strut
  \end{minipage} & \begin{minipage}[t]{0.07\columnwidth}\raggedright
  554\strut
  \end{minipage} & \begin{minipage}[t]{0.07\columnwidth}\raggedright
  713\strut
  \end{minipage} & \begin{minipage}[t]{0.07\columnwidth}\raggedright
  222\strut
  \end{minipage}\tabularnewline
  \begin{minipage}[t]{0.07\columnwidth}\raggedright
  120\strut
  \end{minipage} & \begin{minipage}[t]{0.07\columnwidth}\raggedright
  703\strut
  \end{minipage} & \begin{minipage}[t]{0.07\columnwidth}\raggedright
  771\strut
  \end{minipage} & \begin{minipage}[t]{0.07\columnwidth}\raggedright
  843\strut
  \end{minipage}\tabularnewline
  \begin{minipage}[t]{0.07\columnwidth}\raggedright
  938\strut
  \end{minipage} & \begin{minipage}[t]{0.07\columnwidth}\raggedright
  871\strut
  \end{minipage} & \begin{minipage}[t]{0.07\columnwidth}\raggedright
  583\strut
  \end{minipage} & \begin{minipage}[t]{0.07\columnwidth}\raggedright
  436\strut
  \end{minipage}\tabularnewline
  \begin{minipage}[t]{0.07\columnwidth}\raggedright
  967\strut
  \end{minipage} & \begin{minipage}[t]{0.07\columnwidth}\raggedright
  653\strut
  \end{minipage} & \begin{minipage}[t]{0.07\columnwidth}\raggedright
  46\strut
  \end{minipage} & \begin{minipage}[t]{0.07\columnwidth}\raggedright
  529\strut
  \end{minipage}\tabularnewline
  \begin{minipage}[t]{0.07\columnwidth}\raggedright
  593\strut
  \end{minipage} & \begin{minipage}[t]{0.07\columnwidth}\raggedright
  186\strut
  \end{minipage} & \begin{minipage}[t]{0.07\columnwidth}\raggedright
  989\strut
  \end{minipage} & \begin{minipage}[t]{0.07\columnwidth}\raggedright
  610\strut
  \end{minipage}\tabularnewline
  \bottomrule
  \end{longtable}
\item
  If you are using a laser diode to build a distance sensor, you need
  some method to determine the travel time. Instead of using pulses and
  a clock, try using phase shifts. What is the wavelength of the
  modulated frequency of 10MHz? If you measure a 10 degree phase shift,
  this value corresponds to what distances? What if the phase shift
  measurement has noise: zero mean with standard deviation 0.1? How does
  one get a good estimate of position if the ranges to be measured are
  from 20 meters to 250 meters?
\item
  Write a Python function to simulate a LIDAR. The simulated LIDAR will
  scan a map and return the distance array. We assume that the obstacle
  map is stored in a two dimensional gridmap, call it map. You can use a
  simple gridmap which uses 0 for a free space cell and 1 for an
  occupied cell. The robot pose (location and orientation) will be
  stored in a list called pose which will hold x, y, theta (where these
  are in map cordinates). Place LIDAR parameters into a list which has
  total sweep angle, sweep increment angle and range. The function call
  will be data = lidar(pose, objmap, params) in which data is the 1D
  array of distance values to obstacles as a function of angle. Test
  this on a map with more than one obstacle. Appendix A shows how one
  may generate a map in a bit map editor like GIMP and then export in a
  plain text format which is easily read into a Python (Numpy) array.
  {[}Although you can fill the grid by a python function which sets the
  values, using the bit map editor will be much easier in the long
  run.{]}
\item
  Assume you have a laser triangulation system as shown in
  \texttt{fig:lasertriangulation2} given by \texttt{industrialvision}
  and that \(f  = 8\)mm, \(b = 30\)cm. What are the ranges for
  \(\alpha\) and \(u\) if we need to measure target distances in a
  region (in cm) \(20 < z < 100\) and \(10 < x < 30\)?
\item
  Assume you have two cameras that are calibrated into a stereo pair
  with a baseline of 10cm, and focal depth of 7mm. If the error is 10\%
  on \(v_1\) and \(v_2\), \(v_1 =  2\)mm and \(v_2 = 3\)mm, what is the
  error on the depth measurement \(z\)? Your answer should be a
  percentage relative to the error free number. Hint: If \(v_1 = 2\)
  then a 10\% error ranges from 1.8 to 2.2. {[}Although not required,
  another way to approach this problem is the total differential from
  calculus.{]}
\item
  Assume you have two cameras that are calibrated into a stereo pair
  with an estimated baseline of 10cm, and focal depth of 10mm. If the
  error is 10\% on the baseline, what is error on the depth measurement
  \(z\) with \(v_1 = 2\)mm and \(v_2 = 3\)mm? Your answer should be a
  percentage relative to the error free number. See the hint above.
\item
  With a single camera, explain how a straight line (produced by a
  laser) can resolve depth information.
\end{enumerate}
