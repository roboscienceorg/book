\hypertarget{moving-the-differential-drive-robot}{%
\section{Moving the Differential Drive
robot}\label{moving-the-differential-drive-robot}}

In the last section, we moved the two link articulator by updating the
position. It is certainly possible to simulate a robot moving through
space by simply jumping positions. Motion effect is produced like a
movie projector gives the impression of motion. So, for animation, this
approach can and often does suffice. However, objects in the world don't
jump positions. Momentum, inertia and limits on acceleration and
velocity do play a significant role. To move a ground robot, the
position should be controlled by velocity. In reality the position is
controlled by control signals to a motor which in turn generates a
velocity, but we will assume we have a perfect motor controller for now;
one that can take a velocity command and achieve that velocity.

In this section we simulate the motion of the differential drive robot
that we introduced in the Terms Chapter shown in
\texttt{ddriveRecalled}.

\begin{quote}
Simple differential drive robot.
\end{quote}

and the associated equations~\texttt{ddkinematicsmodel}

\[\begin{aligned}
\boxed{
\begin{array}{l}
 \dot{x} = \frac{r}{2} (\dot{\phi_1}+\dot{\phi_2})\cos(\theta) \\[5mm]
\dot{y} = \frac{r}{2} (\dot{\phi_1}+\dot{\phi_2})\sin(\theta) \\[5mm]
\dot{\theta} = \frac{r}{2L} (\dot{\phi_1}-\dot{\phi_2})
\end{array}}
\end{aligned}\]

where \(\dot{\phi_1}\) and \(\dot{\phi_2}\) are the right and left wheel
rotational speeds (respectively), \(r\) is wheel radius and \(2L\) is
the axle length.

In practice you want to prescribe \(x(t), y(t)\) to obtain
\(\dot{\phi_1},\dot{\phi_2}\). Clearly if you have \(x(t), y(t)\),
differentiation will yield \(\dot{x}(t), \dot{y}(t)\), so we may assume
that we know \(\dot{x}(t), \dot{y}(t)\). Using \(\dot{x}\) and
\(\dot{y}\) we may drive the robot along the curve of interest. We will
explore methods to do this below.

Our work last chapter has given us

\[\theta(t) = \theta(0) + \int_0^t \frac{r}{2L} \left(\frac{d\phi_1}{d\tau}-\frac{d\phi_2}{d\tau}\right)d\tau\]

If you have the mathematical form to determine \(\dot{\phi_i}\), you can
proceed like in the last chapter. But it might not have a standard
functional form. In general, looking for an analytic function for the
position is not possible. Practically you don't actually have a function
representation of \(\phi(t)\). We are normally sampling the wheel
angular velocity during runtime. How should we formulate and proceed in
that case?

So what if we relax the constant velocity assumption. This gives rise to
two additional issues. The first is that you may not be able to gain an
antiderivative of the wheel velocities to find \(\theta(t)\). If you are
able to find \(\theta\), the right hand sides for \(\dot{x}\) and
\(\dot{y}\) normally are not integrable. A simple example below
demonstrates issues with finding antiderivatives.

Let \(\dot{\phi_1} = e^{-t^2}\) and \(\dot{\phi_2} = t\)

\[\theta(t) = \theta(0) + \int_0^t \frac{r}{2L} \left(e^{-\tau^2}-\tau\right)d\tau = ???\]

This integral cannot be resolved. Meaning we cannot find an analytic
antiderivative. It is possible to approximate it either with a Taylor
expansion or numerical formulation, but it is an example of a vast
number of functions which we must stop at this step.

In practice it is not possible to instantaneously jump wheel speeds as
we did in our analytical approach previously. Inertia in the system
(mass, inductance, power limits) means that it is not possible to
instantaneous jumps in velocity. In addition, it is not possible to have
perfect velocities when surfaces and power are not consistent. These
errors in wheel velocites can over time translate to significant errors
in position.

\hypertarget{a-numerical-approach}{%
\subsection{A numerical approach}\label{a-numerical-approach}}

We will use Euler's (\emph{Oil-ler's}) method for solving the
differential equations. \texttt{Euler’s\ method} approximates the
derivatives with a forward finite difference and converts the
differential equation into a difference equation. The difference
equations are algebraic and can be evaluated numerically. This is also
known as a finite difference method. Let the time between measurements
be denoted by \(\Delta t\). We discretize (or approximate) the time
variable and the three state variables using discrete variables. This
simple means we have a sequence of numbers \(\{x_k\}\) instead of a
function \(x(t)\). Technically we should use a different variable, but I
will often be efficient\footnote{that would be a \emph{codeword} for
  sloppy} and reuse the variable even though one denotes a function of
time and one denotes a sequence.

\[t_k \equiv k\Delta t, \quad t_{k+1} = (k+1)\Delta t\]

\[x_k \equiv x(t_k), ~~~ y_k \equiv y(t_k)\]

\[\omega_{1, k}\equiv \dot{\phi}_{1}(t_k), ~~~
\omega_{2, k}\equiv \dot{\phi}_{2}(t_k)\]

Recall that if \(x\) is position then \(\dot{x}=dx/dt\) is velocity (and
\(\ddot{x}=d^2x/dt^2\) is acceleration). From basic calculus, we recall
that we may approximate a derivative using a forward
\texttt{finite\ difference}:

\[\dot{x} \approx \frac{x(t+\Delta t) - x(t)}{\Delta t}.\]

Using this we can take a time step of \(\Delta t\) forward (meaning
\(t_{k+1} = t_k + \Delta t\)) and Euler's method gives us

\[x(t_{k+1}) = x(t_k) + (\Delta t)x'(t_k) \quad \mbox{and}
\quad y(t_{k+1}) = y(t_k) + (\Delta t)y'(t_k).\]

And so we can write our differential equations as difference equations,

\[\begin{aligned}
\begin{array}{l}
\displaystyle \frac{x(t+\Delta t) - x(t)}{\Delta t}\approx \dot{x} = \frac{r}{2} (\dot{\phi_1}+\dot{\phi_2})\cos(\theta) \\[5mm]
\displaystyle \frac{y(t+\Delta t) - y(t)}{\Delta t}\approx \dot{y} = \frac{r}{2} (\dot{\phi_1}+\dot{\phi_2})\sin(\theta) \\[5mm]
\displaystyle \frac{\theta (t+\Delta t) - \theta (t)}{\Delta t}\approx \dot{\theta} = \frac{r}{2L} (\dot{\phi_1}-\dot{\phi_2})
\end{array}
\end{aligned}\]

After some algebra, we obtain:

\[\begin{aligned}
\begin{array}{l}
 x(t+\Delta t) \approx x(t) +\frac{r\Delta t}{2} (\dot{\phi_1}+\dot{\phi_2})\cos(\theta) \\[5mm]
 y(t+\Delta t) \approx y(t) +\frac{r\Delta t}{2} (\dot{\phi_1}+\dot{\phi_2})\sin(\theta) \\[5mm]
\theta (t+\Delta t) \approx \theta (t) +\frac{r\Delta t}{2L} (\dot{\phi_1}-\dot{\phi_2}).
\end{array}
\end{aligned}\]

Using the discrete (sample) variables, \(x(t_k) \to x_k\), etc, we can
rewrite the expression in terms of the discrete variables. Given
starting configuration and wheel velocity measurements, we have the
following difference equations:

\[\begin{aligned}
\begin{array}{l}
 x_{k+1} = x_k + \frac{r\Delta t}{2} (\omega_{1, k}+\omega_{2, k})\cos(\theta_k) \\[5mm]
y_{k+1} = y_k + \frac{r\Delta t}{2} (\omega_{1, k}+\omega_{2, k})\sin(\theta_k) \\[5mm]
\theta_{k+1} = \theta_k + \frac{r\Delta t}{2L} (\omega_{1, k}-\omega_{2, k})
\end{array}
\end{aligned}\]

These equations are the main model for approximating motion of a
differential drive robot. It has also been used as a first approximation
for a tractor or tank drive system.

\hypertarget{listDDstep}{%
\label{listDDstep}}%
\begin{Shaded}
\begin{Highlighting}[]
\KeywordTok{function}\NormalTok{ ddstep(xc}\OperatorTok{,}\NormalTok{ yc}\OperatorTok{,}\NormalTok{ qc}\OperatorTok{,}\NormalTok{r}\OperatorTok{,}\NormalTok{l}\OperatorTok{,}\NormalTok{dt}\OperatorTok{,}\NormalTok{w1}\OperatorTok{,}\NormalTok{w2)}
\NormalTok{   xn }\OperatorTok{=}\NormalTok{ xc }\OperatorTok{+}\NormalTok{ (r}\OperatorTok{*}\NormalTok{dt}\OperatorTok{/}\FloatTok{2.0}\NormalTok{)}\OperatorTok{*}\NormalTok{(w1}\OperatorTok{+}\NormalTok{w2)}\OperatorTok{*}\NormalTok{cos(qc)}
\NormalTok{   yn }\OperatorTok{=}\NormalTok{ yc }\OperatorTok{+}\NormalTok{ (r}\OperatorTok{*}\NormalTok{dt}\OperatorTok{/}\FloatTok{2.0}\NormalTok{)}\OperatorTok{*}\NormalTok{(w1}\OperatorTok{+}\NormalTok{w2)}\OperatorTok{*}\NormalTok{sin(qc)}
\NormalTok{   qn }\OperatorTok{=}\NormalTok{ qc }\OperatorTok{+}\NormalTok{ (r}\OperatorTok{*}\NormalTok{dt}\OperatorTok{/}\NormalTok{(}\FloatTok{2.0}\OperatorTok{*}\NormalTok{l))}\OperatorTok{*}\NormalTok{(w1}\OperatorTok{{-}}\NormalTok{w2)}
   \KeywordTok{return}\NormalTok{ (xn}\OperatorTok{,}\NormalTok{yn}\OperatorTok{,}\NormalTok{qn)}
\KeywordTok{end}
\end{Highlighting}
\end{Shaded}

Assume that \(r=1\), \(dt = 0.1\), \(w1=w2=2\) and \(l=6\) and take the
initial pose to be \(x=1\), \(y=2\) and \(\theta = q =0.7\). The
following is a Python program to take 10 steps with the 0.1 time step:

\hypertarget{listDDstepwrapper}{%
\label{listDDstepwrapper}}%
\begin{Shaded}
\begin{Highlighting}[]
\NormalTok{xc }\OperatorTok{=} \FloatTok{1}\OperatorTok{;}\NormalTok{ yc }\OperatorTok{=} \FloatTok{2}\OperatorTok{;}\NormalTok{ qc  }\OperatorTok{=} \FloatTok{0.7}
\NormalTok{t }\OperatorTok{=} \FloatTok{0}\OperatorTok{;}\NormalTok{  dt }\OperatorTok{=} \FloatTok{0.1}
\KeywordTok{for}\NormalTok{ i}\OperatorTok{=} \FloatTok{1}\OperatorTok{:}\FloatTok{10}
   \KeywordTok{global}\NormalTok{ xc}\OperatorTok{,}\NormalTok{ yc}\OperatorTok{,}\NormalTok{ qc}\OperatorTok{,}\NormalTok{ t}\OperatorTok{,}\NormalTok{ dt}
\NormalTok{   xc}\OperatorTok{,}\NormalTok{ yc}\OperatorTok{,}\NormalTok{ qc }\OperatorTok{=}\NormalTok{ ddstep(xc}\OperatorTok{,}\NormalTok{ yc}\OperatorTok{,}\NormalTok{ qc}\OperatorTok{,}\FloatTok{1.0}\OperatorTok{,}\FloatTok{6.0}\OperatorTok{,}\NormalTok{dt}\OperatorTok{,}\FloatTok{2.0}\OperatorTok{,}\FloatTok{2.0}\NormalTok{)}
\NormalTok{   t }\OperatorTok{=}\NormalTok{ t }\OperatorTok{+}\NormalTok{ dt}
\NormalTok{   println(round(t}\OperatorTok{,}\NormalTok{digits}\OperatorTok{=}\FloatTok{3}\NormalTok{)}\OperatorTok{,} \StringTok{" "}\OperatorTok{,}\NormalTok{ round(xc}\OperatorTok{,}\NormalTok{digits}\OperatorTok{=}\FloatTok{5}\NormalTok{)}\OperatorTok{,} \StringTok{" "}\OperatorTok{,}\NormalTok{ round(yc}\OperatorTok{,}\NormalTok{digits}\OperatorTok{=}\FloatTok{5}\NormalTok{)}\OperatorTok{,} \StringTok{" "}\OperatorTok{,}\NormalTok{round(qc}\OperatorTok{,}\NormalTok{digits}\OperatorTok{=}\FloatTok{5}\NormalTok{))}
\KeywordTok{end}
\end{Highlighting}
\end{Shaded}

The output:

\begin{verbatim}
0.1 1.15297 2.12884 0.7
0.2 1.30594 2.25769 0.7
0.3 1.45891 2.38653 0.7
0.4 1.61187 2.51537 0.7
0.5 1.76484 2.64422 0.7
0.6 1.91781 2.77306 0.7
0.7 2.07078 2.9019 0.7
0.8 2.22375 3.03075 0.7
0.9 2.37672 3.15959 0.7
1.0 2.52968 3.28844 0.7
\end{verbatim}

The Euler approximation amounts to assuming the vehicle has constant
wheel velocity over the interval \(\Delta t\), see
\texttt{fig:piecewiseconst}. The assumption of piecewise constant
velocity does not hold in the general case and so we see accumulating
drift when comparing the robot's true path and the approximated one.

\begin{quote}
Piecewise Constant nature of the Euler Approximation.
\end{quote}

A simple modification of the code can accept other wheel speeds. For
example, if the wheel speeds are given by \(w1 = 0.1 + 2*t\) and
\(w2 = 0.1\), we would have

\hypertarget{listDDvarwheel}{%
\label{listDDvarwheel}}%
\begin{Shaded}
\begin{Highlighting}[]
\NormalTok{xc }\OperatorTok{=} \FloatTok{1}\OperatorTok{;}\NormalTok{ yc }\OperatorTok{=} \FloatTok{2}\OperatorTok{;}\NormalTok{ qc  }\OperatorTok{=} \FloatTok{0.7}
\NormalTok{t }\OperatorTok{=} \FloatTok{0}\OperatorTok{;}\NormalTok{  dt }\OperatorTok{=} \FloatTok{0.1}
\KeywordTok{for}\NormalTok{ i }\OperatorTok{=} \FloatTok{1}\OperatorTok{:}\FloatTok{10}
    \KeywordTok{global}\NormalTok{ t}\OperatorTok{,}\NormalTok{ dt}\OperatorTok{,}\NormalTok{ xc}\OperatorTok{,}\NormalTok{ yc}\OperatorTok{,}\NormalTok{ qc}
\NormalTok{    w1 }\OperatorTok{=} \FloatTok{0.1} \OperatorTok{+} \FloatTok{2}\OperatorTok{*}\NormalTok{t}
\NormalTok{    w2 }\OperatorTok{=} \FloatTok{0.1}
\NormalTok{    xc}\OperatorTok{,}\NormalTok{ yc}\OperatorTok{,}\NormalTok{ qc }\OperatorTok{=}\NormalTok{ ddstep(xc}\OperatorTok{,}\NormalTok{ yc}\OperatorTok{,}\NormalTok{ qc}\OperatorTok{,}\FloatTok{1.0}\OperatorTok{,}\FloatTok{6.0}\OperatorTok{,}\NormalTok{dt}\OperatorTok{,}\NormalTok{w1}\OperatorTok{,}\NormalTok{w2)}
\NormalTok{    t }\OperatorTok{+=}\NormalTok{  dt}
\NormalTok{    println(round(t}\OperatorTok{,}\NormalTok{digits}\OperatorTok{=}\FloatTok{3}\NormalTok{)}\OperatorTok{,} \StringTok{" "}\OperatorTok{,}\NormalTok{ round(xc}\OperatorTok{,}\NormalTok{digits}\OperatorTok{=}\FloatTok{5}\NormalTok{)}\OperatorTok{,} \StringTok{" "}\OperatorTok{,}\NormalTok{ round(yc}\OperatorTok{,}\NormalTok{digits}\OperatorTok{=}\FloatTok{5}\NormalTok{)}\OperatorTok{,} \StringTok{" "}\OperatorTok{,}\NormalTok{round(qc}\OperatorTok{,}\NormalTok{digits}\OperatorTok{=}\FloatTok{5}\NormalTok{))}
\KeywordTok{end}
\end{Highlighting}
\end{Shaded}

The output:

\begin{verbatim}
0.1 1.00765 2.00644 0.7
0.2 1.02295 2.01933 0.70167
0.3 1.04586 2.03869 0.705
0.4 1.07632 2.06461 0.71
0.5 1.11424 2.0972 0.71667
0.6 1.15948 2.13662 0.725
0.7 1.21188 2.18304 0.735
0.8 1.27122 2.23668 0.74667
0.9 1.33728 2.29781 0.76
1.0 1.40976 2.3667 0.775
\end{verbatim}

You can plot the motion in Python. Another example with circular motion:

\hypertarget{listDDplot}{%
\label{listDDplot}}%
\begin{Shaded}
\begin{Highlighting}[]
\NormalTok{N }\OperatorTok{=} \FloatTok{200}
\NormalTok{x }\OperatorTok{=}\NormalTok{ zeros(N)}
\NormalTok{y }\OperatorTok{=}\NormalTok{ zeros(N)}
\NormalTok{q }\OperatorTok{=}\NormalTok{ zeros(N)}

\NormalTok{x[}\FloatTok{1}\NormalTok{] }\OperatorTok{=} \FloatTok{1.0}\OperatorTok{;}\NormalTok{ y[}\FloatTok{1}\NormalTok{] }\OperatorTok{=} \FloatTok{2.0}\OperatorTok{;}\NormalTok{ q[}\FloatTok{1}\NormalTok{]  }\OperatorTok{=} \FloatTok{0.7}
\NormalTok{t }\OperatorTok{=} \FloatTok{0}\OperatorTok{;}\NormalTok{  dt }\OperatorTok{=} \FloatTok{0.1}
\KeywordTok{for}\NormalTok{ i }\OperatorTok{=} \FloatTok{1}\OperatorTok{:}\NormalTok{N}\OperatorTok{{-}}\FloatTok{1}
    \KeywordTok{global}\NormalTok{ t}\OperatorTok{,}\NormalTok{ dt}\OperatorTok{,}\NormalTok{ x}\OperatorTok{,}\NormalTok{ y}\OperatorTok{,}\NormalTok{ q}
\NormalTok{    w1 }\OperatorTok{=} \FloatTok{0.1}
\NormalTok{    w2 }\OperatorTok{=} \FloatTok{0.5}
\NormalTok{    x[i}\OperatorTok{+}\FloatTok{1}\NormalTok{]}\OperatorTok{,}\NormalTok{ y[i}\OperatorTok{+}\FloatTok{1}\NormalTok{]}\OperatorTok{,}\NormalTok{ q[i}\OperatorTok{+}\FloatTok{1}\NormalTok{] }\OperatorTok{=}\NormalTok{ ddstep(x[i]}\OperatorTok{,}\NormalTok{ y[i]}\OperatorTok{,}\NormalTok{ q[i]}\OperatorTok{,}\FloatTok{1.0}\OperatorTok{,}\FloatTok{6.0}\OperatorTok{,}\NormalTok{dt}\OperatorTok{,}\NormalTok{w1}\OperatorTok{,}\NormalTok{w2)}
\NormalTok{    t }\OperatorTok{+=}\NormalTok{  dt}
\KeywordTok{end}

\NormalTok{display(plot(x}\OperatorTok{,}\NormalTok{y))}
\NormalTok{readline()}
\end{Highlighting}
\end{Shaded}

\hypertarget{listPlotExample}{%
\label{listPlotExample}}%
\begin{Shaded}
\begin{Highlighting}[]
\KeywordTok{using}\NormalTok{ Plots}
\NormalTok{N}\OperatorTok{=}\FloatTok{100}
\NormalTok{t0 }\OperatorTok{=} \FloatTok{0.0}
\NormalTok{t1 }\OperatorTok{=} \FloatTok{2.0}
\NormalTok{t }\OperatorTok{=} \DataTypeTok{LinRange}\NormalTok{(t0}\OperatorTok{,}\NormalTok{t1}\OperatorTok{,}\NormalTok{N)}
\NormalTok{dt }\OperatorTok{=}\NormalTok{ (t1}\OperatorTok{{-}}\NormalTok{t0)}\OperatorTok{/}\NormalTok{N}
\NormalTok{one }\OperatorTok{=}\NormalTok{ ones((N))}
\NormalTok{xp }\OperatorTok{=}\NormalTok{ zeros((N))}
\NormalTok{yp }\OperatorTok{=}\NormalTok{ zeros((N))}
\NormalTok{th }\OperatorTok{=}\NormalTok{ zeros((N))}

\NormalTok{x }\OperatorTok{=}\NormalTok{ t }\OperatorTok{.*}\NormalTok{ t}
\NormalTok{y }\OperatorTok{=}\NormalTok{ t}

\NormalTok{p }\OperatorTok{=}\NormalTok{ plot(x}\OperatorTok{,}\NormalTok{y}\OperatorTok{,}\NormalTok{ title }\OperatorTok{=} \StringTok{"Quadratic Path"}\NormalTok{)}
\NormalTok{display(p)}
\NormalTok{readline()}
\end{Highlighting}
\end{Shaded}

Generate wheel speeds:

\hypertarget{listwheelspeed}{%
\label{listwheelspeed}}%
\begin{Shaded}
\begin{Highlighting}[]
\KeywordTok{using}\NormalTok{ Plots}
\NormalTok{N}\OperatorTok{=}\FloatTok{100}
\NormalTok{t0 }\OperatorTok{=} \FloatTok{0.0}
\NormalTok{t1 }\OperatorTok{=} \FloatTok{2.0}
\NormalTok{t }\OperatorTok{=} \DataTypeTok{LinRange}\NormalTok{(t0}\OperatorTok{,}\NormalTok{t1}\OperatorTok{,}\NormalTok{N)}
\NormalTok{one }\OperatorTok{=}\NormalTok{ ones((N))}
\NormalTok{doty}\OperatorTok{=}\NormalTok{one}
\NormalTok{dotx}\OperatorTok{=}\FloatTok{2}\OperatorTok{*}\NormalTok{t}
\NormalTok{ddoty}\OperatorTok{=}\FloatTok{0}
\NormalTok{ddotx}\OperatorTok{=}\FloatTok{2}\OperatorTok{*}\NormalTok{one}

\NormalTok{r }\OperatorTok{=} \FloatTok{1.0}
\NormalTok{L }\OperatorTok{=} \FloatTok{4.0}
\NormalTok{v }\OperatorTok{=}\NormalTok{ sqrt.(dotx}\OperatorTok{.*}\NormalTok{dotx .}\OperatorTok{+}\NormalTok{ doty}\OperatorTok{.*}\NormalTok{doty)}
\NormalTok{kappa }\OperatorTok{=}\NormalTok{ (dotx}\OperatorTok{.*}\NormalTok{ddoty .}\OperatorTok{{-}}\NormalTok{ doty}\OperatorTok{.*}\NormalTok{ddotx)}\OperatorTok{./}\NormalTok{(v}\OperatorTok{.*}\NormalTok{v}\OperatorTok{.*}\NormalTok{v)}
\NormalTok{dotphi1 }\OperatorTok{=}\NormalTok{ (v}\OperatorTok{./}\NormalTok{r)}\OperatorTok{.*}\NormalTok{(kappa}\OperatorTok{*}\NormalTok{L .}\OperatorTok{+}\FloatTok{1}\NormalTok{)}
\NormalTok{dotphi2 }\OperatorTok{=}\NormalTok{ (v}\OperatorTok{./}\NormalTok{r)}\OperatorTok{.*}\NormalTok{(}\OperatorTok{{-}}\NormalTok{kappa}\OperatorTok{*}\NormalTok{L.}\OperatorTok{+}\FloatTok{1}\NormalTok{)}

\NormalTok{p}\OperatorTok{=}\NormalTok{plot(t}\OperatorTok{,}\NormalTok{[dotphi1}\OperatorTok{,}\NormalTok{dotphi2]}\OperatorTok{,}\NormalTok{title }\OperatorTok{=} \StringTok{"Wheel Speeds"}\OperatorTok{,}\NormalTok{label }\OperatorTok{=}\NormalTok{ [}\StringTok{"Right"} \StringTok{"Left"}\NormalTok{])}
\NormalTok{display(p)}
\NormalTok{readline()}
\end{Highlighting}
\end{Shaded}

And the section of code to check:

\hypertarget{listcheckwheelspeed}{%
\label{listcheckwheelspeed}}%
\begin{Shaded}
\begin{Highlighting}[]
\NormalTok{xp[}\FloatTok{1}\NormalTok{] }\OperatorTok{=} \FloatTok{0.0}
\NormalTok{yp[}\FloatTok{1}\NormalTok{] }\OperatorTok{=} \FloatTok{0.0}
\NormalTok{th[}\FloatTok{1}\NormalTok{] }\OperatorTok{=} \FloatTok{1.5707963267949}

\KeywordTok{for}\NormalTok{ i}\OperatorTok{=}\FloatTok{1}\OperatorTok{:}\NormalTok{N}\OperatorTok{{-}}\FloatTok{1}
\NormalTok{    xp[i}\OperatorTok{+}\FloatTok{1}\NormalTok{] }\OperatorTok{=}\NormalTok{ xp[i] }\OperatorTok{+}\NormalTok{ (r}\OperatorTok{*}\NormalTok{dt}\OperatorTok{/}\FloatTok{2.0}\NormalTok{)}\OperatorTok{*}\NormalTok{(dotphi1[i]}\OperatorTok{+}\NormalTok{dotphi2[i])}\OperatorTok{*}\NormalTok{cos(th[i])}
\NormalTok{    yp[i}\OperatorTok{+}\FloatTok{1}\NormalTok{] }\OperatorTok{=}\NormalTok{ yp[i] }\OperatorTok{+}\NormalTok{ (r}\OperatorTok{*}\NormalTok{dt}\OperatorTok{/}\FloatTok{2.0}\NormalTok{)}\OperatorTok{*}\NormalTok{(dotphi1[i]}\OperatorTok{+}\NormalTok{dotphi2[i])}\OperatorTok{*}\NormalTok{sin(th[i])}
\NormalTok{    th[i}\OperatorTok{+}\FloatTok{1}\NormalTok{] }\OperatorTok{=}\NormalTok{ th[i] }\OperatorTok{+}\NormalTok{ (r}\OperatorTok{*}\NormalTok{dt}\OperatorTok{/}\NormalTok{(}\FloatTok{2.0}\OperatorTok{*}\NormalTok{L))}\OperatorTok{*}\NormalTok{(dotphi1[i]}\OperatorTok{{-}}\NormalTok{dotphi2[i])}
\KeywordTok{end}

\NormalTok{plot(x}\OperatorTok{,}\NormalTok{y}\OperatorTok{,}\NormalTok{ label }\OperatorTok{=} \StringTok{"Original Path"}\NormalTok{ )}
\NormalTok{p }\OperatorTok{=}\NormalTok{ plot}\OperatorTok{!}\NormalTok{(xp}\OperatorTok{,}\NormalTok{yp}\OperatorTok{,}\NormalTok{  title}\OperatorTok{=}\StringTok{"Path"}\OperatorTok{,}\NormalTok{ label }\OperatorTok{=} \StringTok{"Robot Path"}\NormalTok{)}
\NormalTok{display(p)}
\NormalTok{readline()}








\NormalTok{The wheel velocities.}







\NormalTok{Comparison of the path and driven path.}
\end{Highlighting}
\end{Shaded}

On a robot, the motor controllers will be taking digital commands which
means the wheel velocities are discrete. This implies that the robot has
fixed wheel velocities during the interval between velocity updates. We
know in the case of the differential drive robot, fixed wheel speeds
means the robot is driving a line or circle. Therefor the DD robot in
this case is following a connected path made up of line or circle
segments, see \texttt{fig:piecewiseconst}. Even when we do have
functional forms for the wheel speeds, the implementation is still
discrete.

It makes sense to treat this as a discrete formula and to write as such:

\[\begin{aligned}
\boxed{
\begin{array}{l}
v_k = \sqrt{\dot{x}(t_k)^2 + \dot{y}(t_k)^2} , \quad\quad
\displaystyle  \kappa_k = \frac{\dot{x}(t_k) \ddot{y}(t_k ) -  \dot{y}(t_k) \ddot{x}(t_k)}{v_k^3}, \\[3mm]
\displaystyle  \omega_{1,k} = \frac{v_k}{r}(\kappa_k L + 1), \quad\quad
\displaystyle  \omega_{2,k} = \frac{v_k}{r}(-\kappa_k L + 1)
\end{array} }
\end{aligned}\]

Determine the wheel velocities to drive through the way points (0,1),
(1,2), (2,5). Recall that we found an interpolating polynomial in the
Lagrange Interpolation section. This provided us with the following:

\[x(t) = -t(t-2) + t(t-1)  = t,\]

\[y(t) =  \frac{1}{2} (t-1)(t-2) - 2t(t-2) + \frac{5}{2} (t)(t-1)= t^2+1 .\]

Next we compute the derivatives

\[\dot{x} = 1,\quad \ddot{x} = 0,\quad \dot{y} = 2t,\quad \ddot{y} = 2\]

and then plug into the equations

\[\kappa = \frac{(1)(2) - (2t)(0)}{\left(1 + 4t^2\right)^{3/2}} = \frac{2}{\left(1 + 4t^2\right)^{3/2}} ,\]

\[\dot{\phi_1} =  \frac{v}{r}\left( \kappa L + 1\right) , \quad \dot{\phi_2} = \frac{v}{r}\left( - \kappa L  + 1\right).\]

\hypertarget{limitations}{%
\subsection{Limitations}\label{limitations}}

In the previous sections we have shown how to drive a robot along any
path that the kinematics admits. In the mathematical examples, there are
no problems with following a precomputed path. However, this is an
example of open loop control and it suffers from many types of error
such as discretization error, non-uniform components, variations in
power, signals and an unpredictable environment. The robot will drift
from the intended path. This drift grows over time.

In practice, we will normally not compute the analytic path from which
to compute the derivatives and such to plug into the inverse kinematics.
We will use more traditional control algorithms to direct the robot such
as a PID controller. We may have a path to follow, but we will not plug
that path into the inverse kinematics. Instead we will extract samples
from the path and feed destination points into the control algorithm.
This does not mean that our efforts working out the inverse kinematics
was wasted. Very much to the contrary. We will still use the IK formulas
in our controllers. Understanding the IK will help in the controller
design. The IK can often help isolate aspects of the system dynamics
which eases controller development or makes it possible to gain a stable
controller.

\textbf{Footnotes}
