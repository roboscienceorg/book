\hypertarget{problems}{%
\section{Problems}\label{problems}}

\begin{enumerate}
\tightlist
\item
  Provide the name and circuit diagram for the following:

  \begin{enumerate}
  \tightlist
  \item
    The circuit element that allows current to flow in one direction.
  \item
    The device that can boost or reduce AC voltage sources.
  \item
    The basic elements that store energy in electrical or magnetic
    fields.
  \item
    The device that reroutes current from ac to dc.
  \item
    The device that prevents oscillations in mechanical switch circuits.
  \end{enumerate}
\item
  Provide a labeled circuit/hardware diagram for a system that has a
  microcontroller driving a brushed DC motor (based on motor encoder
  output so that the system can control the actual speed and direction
  of the motor using a simple feedback based control loop). Assume that
  the stall current of the motor and the motor operating voltages are
  well in excess of what the microcontroller can source. {[}Note: Your
  microcontroller has PWM, GPIO, I2C, UART, lines for this
  application.{]} Explain each part of the diagram.
\item
  Assume that you can provide input for a motor controller in terms of
  percent of duty cycle (0-100): u. Also assume that at a 10 Hz rate you
  get a reading from an encoder that provides the output rpm of the
  wheel. Write a function that controls the rpm (range is 0 to 350),
  based on the value in u.
\item
  What is a PWM signal?
\item
  Can you think of a circuit to accept DC power which could hook up to
  the batteries either way. {[}Meaning that if the user hooks up the
  wires backwards, it automatically still works{]}.
\item
  In an H bridge, are there switch combinations that cause problems? Why
  or why not?
\item
  When robots are rolling down a hill, the electric motors can act as
  generators. The current generated may damage the motor controllers. Is
  there a design that might be able to route the generated power to an
  on-board battery charger? Provide a circuit.
\end{enumerate}
