\hypertarget{configuration-space-and-reach-for-simple-vehicles}{%
\section{Configuration space and reach for simple
vehicles}\label{configuration-space-and-reach-for-simple-vehicles}}

In this section, we determine the reach (and time limited reach) of the
robot from a given point and the possible paths between two points. Does
the reach cover the plane or are there some points in the plane which
cannot be reached? First, we make precise what is meant by reach
\texttt{lavalle2006}. Let \(X\) be the state space,
\({\cal U} \subset X\) be the set of all permissible trajectories on
\([0,\infty)\) and \(R(q_0,{\cal U} )\) denote the reachable set from
\(x_0\).

We define the reachable set as

\[R(x_0,{\cal U} ) = \left\{  x_1 \in X | \exists \tilde{u}\in {\cal U} \mbox{ and } \exists t \in [0,\infty) \mbox{ s.t. } x(t) = x_1 \right\}\]

Let \(R(q_0,{\cal U} ,t)\) denote the time-limited reachable set from
\(x_0\).

We define the time-limited reachable set as

\[R(x_0,{\cal U},t ) = \left\{ x_1 \in X | \exists \tilde{u}\in {\cal U} \mbox{ and } \exists \tau \in [0,t] \mbox{ s.t. } x(\tau) = x_1 \right\}\]

The Dubins Car, \texttt{dubins}, is a vehicle that can move straight
forward or turn at any curvature up to some maximum curvature. This
vehicle provides a geometric motion model for automobiles and can be
used to understand basic optimal path planning. The Reeds-Shepps Car,
\texttt{reeds}, extends the Dubins vehicle to include reverse motion.
This greatly enhances maneuverability. Small back and forth motions can
realign a vehicle to a new orientation. This means if the robot arrives
at a destination point with the wrong orientation, it can be corrected
locally (assuming sufficient room about the point).

Dubins showed that a vehicle which can go only forward and turn at any
curvature up to some maximum curvature can reach any point in the plane
in the absence of obstacles \texttt{dubins}. Optimality of solutions is
discussed in \texttt{kelly2013mobile}, \texttt{lavalle2006}. A slight
generalization is given in \texttt{reeds} for a car that can go forwards
and backwards. In \texttt{reeds}, \texttt{sussman},
\texttt{lavalle2006}, it is shown that optimal solutions are piecewise
collections of line segments and maximum curvature circles. Since the
DDD (dual differential drive) and FWS (four wheel steer) designs have
less restrictive motion, we can answer the reach question. The entire
plane can be covered. The question of optimal paths will be left for a
future study. The FWS system we have built is targeted for an
environment filled with obstacles. Our main concern is reach in the
presence of obstacles, for which the reach and the optimal path results
for Dubins and Reeds-Shepps are no longer valid.

Both the DDD and FWS designs are more maneuverable than the Dubins
vehicle, and so we expect more flexibility in dealing with obstacles.
The time limited reach of the Dubins Car is the forward fan seen in
\texttt{fig:fmotion} and the time limited reach of the Reeds-Shepps car
is the open set about the initial point \texttt{lavalle2006}. Since both
the DDD and FWS systems include the motion patterns found in the
Reeds-Shepps car, the time limited reach for these two designs is an
open set about the initial point: there exists a set \(U\), open, such
that \(U \subset R(x_0,{\cal U},t )\). This is possible due to the
ability to perform back and forth maneuvers like that found in parallel
parking.

\hypertarget{rigid-motion}{%
\subsection{Rigid Motion}\label{rigid-motion}}

The FWS can move from point to point and then adjust orientation as
required. If there exists a path between two points, the FWS axle can
traverse the path via the waypoints, re-orient at each point and reach
the goal location. Thus it can follow a piecewise linear path between
two configuration space locations. A smooth path can be found by using a
b-spline and if curvature exceeds the maximum bound, the vehicle can
stop, re-orient and then continue. Traversal is possible if the start
and goal locations are path connected and that path locations with
curvature above \(R\) have a disk of radius \(r\) centered at the path
point which does not intersect any obstacle.

The DDD design has additional constraints compared to the FWS design.
The solution that \texttt{reeds}, \texttt{sussman}, \texttt{lavalle2006}
suggest is to perform a series of short adjustment maneuvers as seen in
\texttt{fig:deltatheta}. Although the results for re-orientation can be
applied to arbitrarily small robots and adjustment regions, in practice
for a given robot or vehicle, the region has some minimum size. Assume
that the adjustment maneuvers falls in a circle of radius \(r\). Let
\(W\) be a bounded domain in \({\Bbb R}^2\), the obstacles be
\({\cal O}_i\) and the free space be given by
\(\Omega = W\setminus \cup_{i}{\cal O}_i\).

\begin{quote}
A series of short adjustment maneuvers to re-orient the vehicle.
\end{quote}

For simplicity here, we assume the domain satisfies a traversability
condition. Let \(D(x,r)\) be the disk of radius \(r\) centered at \(x\).
\(\Omega\) is said to be disk traversable if for any two points
\(x_0,x_1 \in \Omega\), there exists a continuous function
\(p(t)\in{\Bbb R}^2\) and \(\epsilon >0\) such that
\(D(p(t),\epsilon)\subset\Omega\) for \(t\in [0,1]\) and \(x_0=p(0)\),
\(x_1=p(1)\). Note that \(p(t)\) generates the curve \(C\) which is a
path in \(\Omega\) and the path is a closed and bounded subset of
\(\Omega\). Navigation along jeep trails, bike trails and large animal
trails (in our case, Cattle and Bison) produces small corridors though
the forest. Along these tracks there is a corridor produced which we
describe as disk traversable.

\leavevmode\hypertarget{disktraverseDDD}{}%
\textbf{Traversability Theorem:} If \(\Omega\) is disk traversable, then
the DDD and FWS vehicles can navigate to the goal ending with the
correct orientation. \textbf{Proof:} See
\texttt{Appendix\ \textless{}appendix\textgreater{}}.

\hypertarget{the-piano-movers-problem---orientation}{%
\section{The Piano Movers Problem -
Orientation}\label{the-piano-movers-problem---orientation}}

Assume you want to route an object with a complicated shape through a
tight sequence of corridors. Routing a complex shape through a narrow
passage is often referred to as the \texttt{piano\ movers\ problem}.
Take a simple example, move the linear robot through the two blocks,
\texttt{robotmustrotate}. It is clear to the human what has to happen.
The robot must rotate. For a holonomic robot, this simply means the
controller issues a rotation command while traveling to the corridor.
For a non-holonomic robot, the control system must change the path so
that upon entry and through the corridor the robot's orientation will
allow for passage. A significant problem arises if the corridor is
curved in a manner that is not supported by the possible orientations
defined by the vehicle dynamics. In plain English, this is when you get
the couch stuck in the stairwell trying to move into your new flat.

\begin{quote}
The object must rotate to fit through the open space.
\end{quote}

As all of us learned when we were very young, we must turn sideways to
fit through a narrow opening.\footnote{Cavers will tell you that you can
  crawl through a vertical gap spanned by the distance of your thumb and
  your fifth (pinky) finger. For the average American, this is a very
  small gap.} This introduces a new aspect to routing, that of
reconfiguration of the robot. Examine a simple reconfiguration which is
simply a change in orientation. As we saw above, each rotation of the
robot induces a different configuration space. \texttt{robotrotation}
shows the idea for three different rotation angles, there are three
different configuration obstacle maps.

\begin{quote}
Different rotations produce different obstacle maps in configuration
space.
\end{quote}

Since each rotation generates a two dimensional configuration space,
they can be stacked up in three dimensions. So we have that
configuration space includes the vertical dimension which is the
rotation angle for the robot - the configuration space is three
dimensional. To restate, the configuration space includes all of the
configuration variables \((x,y, \theta)\) is now a three dimensional
configuration space which is shown in \texttt{robotrotation3D}. So,
although the workspace is two dimensional, the configuration space is
three dimensional and are different objects.

\begin{quote}
The different rotations can be stacked where the vertical dimension is
the rotation angle.
\end{quote}

For a three dimensional object with a fixed orientation, would have a
three dimensional configuration space. For toolheads, only pitch and yaw
matter. To locate a point on a sphere you need two variables (think
about spherical coordinates): \(\theta\) the angle in the \(x\)-\(y\)
plane and \(\phi\) the angle from the \(z\) axis (or out of the plane if
you prefer). For each pair \((\theta, \phi)\) we have a 3D section. This
tells us that the configuration space is five dimensional. When roll,
pitch and yaw all matter then we have a 6 dimensional configuration
space. If the robot is configurable with other elements, then each
parameter defining the configuration would also add a variable to the
mix and increase the dimension of the configuration space.

The construction of configuration space then is built like slices in a
3D printer. Routing or path planning must be done in the full
configuration space. For the current example, we must route in 3D which
will translate to position and orientation routing in the workspace,
\texttt{obst4}.

\begin{quote}
We can see that there is a path that includes the rotation.
\end{quote}

\hypertarget{two-link-arm-revisited}{%
\subsection{Two Link Arm Revisited}\label{two-link-arm-revisited}}

Articulated (multilink) robot arms also have size and orientation.
Determining which configurations and which physical positions are
actually realizable is more complicated. The size of the robot arm will
affect the regions which the end effector can reach but obstacle
inflation does not give the same workspace. The end effector is designed
to touch an object and from that perspective little inflation is
required. However the base link of the arm might be very wide and does
affect the useable workspace. A simple obstacle inflation approach will
not work with manipulators. The reason is that how you travel affects
your reach. \texttt{Fig:pathmatters} shows how the path matters to
access. A more situation can be found in \texttt{Fig:nopaththrough}.
Even though the articulator is small enough to pass through the gap, it
cannot due to the other physical restrictions.

\begin{quote}
The elbow down approach is blocked, but not the elbow up position.

Neither configuration of the robot arm can reach the point.
\end{quote}

\hypertarget{appendix}{%
\section{Appendix}\label{appendix}}

The proof for the
\texttt{Traversability\ Theorem~\textless{}disktraverseDDD\textgreater{}},
statement reproduced below, is given here.

If \(\Omega\) is disk traversable, then the DDD and FWS vehicles can
navigate to the goal ending with the correct orientation.

\textbf{Proof:} Let \(C\) be the path from \(x_0\) to \(x_1\). At each
point of the path there exists an open disk of radius \(\epsilon\) which
does not intersect an obstacle. The intersection of the curve \(C\) with
the open disk induces an open set in \(C\). The collection of open sets
is an open cover of the curve \(C\). Since the curve is a closed and
bounded set, and thus compact, there is a finite subcover of open
intervals \texttt{munkres2000topology}. These correspond to a finite set
of open disks which cover the path. The vehicle may travel a straight
line from disk center to disk center. At each center the vehicle may
reorient if required. The time limited reach for the DDD drive is a
proper subset of the FWS reach, and follows from the DDD result.

\textbf{Footnotes}
