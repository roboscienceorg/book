\hypertarget{problems}{%
\section{Problems}\label{problems}}

\begin{enumerate}
\item
  Write a Python function to compute wheel angles in the Ackerman system
  given the desired vehicle turn angle and frame parameters. This the
  function should have arguments {(theta, l1, l2)} and function should
  return the two wheel angles {(theta\_l, theta\_r)}.
\item
  What are the motion equations for the Ackerman drive? {[}Meaning
  forward and angular velocity as a function of wheel speed.{]} Assume
  wheel radius is \(r\).
\item
  A dual Ackerman drive would steer both front and rear wheels using an
  Ackerman steering approach. What would the pros and cons for this
  system compared to a single Ackerman drive?
\item
  Assume that you have a rectangular Mechanum robot with
  \(L_1 = 0.30\)m, \(L_2 = 0.20\)m and \(r=0.08\)m. Find the path of the
  robot for the given wheel rotations:
  \(\dot{\phi}_1 = 0.75*\cos(t/3.0)\),
  \(\dot{\phi}_2 = 1.5*cos(t/3.0)\), \(\dot{\phi}_3 = -1.0\),
  \(\dot{\phi}_4 = 0.5\). Start with \(x, y, \theta = 0\) and set
  \(t=0\), \(\Delta t = 0.05\). Run the simulation for 200 iterations
  (or for 10 seconds). Keeping the x and y locations in an array is an
  easy way to generate a plot of the robot's path. If x, y are arrays of
  x-y locations then try

\begin{verbatim}
import pylab as plt
plt.plot(x,y,'b.')
plt.show()
\end{verbatim}

  Showing the orientation takes a bit more work. Matplotlib provides a
  vector plotting method. You need to hand it the location of the vector
  and the vector to be plotted, \((x,y,u,v)\), where \((x,y)\) s the
  vector location and \((u,v)\) are the x and y components of the
  vector. You can extract those from \(\theta\) using
  \(u = s*\cos(\theta)\) and \(v = s*\sin(\theta)\) where \(s\) is a
  scale factor (to give a good length for the image, e.g. 0.075). The
  vector plot commands are then

\begin{verbatim}
plt.quiver(u,v,c,s,scale=1.25,units='xy',color='g')
plt.savefig('mecanumpath.pdf')
plt.show()
\end{verbatim}
\item
  Real motion and measurement involves error and this problem will
  introduce the concepts. Assume that you have a differential drive
  robot with wheels that are 20cm in radius and L is 12cm. Using the
  differential drive code (forward kinematics) from the text, develop
  code to simulate the robot motion when the wheel velocities are
  \(\dot{\phi}_1 = 0.25t^2\), \(\dot{\phi}_2 = 0.5t\). The starting
  location is {[}0,0{]} with \(\theta = 0\).

  \begin{enumerate}
  \def\labelenumii{\alph{enumii}.}
  \item
    Plot the path of the robot on \(0\leq t \leq 5\). It should end up
    somewhere near {[}50,60{]}.
  \item
    Assume that you have Gaussian noise added to the omegas each time
    you evaluate the velocity (each time step). Test with \(\mu = 0\)
    and \(\sigma = 0.3\). Write the final location (x,y) to a file and
    repeat for 100 simulations. Hint:

\begin{verbatim}
mu, sigma = 0.0, 0.3
xerr = np.random.normal(mu,sigma, NumP)
yerr = np.random.normal(mu,sigma, NumP)
\end{verbatim}
  \item
    Generate a plot that includes the noise free robot path and the
    final locations for the simulations with noise. Hint:

\begin{verbatim}
import numpy as np
import pylab as plt
...
plt.plot(xpath,ypath, 'b-', x,y, 'r.')
plt.xlim(-10, 90)
plt.ylim(-20, 80)
plt.show()
\end{verbatim}
  \item
    Find the location means and 2x2 covariance matrix for this data set,
    and compute the eigenvalues and eigenvectors of the matrix. Find the
    ellipse that these generate. {[}The major and minor axes directions
    are given by the eigenvectors. Show the point cloud of final
    locations and the ellipse in a graphic (plot the data and the
    ellipse). Hint:

\begin{verbatim}
from scipy import linalg
from matplotlib.patches import Ellipse
s = 2.447651936039926
#  assume final locations are in x & y
mat = np.array([x,y])
#  find covariance matrix
cmat = np.cov(mat)
# compute eigenvals and eigenvects of covariance
eval, evec = linalg.eigh(cmat)
r1 = 2*s*sqrt(evals[0])
r2 = 2*s*sqrt(evals[1])
#  find ellipse rotation angle
angle = 180*atan2(evec[0,1],evec[0,0])/np.pi
# create ellipse
ell = Ellipse((np.mean(x),np.mean(y)),r1,r2,angle)
#  make the ellipse subplot
a = plt.subplot(111, aspect='equal')
ell.set_alpha(0.1)    #  make the ellipse lighter
a.add_artist(ell)   #  add this to the plot
\end{verbatim}
  \end{enumerate}
\item
  Describe the different styles of Swedish wheel.
\item
  Find the analytic wheel velocities and initial pose for a Mecanum
  robot tasked to follow (\(r=3\), \(L_1 = 10\), \(L_2=10\) all in

  cm) the given paths (path units in m). Plot the paths and compare to
  the actual functions to verify.

  \begin{enumerate}
  \def\labelenumii{\alph{enumii}.}
  \tightlist
  \item
    \(y=(3/2)x + 5/2\)
  \item
    \(y = x^{2/3}\)
  \end{enumerate}
\item
  What are the wheel velocity formulas for a four wheel Mechanum robot,
  (\(r=3\), \(L_1 = 10\), \(L_2=10\) all in cm) which drives in the
  circular path \((x-3)^2/16 + (y-2)^2/9 = 1\) and always faces the
  center of the circle.
\item
  In Veranda, drive a Mecanum robot along a square with corners (0,0),
  (10,0), (10,10), (0,10), \(L_1 = 0.30\), \(L_2 = 0.20\) and
  \(r=0.08\). You should stop and ``turn'' at a corner, but keep the
  robot faced in the x-axis direction. Drive the edges at unit speed.
  Use a video screen capture program to record the results.
\item
  In Veranda, drive the Mecanum robot in an infinity (\(\infty\)) shape.
  Use a video screen capture program to record the results.
\end{enumerate}
