\hypertarget{problems}{%
\section{Problems}\label{problems}}

\begin{enumerate}
\tightlist
\item
  Using ROS and Python, write a chat program (call it \emph{chat.py}).
  First prompt the user for their name. Write to all members in the chat
  group that this person has entered the chat. In a loop, grab user
  inputs and broadcast to the chat with format: name: \textless user
  input\textgreater{} . Echo to the terminal all strings sent to the
  chat.
\item
  Using ROS and Python, modify the example programs in the text on the
  kinematics of the two link manipulator.

  \begin{enumerate}
  \tightlist
  \item
    Write a program that creates a list of 100 equally spaced points
    along the path \(y = 15 -  x\) for \(0 \leq x \leq 10\) and
    publishes those points on the topic /physData using a multiarray
    floating data type, i.e. values x and y are floats. Publish the data
    at 5Hz.
  \item
    Write a program that subscribes to topic /physData, plugs the values
    in, computes the serial two link inverse kinematics to gain the
    servo angles (pick one of the +/-) and publishes the angles to the
    topic /thetaData. You may assume the link arms are \(a_1=a_2 = 10\).
    Format will be the same as the previous topic.
  \item
    Write a program that subscribes to both /physData and /thetaData.
    The program should plug the angles into the forward kinematics and
    check against the data in /physData. It should plot the original
    curve in green and the ``check'' in blue.
  \end{enumerate}
\item
  Assume that you have a parallel two link manipulator with
  \(L_0 = 10\)cm, \(L_1 = 15\)cm and \(L_2 = 20\)cm.

  \begin{enumerate}
  \tightlist
  \item
    Write a ROS program that creates a list of 100 equally spaced points
    along the path \(x = 7\cos(t)+10\), \(y = 5\sin(t) + 15\) and
    publishes those points on the topic /physData using a multiarray
    floating data type, i.e. values x and y are floats. Publish the data
    at 5Hz.
  \item
    Write a ROS program that subscribes to topic /physData, plugs the
    values in, computes the serial two link inverse kinematics to gain
    the servo angles and publishes the angles to the topic /thetaData.
    Format will be the same as the previous topic.
  \item
    Write a ROS program that subscribes to both /physData and
    /thetaData. The program should plug the angles into the forward
    kinematics and check against the data in /physData. It should plot
    the original curve in green and the ``check'' in blue.
  \end{enumerate}
\item
  Using ROS and python write a program that will add padding to
  obstacles while shrinking the footprint of the robot to a point.
  Assume that you have a circular robot with radius 10 and starting pose
  (15,15,90).

  \begin{enumerate}
  \tightlist
  \item
    Write a program that will publish the pose of the robot on the topic
    \texttt{/robot/pose} and the footprint type of the robot on
    \texttt{/robot/footprint} as a string (For example circle or
    polygon). Also publish the radius of the robot on
    \texttt{/robot/radius} as a uint16 message type.
  \item
    Write a program that will publish a list of obstacles as polygons on
    the topic \texttt{/obstacles}. For this program let the obstacles be
    the following:

    \begin{enumerate}
    \tightlist
    \item
      Rectangle with the vertices (40,30), (50,5), (50, 30) (40,30).
    \item
      Rectangle with the vertices (40,5), (50,5), (50,0), (40,5).
    \end{enumerate}
  \item
    Write a program that subscribes to \texttt{/robot/pose},
    \texttt{/robot/footprint}, and \texttt{/obstacles}. Based on the
    footprint string, this program should be able to subscribe to either
    the robot radius or dimension topics for circular and rectangular
    robots. This program will reduce the robot footprint to a point, add
    padding to the obstacles, and plot the robot as a point and padded
    obstacles with the maximum x and y values being 70 and 30.
  \end{enumerate}
\item
  Rework the previous problem assuming that you have a rectangular robot
  with \(width=10\) and \(length=20\) and initial pose (0,10,0).
\item
  Plot the padding of obstacles using the ros nodes in the previous
  problem with the initial pose of the robot being (a) (5,10,30), (b)
  (5,10,70), and (c) (5,25,-90).
\item
  Write a program that will publish the changing poses of a rectangular
  robot over time from to in increments of . Assume the inital pose is
  (5,10, -90) and \(width=10\) and \(length=20\). Use the programs from
  problem 6 to publish the obstacles and display the padding.
\end{enumerate}
