\hypertarget{problems}{%
\section{Problems}\label{problems}}

\begin{enumerate}
\tightlist
\item
  How would you define a robot?
\item
  What are the two main types of robots as presented by the text?
\item
  Can you think of another way to classify robotic systems? What are the
  strengths and weaknesses of the classification?
\item
  What problems does a mobile robot face that a stationary robot does
  not? What about the other way around?
\item
  What are the differences between robots that are considered Mobile
  Machines and those that are considered to be Manufacturing Machines?
\item
  Do you think the \emph{Robotic Appliance} and \emph{Robotic Agent}
  partitioning is a more effective way to classify robots? Why or why
  not?
\item
  List several approaches that industry has used so robots can navigate
  in an environment; mentioning an advantage and disadvantage of each.
\item
  Describe the information gathered by a RGBD sensor such as the
  Microsoft Kinect.
\item
  Describe the information gathered by a stereo camera pair.
\item
  List out different ways one could assist a robot in navigating around
  a building or inside a building when GPS is not an option. {[}Think
  sensors.{]}
\item
  What is an FPGA? What is a GPU? What is a TPU? What are their
  strengths and weaknesses compared to traditional CPUs?
\item
  What are Isaac Asimov's Three Laws of Robotics? What do they mean? Are
  they complete, meaning are they a sufficient set of rules?
\item
  Work in robotics can replace people with machines. This results in job
  loss. Discuss the ethics of working in the robotics industry.
\item
  Military robotics is a growing industry. Although many systems have a
  high degree of autonomy, use of deadly force is left for the human.
  Discuss the ethical issues in allowing the robot to make these
  decisions.
\item
  If an autonomous system by design or error causes an accident, who is
  liable?
\item
  List a few ways biology has inspired robotics.
\end{enumerate}
