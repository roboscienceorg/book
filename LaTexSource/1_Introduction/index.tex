\begin{quote}
mainmatter lfoot{[}thepage{]}\{rightmark\}
rfoot{[}leftmark{]}\{thepage\}
\end{quote}

\hypertarget{introduction}{%
\section{Introduction}\label{introduction}}

Growing up in the modern age means many things to many people. For those
of us reading (or writing) a book on robotics, it means getting a
healthy dose of technology. Hollywood has raised us on spaceships,
cloning, alien worlds and intelligent machines. This was, however, not
incredibly unrealistic as progress in science and technology has been
advancing at an ever increasing rate. We have come to expect something
new, maybe even dramatic, every single year...and for the most part
haven't been disappointed! Over the years we have seen significant
developments in medicine, space, electronics, communications and
materials. There has always been excitement regarding the latest
development. Even though the world around us has been struggling with
war, poverty and disease, science and technology offers us a reprieve.
It is an optimistic view that things can and will get better. In full
disclosure - we love technology! It is the magic of our age and learning
how the magic works only makes it more fun. In no way does the author
claim that technology is the answer to our problems. That clearly lies
with our willingness to look beyond our differences with acceptance,
compassion and grace. If technology can bring us together, then it has
succeeded in helping us far beyond our wildest dreams.

\texttt{Robotics} is a shining example of technical optimism - a belief
that the human condition can be improved through sufficiently advanced
technology. It is the premise that a machine can engage in the
difficult, the tedious, and the dangerous; leaving humans out of harms
way. Technology is a fancy word for tool utilization. Even though
biologists have long shown that humans were not alone in their usage of
tools, we are indisputably the master tool users on the planet. Tools
extend our grasp, our strength and our speed. We know of no technology
that aims to extend human capability like robotics does.

Humans have an insatiable curiosity, a drive to create, and a
considerable amount of self-interest. Building machines which look like
us, act like us and do things like us was the engineering manifest
destiny. Although we have succeeded in building machines that do
complicated tasks, we really place the value in what we learn about
ourselves in the process. A process we embark on here.

1\_whatisrobot 2\_overview 3\_fundamental 4\_LastWords
5\_Introduction\_Problems
