\hypertarget{stdr---simple-two-dimensional-robot-simulator}{%
\section{STDR - Simple Two Dimensional Robot
Simulator}\label{stdr---simple-two-dimensional-robot-simulator}}

Note

Update for new text. Keep as ROS 1 material.

Earlier you installed and tested the STDR simulator. Now we will use it
to simulate a robot moving around in plane. The same ROS
publish-subscribe interface is used here. As before, you write a control
program that publishes motion commands to the STDR Simulator that you
loaded earlier. The simulator currently implements a ``magic round
robot'' that can move freely in any direction. This is not like your
automobile which can only move forward (and turn in a specific
manner).~\footnote{Although this may seem completely made up, we will
  see in later chapters that there are robots that have this type of
  motion.} The robot can be driven by using the examples in chapter one
with the joy and keyboard teleop nodes when the STDR simulator was
initially downloaded and built. There are other ways the robot can be
driven around which we demonstrate in this section; using either a
multiarray message or the twist message that contains kinematic
parameters.

\begin{quote}
STDR Simulator.
\end{quote}

The simulator takes velocity commands in the \(x\) and \(y\) directions
and moves the robot with those velocities. This allows for any sort of
robot to be simulated by having an external node handle the specific
robot's kinematics. So then the sim does not need to preprogram all the
different popular styles of robots. The sim subscribes to a Twist
message (discussed below in the messages section) containing the robot
velocities. It will perform the time steps (integrations) to move the
robot. It is important that the user provides accurate velocity commands
based on the wheels and vehicle design.

\begin{quote}
STDR Communications
\end{quote}

To get you up and running, we have provided a differential drive robot
node which will convert wheel commands to correct robot velocities based
on the differential drive kinematics. First, we show you how to run the
simulator. Following that we demonstrate how to move the wheels (to move
the robot).

\hypertarget{running-stdr}{%
\subsection{Running STDR}\label{running-stdr}}

In order to run the STDR simulator the user will need to run roslaunch
in order for it to be started with both the map and robot. For example,
to start the simulator with the robot and a map containing no obstacles
one would run the following inside of a terminal:

\begin{verbatim}
roslaunch stdr_launchers no_obst_sim.launch
\end{verbatim}

The roslaunch command does use tab completion so other launch files are
also accessible that will start up the simulator and all required nodes
to start simulation of the robot. There are other launch files that
include different maps and robots. These launch files will be named so
that the user can easily tell which map and kinematic model that the
robot will be using.

For example:

\begin{verbatim}
roslaunch stdr_launchers omni_wheeled_no_obst_sim.launch
roslaunch stdr_launchers diff_drive_no_obst_sim.launch
\end{verbatim}

Once roslaunch executes the user will be greeted by an application
looking similar to the one in \texttt{fig:stdr\_sim}. Roslaunch also
starts up the ros master if there isn't one already running on the
machine. It also starts up all the nodes and they can be viewed in
\texttt{fig:stdr\_node\_graph}. This graph shows the ROS nodes running
for just the STDR simulator with the DDFK node and not any control code
you may wish to run. So the actual node complexity is a bit more than
what \texttt{fig:stdr\_basic} implies since the STDR node is really a
placeholder for the graph shown in \texttt{fig:stdr\_node\_graph}.

\begin{quote}
Nodes running after STDR Simulator launch but before you launch your
control code.
\end{quote}

The message topics also get started and can be viewed by doing a
rostopic list. It is an extensive list and provides a look under the
hood for the simulator. While getting started you will not need to
interact with these topics, but later when we are working with sensors,
you will need to subscribe to some of the sensor topics.

\begin{verbatim}
/map
/map_metadata
/robot0/cmd_vel
/robot0/dt
/robot0/laser_0
/robot0/odom
/robot0/pose2D
/robot0/sonar_0
/robot0/sonar_1
/robot0/sonar_2
/robot0/sonar_3
/robot0/sonar_4
/rosout
/rosout_agg
/stdr_server/active_robots
/stdr_server/co2_sources_list
/stdr_server/delete_robot/cancel
/stdr_server/delete_robot/feedback
/stdr_server/delete_robot/goal
\end{verbatim}

\begin{verbatim}
/stdr_server/delete_robot/result
/stdr_server/delete_robot/status
/stdr_server/register_robot/cancel
/stdr_server/register_robot/feedback
/stdr_server/register_robot/goal
/stdr_server/register_robot/result
/stdr_server/register_robot/status
/stdr_server/rfid_list
/stdr_server/sound_sources_list
/stdr_server/sources_visualization_markers
/stdr_server/spawn_robot/cancel
/stdr_server/spawn_robot/feedback
/stdr_server/spawn_robot/goal
/stdr_server/spawn_robot/result
/stdr_server/spawn_robot/status
/stdr_server/thermal_sources_list
/tf
/tf_static
\end{verbatim}

\hypertarget{driving-the-robot---ros-stdr-messages}{%
\subsection{Driving the Robot - ROS STDR
Messages}\label{driving-the-robot---ros-stdr-messages}}

Once the simulator is up and running, you can drive the robot as before
using the teleop or joystick controls. As mentioned above, we can write
our own node to control the robot. This node needs to publish to either
the differential drive forward kinematics or directly to the simulator.
For simulating a differential drive, you will need to write a wheel
control node such as the example below which publishes left and right
wheel velocities. That node then coverts those to robot velocities and
sends the information to the STDR simulator. For your own custom robot,
you will need to write a forward kinematics node which connects to the
simulator. You would then send wheel velocities to your custom FK node.

\hypertarget{multiarray}{%
\subsubsection{MultiArray}\label{multiarray}}

In order to drive the robot around in the simulator for a differential
drive robot, the wheel velocities, wheel radius, and the axle length are
needed to be published on the \texttt{/kinematic\_param} topic as an
tuple containing four values.

The Python MultiArray is implemented as a tuple. A tuple is similar to a
list but not mutable like lists. They are distinguished from lists by
the use of parenthesis instead of brackets.

\begin{verbatim}
>>> # tuple
...
>>> a = (1,2,3)
>>> a[0]
1
>>> a[1]
2
>>> a[1] = 4
Traceback (most recent call last):
  File "<stdin>", line 1, in <module>
TypeError: 'tuple' object does not support item assignment
>>>
\end{verbatim}

\begin{verbatim}
import rospy
from math import *
import numpy as np
from std_msgs.msg import Float64MultiArray
from std_msgs.msg import MultiArrayLayout
from std_msgs.msg import MultiArrayDimension
r = 2.0
l = 3.0
def talker(w1, w2, r, l):
    pub = rospy.Publisher('kinematic_params', Float64MultiArray, queue_size=1)
    rospy.init_node('talker', anonymous=True)
    rate = rospy.Rate(10) # 10hz
    layout = MultiArrayLayout()
    layout.dim.insert(0, [MultiArrayDimension()] )
    while not rospy.is_shutdown():
        data = Float64MultiArray(data=[])
        data.layout = MultiArrayLayout()
        data.layout.dim = [MultiArrayDimension()]
        data.layout.dim[0].label = "Parameters"
        data.layout.dim[0].size = 4
        data.layout.dim[0].stride = 1
        data.data = [w1,w2,r,l]
        pub.publish(data)
        rate.sleep()

if __name__ == '__main__':
        try:
            talker(1.5,1.0,r,l)
        except rospy.ROSInterruptException:
            pass
\end{verbatim}

Similarly for a omni wheel robot the four wheel velocities would be
published followed by the wheel radius, front axle length, and lastly
the back axle length.

\hypertarget{twist-message}{%
\subsubsection{Twist Message}\label{twist-message}}

Communication with the simulator is through a ROS topic using the Twist
message type. The twist message is a compact array format that can be
more efficient than the string format used in the Two Link Manipulator.
The Twist format is

\begin{verbatim}
# This expresses velocity in free space broken into its  linear and angular parts.
Vector3  linear
Vector3  angular
\end{verbatim}

The twist message is contained in the geometry package:

\begin{verbatim}
from geometry_msgs.msg import Twist
\end{verbatim}

To set twist values on the publishing side, you can set the

\begin{verbatim}
mytwist = Twist()
mytwist.linear.x = x_vel
mytwist.linear.y = y_vel
mytwist.linear.z = z_vel
\end{verbatim}

\begin{verbatim}
mytwist.angular.x = x_ang_vel
mytwist.angular.y = y_ang_vel
mytwist.angular.z = z_ang_vel
pub.publish(mytwist)
\end{verbatim}

For the subscriber, you can access the data via:

\begin{verbatim}
def callback(msg):
    rospy.loginfo("Received a /cmd_vel message!")
    rospy.loginfo("Linear Components: [%f, %f, %f]"%(msg.linear.x, msg.linear.y, msg.linear.z))
    rospy.loginfo("Angular Components: [%f, %f, %f]"%(msg.angular.x, msg.angular.y, msg.angular.z))
\end{verbatim}
