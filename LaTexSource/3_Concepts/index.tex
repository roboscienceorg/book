\hypertarget{Chap:Terms}{%
\section{Fundamental Concepts}\label{Chap:Terms}}

Getting the language down is the first step. Robotics is like any other
engineering field with lots of jargon and specialized terms. The terms
do convey important concepts which we will introduce here.

There are three examples which we will examine: the serial two link arm,
the parallel two link arm and the differential drive mobile robot. The
serial two link manipulator is a simple robot arm that has two straight
links each driven by an actuator (like a servo). They serve as basic
examples of common robotic systems and are used to introduce some basic
concepts. The parallel two link arm is a two dimensional version of a
common 3D Printer known as the Delta configuration.

We will do one detour and present a very popular formalism to describe
manipulators. It will give a general way to mathematically describe
multilink robotic arms. The formalism will be applied to the previous
two link example to bring the discussion full circle. Last is the
differential drive mobile robot. This design has two drive wheels and
then a drag castor wheel. The two drive wheels can operate independently
like a skid steer ``Bobcat''.

1\_Terms 2\_ReferenceFrames 3\_PlanarManipulators 4\_Denavit-Hartenberg
5\_Mobile 6\_Terms\_Problems
