\hypertarget{workspaces-and-configuration-space}{%
\section{Workspaces and Configuration
Space}\label{workspaces-and-configuration-space}}

There are several frames of reference which are important to the robot.
The robot operates in the physical world and it can be tracked by an
external frame of reference. This is known as a \emph{world} or
\emph{global} \texttt{reference\ frame}. This is normally the observer's
frame of reference. Frames of reference on or within the robot are known
as \emph{local reference frames}. Each joint or actuator can have a
frame of reference known as the \emph{joint reference}. These are useful
in understanding the transformations induced by each joint which leads
to a kinematic model of the manipulator. For manipulation, the position
and orientation of the tool needs tracking and will require a frame of
reference known as the \emph{tool reference}. Relating the various
frames of reference requires some knowledge of coordinate systems and
transforms which is found in standard courses in linear algebra.

The operating environment for a robot is known as the \emph{workspace}.
It is defined as the volume or region for which the robot can operate.
For robots with manipulators, traditionally the \texttt{workspace} is
all points that the tool end can reach. For mobile robots it is the
region that the robot can and may move into. Any obstacle or constraint
will be called a workspace obstacle or workspace constraint. Motion of a
robot or articulator through the workspace is referred to by the term
\emph{workspace path}. Building a robot from rotary and linear joints
means that our effector end (tool end) can be moved in an angular or
linear fashion. This means that there is a natural coordinate system for
the workspace. The three that are commonly used are Cartesian,
Cylindrical or Spherical.

The range of all possible parameter values that the robot can modify is
known as the \emph{configuration space}. It is the span of the machine
when the actuators are run through their different positions. The
dimension of the \texttt{configuration\ space} is the degrees of
freedom. The difference between workspace and configuration space might
be confusing at first. Workspace is the physical one, two or three
dimensions, where is robot operates, whereas configuration space is made
up from the different configurations that the links and servos can
define. In the case of manipulators, it is common to represent
configuration space by the joint variables implicitly knowing the
relation between the joint variable and the configuration. We will use
\(q\) for a configuration and \({\cal Q}\) for configuration space to
distinguish it from workspace variables.

We can denote the configuration space occupied by the robot by \(R(q)\).
A configuration space obstacle is \({\cal Q}{\cal O}_i\),

\[{\cal Q}{\cal O}_i = \left\{ q\in {\cal Q} ~|~ R(q) \bigcap {\cal W}{\cal O}_i \neq \emptyset\right\}.\]

Free configuration space is then

\[{\cal Q}_\text{free} = {\cal Q}\setminus \left( \bigcup_i {\cal Q}{\cal O}_i\right).\]

Reaching a point in space requires a particular configuration of the
joints or motors. So there is a point in configuration space that
relates to a point in the workspace. Understanding the relation between
the articulators and the point in space turns out to be a very hard
problem. If you know the position of each joint, you can then generate a
mapping from the set of joint positions to a point in the workspace.
This is known as forward kinematics. Having a target point out in space
and asking what are the joint positions has to do with inverting the
kinematic equations and is known as \emph{inverse kinematics}. The
forward kinematics are expressed as a system of algebraic equations.
There is no general rule that these equations will be invertible. So,
\emph{IK}, inverse kinematics equations may not be available. Later in
this text we will explore numerical approaches .

Just having a relation between the physical workspace and the joint
(wheel, motor, etc) configuration is not the goal in robots. We normally
want to do something. We want to move. We might be welding along a seam
or driving a path. Our controls are working with the actuators and those
are translated over into the workspace through some rather complicated
mathematical expressions. The interesting challenge is to take a desired
path in the workspace, say the welder path, and figure out the motion of
the joints (motors) that produce this path. Then optimize based on
workspace obstacles, machine constraints and other considerations. This
is the subject of \emph{planning} which will be touched on later as well
as in courses on planning.

\hypertarget{forward-position-kinematics}{%
\subsection{Forward Position
Kinematics}\label{forward-position-kinematics}}

The \texttt{forward\ position\ kinematics} (FPK) solves the following
problem: ``Given the joint positions, what is the corresponding end
effector's pose?'' If we let \(x = (x_1, x_2, x_3)\) be the position as
a function of time and \(p = (p_1, p_2, \dots , p_n)\) the equations
that transform \(p\) into \(x\) are the forward kinematic equations

\[x = F(p).\]\[A three link planar manipulator.\]\[The mapping from configuration space to
workspace.\]

\hypertarget{forward-position-kinematics-for-serial-chains}{%
\subsection{Forward Position Kinematics for Serial
Chains}\label{forward-position-kinematics-for-serial-chains}}

The solution is always unique: one given joint position vector always
corresponds to only one single end effector pose. The FK problem is not
difficult to solve, even for a completely arbitrary kinematic structure.
We may simply use straightforward geometry, use transformation matrices
or the tools developed in standard engineering courses such as statics
and dynamics.

\hypertarget{forward-position-kinematics-for-parallel-chains-stewart-gough-manipulators}{%
\subsection{Forward Position Kinematics For Parallel Chains
(Stewart-Gough
Manipulators)}\label{forward-position-kinematics-for-parallel-chains-stewart-gough-manipulators}}

The solution is not unique: one set of joint coordinates has more
different end effector poses. In case of a Stewart platform there are 40
poses possible which can be real for some design examples. Computation
is intensive but solved in closed form with the help of algebraic
geometry.

\hypertarget{inverse-position-kinematics}{%
\subsection{Inverse Position
Kinematics}\label{inverse-position-kinematics}}

The \texttt{inverse\ position\ kinematics} (IPK) solves the following
problem: ``Given the actual end effector pose, what are the
corresponding joint positions?'' In contrast to the forward problem, the
solution of the inverse problem is not always unique: the same end
effector pose can be reached in several configurations, corresponding to
distinct joint position vectors. A 6R manipulator (a serial chain with
six revolute joints) with a completely general geometric structure has
sixteen different inverse kinematics solutions, found as the solutions
of a sixteenth order polynomial.

One may have the exact expression for the forward kinematics.

\[\begin{pmatrix} \theta_1(t), ... , \theta_n(t)
           \end{pmatrix}\to p(t)\]

However it is MUCH harder to find the IPK, i.e. the angle functions
(\(\theta_k\)) as a function of end effector position.

\[p(t) \to \begin{pmatrix} \theta_1(t), ... , \theta_n(t)
           \end{pmatrix}\]

\hypertarget{forward-velocity-kinematics}{%
\subsection{Forward Velocity
Kinematics}\label{forward-velocity-kinematics}}

The \texttt{forward\ velocity\ kinematics} (FVK) solves the following
problem: ``Given the vectors of joint positions and joint velocities,
what is the resulting end effector twist?'' The solution is always
unique: one given set of joint positions and joint velocities always
corresponds to only one single end effector twist. Using \(x\) to the
the position vector as a function of time and \(p\) the joint parameters
as a function of time, let the forward position kinematics be given by
\(x = F(p)\). Then the forward velocity kinematics can be derived from
the forward position kinematics by differentiation (and chain rule). A
compact notation uses the Jacobian of the forward kinematics:

\[v = J_F(p) q, \quad  \mbox{ where } \quad v = \frac{dx}{dt}, ~ q = \frac{dp}{dt}.\]

\hypertarget{inverse-velocity-kinematics}{%
\subsection{Inverse Velocity
Kinematics}\label{inverse-velocity-kinematics}}

Assuming that the inverse position kinematics problem has been solved
for the current end effector pose, the
\texttt{inverse\ velocity\ kinematics} (IVK) then solves the following
problem: ``Given the end effector twist, what is the corresponding
vector of joint velocities?'' Under the assumption that the Jacobian is
invertible (square and full rank) we can find \(J^{-1}\) and express

\[q = J_F(p)^{-1} v = J_F\left( F^{-1}(x) \right) v\]

\hypertarget{forward-force-kinematics}{%
\subsection{Forward Force Kinematics}\label{forward-force-kinematics}}

The \texttt{forward\ force\ kinematics} (FFK) solves the following
problem: ``Given the vectors of joint force/torques, what is the
resulting static wrench that the end effector exerts on the
environment?'' (If the end effector is rigidly fixed to a rigid
environment.)

\hypertarget{inverse-force-kinematics}{%
\subsection{Inverse Force Kinematics}\label{inverse-force-kinematics}}

Assuming that the inverse position kinematics problem has been solved
for the current end effector pose, the
\texttt{inverse\ force\ kinematics} (IFK) then solves the following
problem: ``Given the wrench that acts on the end effector, what is the
corresponding vector of joint forces/torques?''

We will not treat forward or inverse force kinematics in this text.
These concepts are treated in courses in statics and mechanics.
