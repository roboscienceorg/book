\hypertarget{terminology}{%
\section{Terminology}\label{terminology}}

In the Introduction, several terms were introduced such as end effector
or actuator. We will round out the common robotics terminology in the
section.

\begin{itemize}
\tightlist
\item
  \texttt{Manipulator}: the movable part of the robot, often this is the
  robotic arm.
\item
  \texttt{Degrees\ of\ Freedom}: the number of independently adjustable
  or controllable elements in the robot. It is also the number of
  parameters that are needed to describe the physical state of the robot
  such as positions, angles and velocities.
\item
  \texttt{End\ Effector}: the end of the manipulator.
\item
  \texttt{Payload}: the amount the robot can manipulate or lift.
\item
  \texttt{Actuator}: the motor, servo or other device which translates
  commands into motion.
\item
  \texttt{Speed}: the linear or angular speed that a robot can achieve.
\item
  \texttt{Accuracy}: how closely a robot can achieve its desired
  position.
\item
  \texttt{Resolution}: the numerical precision of the device, usually
  with respect to the end-effector. This can also be measured in terms
  of repeatability. Related to accuracy vs precision in general
  measurements.
\item
  \texttt{Sensor}: any device that takes in environmental information
  and translates it to a signal for the computer such as cameras,
  switches, ultrasonic ranges, etc.
\item
  \emph{Controller}: can refer to the hardware or software system that
  provides low level control of a physical device (mostly meaning
  positioning control), but may also refer to the robot control overall.
\item
  \texttt{Processor}: the cpu that controls the system. There may be
  multiple cpus and controllers or just one unit overall.
\item
  \texttt{Software}: all of the code required to make the system
  operate.
\item
  \texttt{Open\ Loop\ control}: a form of robot control that does not
  use feedback and relies on timed loops for placement.
\item
  \texttt{Closed\ Loop\ control}: using sensor feedback to improve the
  control accuracy.
\end{itemize}

Motion is achieved by some device that converts some energy source into
motion. Most often these are electric motors (even non-electric systems
often have electrically controlled components like valves.) However, it
is useful to not focus on the type of equipment, but just the type
motion induced. For simplicity anything that induces motion will be
called actuators for most of the text. Actuators apply forces to the
various robotic components in the system which in turn generates motion.
The connections between actuators are known as \emph{links}. For this
work we will assume they are rigid and fixed in size. Connecting links
are joints. This allows the links to move with respect to each other.
There are two types of common joints, \emph{rotary} and \emph{linear}
joints. The name essentially indicates what it does. A
\texttt{rotary\ actuator} allows the relative angle between the links to
change. A \texttt{linear\ actuator} changes the length of a link.
Examples of rotary joints are \texttt{revolute}, \texttt{cylindrical},
\texttt{helical}, \texttt{universal} and \texttt{spherical}
joints~\texttt{fig:robotjoints}. Linear joints are also referred to as
prismatic joints.

\begin{quote}
Some common robot joints.
\end{quote}

All of the machines we will study have moving components. The complexity
of the system depends on the number of components and the
interconnections therein. For example, a robotic arm may have three or
four joints that can be moved or varied. A vehicle can have
independently rotated wheels. The number of independently moving
components is referred to as the \emph{degrees of freedom}; the number
of actuators that can induce unique configurations in the system. This
mathematical concept comes from the number of independent variables in
the system. It gives a measure of complexity. Higher degrees of freedom,
just as higher dimensions in an equation, indicate a system of higher
complexity. This concept of degrees of freedom is best understood from
examples.

Consider a computer-controlled router that can move the tool head freely
in the \(x\) and \(y\) directions. This device has \emph{two degrees of
freedom}. It is like a point in the plane which has two parameters to
describe it. Going one step further, consider a 3D printer. These
devices can move the extruder head back and forth in the plane like the
router, but can also move up and down (in \(z\)). With this we see three
degrees of motion or freedom. While it may seem from these two examples
that the degrees of freedom come from the physical dimensions, please
note that this is not the case. Consider the 3D printer again. If we
added a rotating extruder head, the degrees of freedom would equal to
four (or more, depending on setup), but the physical dimensions would
stay at three.

Consider a welder that can position its tool head at any point in a
three dimensional space. This implies three degrees of freedom. We
continue and assume that this welder must be able to position its tool
head orthogonal to the surface of any object it works on. This means the
tool must be able to rotate around in space - basically pan and tilt.
This is two degrees of freedom. Now if we attach the rotating tool head
to the welder, we have five degrees of freedom: 5DOF.

Each joint in a robotic arm typically generates a degree of freedom. To
access any point in space from any angle requires five degrees of
freedom (\(x,y,z,pan,tilt\)). So why would we need more? Additional
degrees of freedom add flexibility when there are obstacles or
constraints in the system. Consider the human arm. The shoulder rotates
with two degrees of freedom. The elbow is a single degree of freedom.
The wrist can rotate (the twisting in the forearm) as well as limited
two degree motion down in the wrist. Thus the wrist can claim three
degrees of freedom. Without the hand, the arm has six degrees of
freedom. So you can approach an object with your hand from many
different directions. You can drive in a screw from any position.

\hypertarget{serial-and-parallel-chain-manipulators}{%
\subsection{Serial and Parallel Chain
Manipulators}\label{serial-and-parallel-chain-manipulators}}

Manufacturing robots typically work in a predefined and restricted
space. They usually have very precise proprioception (the knowledge of
relative position and forces) within the space. It is common to name the
design class after the coordinate system which the robot naturally
operates in. For example, a cartesian design (similar to gantry systems)
is found with many mills and routers, heavy lift systems, 3D Printers
and so forth, see \texttt{gantrysample}-left. Actuation occurs in the
coordinate directions and is described by variable length linear
segments (links) or variable positioning along a segment. This greatly
simplifies the mathematical model of the machine and allows efficient
computation of machine configurations.

In two dimensions, one can rotate a linear actuator about a common
center producing a radial design which would use a polar coordinate
description. Adding a linear actuator on the \(z\) axis gives a
cylindrical coordinate description, \texttt{gantrysample}-right

\begin{quote}
Basic designs. (left) Cartesian design - Muhammad Furqan, grabcad.com
(right) Cylindrical design - Mark Dunn, grabcad.com
\end{quote}

A \texttt{serial\ chain\ manipulator} is a common design in industrial
robots. It is built as a sequence of links connect by actuated joints
(normally seen as a sequence starting from an attached base and
terminating at the end-effector. By relating the links to segments and
joints as nodes, we see that serial link manipulators can be seen as
graphs with no loops or cycles. The classical robot arm is an example of
a serial chain manipulator, \texttt{serialparallel}-(left). Robot arms
normally employ fixed length links and use rotary joints. This are often
called articulated robots or the arm is called an articulator. Very
general tools exist to construct mathematical descriptions of arm
configuration as a function of joint angles. A formalism developed by
Denavit and Hartenberg can be used to obtain the equations for position.

\begin{quote}
Robot arms(left) Articulated - Ivo Jardim, grabcad.com, (right) Delta
Design - Ivan Volpe, grabcad.com

Stewart Platform - Micheal Meng, grabcad.com
\end{quote}

Another popular approach is the \texttt{parallel\ chain\ manipulator},
which uses multiple serial chains to control the end-effector. A couple
of examples, a Delta Robot, seen in \texttt{serialparallel}-(right) and
\texttt{fig:stewart}. Combinations of articulators can built to mimic a
human hand as seen in \texttt{armsample-c}.

\begin{quote}
Articulated with hand gripper - Chris Christofferson, grabcad.com
\end{quote}

\hypertarget{basic-machine-elements}{%
\subsection{Basic Machine Elements}\label{basic-machine-elements}}

We have been designing and using machines for thousands of years. There
is a wealth of very interesting designs to do a myriad of things. We
will review a few common designs in terms of basic function.

Sources of force in a robot arise from elecromagnetic devices such as DC
motors or chemical processes such as internal combustion. The force
produced is normally not in correct form for use on the robot. It could
be in the wrong direction, speed, magnitude of force, etc and needs to
be changed. This is where gears, joints, rods and other essential
components enter the design.

In robotics we see many electrically powered systems. The main method
for converting electrical power into mechanical power is to use a
rotational motor design. For driving wheels, propellors, or other
devices that use rotation energy this works very well. Getting linear
motion requires some additional components. Although direct linear
motion is possible through a solenoid type design, it is very common to
find a electric motor and some type of gearing system to create the
linear motion.

Gears are rotating elements with cut teeth that mesh with each other.
They are capable of transmission of power and can change speed, torque
or direction. Gears can be categorized according to relative relation of
their rotation axes: parallel axes, intersecting axes,
nonparallel-nonintersecting, and other.

Parallel axes contain spur gears, helical gears, internal gears and gear
rack designs. Intersecting axes include straight and spiral bevel gears,
miter gears. Nonparallel-nonintersecting gears have worm gears designs
and screw gear designs, \texttt{fig:typesofgears}.

\begin{quote}
Sample of different gear designs.
\end{quote}

Beyond axes, the way the teeth fall on the gear are important. An
external design is where the teeth lie on the outside (outer surface) of
the gear and internal gears are ones where the teeth lie on the inside.
Internal gears are nice in that they don't reverse the shaft rotation
direction. The teeth can run parallel to the rotation axis such as seen
in spur gears or straight cut gears. These are the simpliest designs and
only work with parallel axes. Helical gears use teeth that are not
parallel to the rotation axis.

In spur designs, the teeth mesh with mostly static contact points and
the engagement is all at once. Spur gears can be noisy at high
rotational speeds. Helical or spiral designs the teeth engage more
gradually and also slide against each other. This produces less noise or
vibrations at the cost of higher energy loss and heat production.

Bevel gear designs (especially spiral) are used in differentials to
transmit driveshaft rotation 90 degrees to axle rotation. Rack and pinon
systems can be used to transmit the rotational motion of a steering
wheel to the linear motion of a rod used to change the wheel angle
(steering). Worm gear designs can provide significant torque as well as
having the property of self-locking which means they are stationary when
no power is applied to the worm gear (the worm wheel cannot drive the
mechanism in reverse).

Most gear designs change angle by 90 degrees (although bevel angles and
teeth designs can work at other angles). However the angle is fixed for
the specific gear. If thesee angles are not fixed, which happens when
suspension systems and steering are employed, then one needs joints that
can address variable angles. Universal and flexible joints are used to
allow for variable changes in rotation axes.

\begin{quote}
Joint examples. (Left) Universal Joint - Devin Dyke, grabcad.com,
(Right) Flexible joint - Chintan (CK) Patel, grabcad.com
\end{quote}

Changing rotational motion to linear motion is important and for us
arises often in manufacturing robots such as CNC machines and 3D
Printers. A simple system is just a threaded rod and nut design (which
is can be seen as a variation of the worm gear concept). By spinning the
rod and not allowing the nut to spin, the nut will move up or down the
rod. The relative motion is determined by the thread pitch. Because all
of the threads of the nut are engaged, there can be considerable
friction.

An improvement over a threaded rod is the ball screw,
\texttt{fig:ballscrew} . The idea is to add small balls (bearings)
between the threads (or teeth if thinking of this like a worm gear).
This reduces friction and backlash as well as increases accuracy. Using
recirculating balls (in a wormdrive design) is how early automobiles
provided smooth blacklash free lower force steering.

\begin{quote}
Ball Screw - Glenn McKechnie, June 2006, Wikipedia
\end{quote}

Other common methods to transfer rotational motion are sprocket and
chains. Using different sizes of sprockets attached via the chain
provides changes in angular speed and torque, called gearing based on
the direct analogy to gears. The chains can be attached to tracks and a
track or tank drive system is produced. A toothed belt is a variation of
this system and is found from timing belts and Gilmer belts to the head
positioning belts for 3D printers.
