\hypertarget{simple-planar-manipulators}{%
\section{Simple Planar Manipulators}\label{simple-planar-manipulators}}

\hypertarget{serial-two-link-manipulator}{%
\subsection{Serial Two Link
Manipulator}\label{serial-two-link-manipulator}}

The last few paragraphs have introduced lots of jargon. To understand
them, it helps to see them in action. The simple
\texttt{two\ link\ manipulator}, \texttt{intro-two-link} is a good place
to start. Imagine a robotic arm that has two straight links with a
rotary joint at the base and a rotary joint connecting the two links. In
practice, these rotary joints would be run by motors or servos and
probably have some limits, but for now we will assume full \(360^\circ\)
motion.

The workspace that the arm operates inside is a disk,
\texttt{two-link-disk}. This is a two dimensional workspace. The figure
indicates the workspace in gray. It may also be the case that there is
something in the workspace, a workspace obstacle indicated in red. This
unit has two joints which define a two dimensional configuration space,
\texttt{intro-config-axis}. The dimension of the configuration space is
the degrees of freedom, and so this has two degrees of freedom. Since
the joint is rotary and moving a full \(360^\circ\) degrees returns you
to the same angle, the two directions wrap back on themselves. This
actually makes the configuration space a two dimensional torus or
``donut''.

We will illustrate what is meant by kinematics and inverse kinematics
using the two link manipulator. Forward kinematics will identify the
location of the end effector as a function of the joint angles,
\texttt{twolinklabeled}-(a). This is easily done using a little
trigonometry. First we find the location of \((\xi, \eta)\) as a
function of \(\theta_1\) and the link length \(a_1\),
\texttt{twolinklabeled}-(b):

\[\xi =  a_1 \cos \theta_1, \quad \eta = a_1 \sin \theta_1\]

The next link can be included with

\[\Delta x =  a_2 \cos (\theta_1 + \theta_2), \quad \Delta y = a_2 \sin ( \theta_1 + \theta_2)\]

Note that \(x = \xi + \Delta x\) and \(y = \eta + \Delta y\). Combining
the expressions, the forward kinematics are:

\[\begin{aligned}
\begin{matrix}
x = a_2\cos (\theta_1+\theta_2) + a_1 \cos \theta_1 \\
y = a_2 \sin (\theta_1 +\theta_2) + a_1\sin \theta_1
\end{matrix}
\end{aligned}\]

As you move the servos in the system, you can change the angles
\(\theta_1\) and \(\theta_2\). The formula~\texttt{twolinkforward} gives
the location of the end effector \((x,y)\) as a function of
\((\theta_1, \theta_2)\). The values \(x\), \(y\) live in the workspace.
The values \(\theta_1\), \(\theta_2\) live in the configuration space.
This is a holonomic system. A common application is to move the end
effector along some path in the workspace. How does one find the
``path'' in configuration space? Meaning how do we find the values of
\(\theta_1\), \(\theta_2\) that give us the correct \(x\), \(y\) values?
This requires inverting the kinematics equations, hence the term inverse
kinematics. The mathematics required is some algebra and trigonometery
for solving \(\theta_1\), \(\theta_2\) in terms of \(x\), \(y\).

To find the inverse kinematics formulas we must appeal to some
trigonometry (law of cosines):

\[x^2 + y^2 = a_1^2 + a_2^2 - 2a_1a_2 \cos (\pi - \theta_2).\]

Using \(\cos(\pi - \alpha) = -\cos(\alpha)\), we solve for \(\cos\) in
Eqn \texttt{eqn:theta2step1}:

\[\cos(\theta_2) = \frac{x^2 + y^2 - a_1^2 - a_2^2}{2a_1a_2 }\equiv D\]

Using a trig formula:

\[\sin(\theta_2) = \pm \sqrt{1-D^2}\]

Dividing the sin and cos expressions to get tan and then inverting:

\[\theta_2 = \tan^{-1}\frac{\pm\sqrt{1-D^2}}{D} = \mbox{atan2}(\pm\sqrt{1-D^2},D)\]

The +/- gives the elbow up and elbow down solutions. A source of errors
arises with the arctan or inverse tangent of the ratio. The inverse
function is multivalued and calculators (as with most software) will
return a single value known as the principle value. However, you may
want one of the different values. The problem normally is that since
\(-y/-x = y/x\) the inverse tangent function will not know which
quadrant to select. So it may hand you a value that is off by
\(\pm \pi\). We suggest that you use atan2 in your calculations instead
of atan which will isolate quadrant and also avoid the divide by zero
problem. We will do the mathematics with \(\tan^{-1}\), but keep our
code using atan2.

Continuing with the derivation, from Figure~\texttt{twolinklabeled2}, we
have

\[\theta_1 = \phi - \gamma = \tan^{-1}\frac{y}{x} - \gamma .\]

If you look at the two dotted blue lines you can see that the line
opposite \(\gamma\) has length \(a_2\sin \theta_2\). The segment
adjacent to \(\gamma\) (blue solid and dotted lines) has length
\(a_1 + a_2\cos \theta_2\). Then

\[\tan \gamma =  \frac{\mbox{Opposite}}{\mbox{Adjacent}} = \frac{a_2\sin \theta_2}{a_1 + a_2\cos\theta_2}\]

which gives us \(\gamma\):

\[\gamma = \tan^{-1} \frac{a_2\sin \theta_2}{a_1 + a_2\cos\theta_2}.\]

Plug \(\gamma\) into Eqn~\texttt{eqn:theta1step1} and we obtain

\[\theta_1 = \tan^{-1}\frac{y}{x} - \tan^{-1} \frac{a_2\sin \theta_2}{a_1 + a_2\cos\theta_2}\]

Given the two link manipulator kinematic equations:

\[\begin{aligned}
\begin{matrix}
x = a_2\cos (\theta_1+\theta_2) + a_1 \cos (\theta_1)\\
y = a_2 \sin (\theta_1 +\theta_2) + a_1\sin (\theta_1)
\end{matrix}
\end{aligned}\]

The inverse kinematics (IK) are

\[D = \frac{x^2 + y^2 - a_1^2 - a_2^2}{2a_1a_2 }\]

\[\theta_1 = \tan^{-1}\frac{y}{x} - \tan^{-1} \frac{a_2\sin \theta_2}{a_1 + a_2\cos\theta_2}, \quad\quad
\theta_2 = \tan^{-1}\frac{\pm\sqrt{1-D^2}}{D}\]

Let \(a_1 = 15\), \(a_2 = 10\), \(x=10\), \(y=8\). Find \(\theta_1\) and
\(\theta_2\):

\begin{enumerate}
\tightlist
\item
  \(D = (10^2 + 8^2 - 15^2-10^2)/(2*15*10) = -0.53667\)
\item
  \(\theta_2 = \tan^{-1}(-\sqrt{1-(-0.53667)^2}/(-0.53667))\approx -2.137278\)
\item
  \(\theta_1 = \tan^{-1}(8/10)-\tan^{-1}[(10\sin(-2.137278))/(15+ 10\cos(-2.137278))] \approx 1.394087\)
\end{enumerate}

Check the answer:\\
\(x = 10*\cos(1.394087-2.137278) + 15*\cos(1.394087) = 10.000\)\\
\(y = 10*\sin(1.394087-2.137278) + 15*\sin(1.394087) = 8.000\)

Note that all angles in this text are in radians unless explicitly
stated as degrees. This is to be consistent with standard math sources
as well as the default for most programming languages. Be careful with
arctan. It can bite you. Here is an example ...

Assume that \((x,y) = (9,10)\) and \((a_1, a_2) = (15,15)\). We compute

\[\begin{aligned}
\begin{array}{l}
D = \displaystyle  \frac{x^2 + y^2 - a_1^2 - a_2^2}{2a_1a_2 }
= \displaystyle \frac{9^2 + 10^2 - 15^2 - 15^2}{2(15)(15) } = -0.5977777777777777 \\[4pt]
\theta_2 = \tan^{-1}\displaystyle\frac{-\sqrt{1-D^2}}{D}  = 0.9300701118289644 \\[4pt]
\theta_1 = \tan^{-1}\displaystyle\frac{y}{x} - \tan^{-1} \displaystyle\frac{a_2\sin \theta_2}{a_1 + a_2\cos\theta_2}
         = 0.3729461690939078
\end{array}
\end{aligned}\]

Now check our answers ...

\[\begin{aligned}
\begin{matrix}
x = a_2\cos (\theta_1+\theta_2) + a_1 \cos \theta_1 = 17.93773762042545 \\
y = a_2 \sin (\theta_1 +\theta_2) + a_1\sin \theta_1 = 19.9308195782505
\end{matrix}
\end{aligned}\]

Not close. What happened? The first problem was that in
\(\displaystyle\frac{-\sqrt{1-D^2}}{D}\) which is
\(\displaystyle\frac{-0.801...}{-0.597...}\) becomes
\(\displaystyle\frac{0.801...}{0.597...}\) and then atan returns a
quadrant I angle of 0.930070... . We needed
\(\theta_2 = 0.93007 + \pi = 4.0716\). Then you get
\(\theta_1 = 1.94374...\).

\[\begin{aligned}
\begin{matrix}
x = a_2\cos (\theta_1+\theta_2) + a_1 \cos \theta_1 = 9.0 \\
y = a_2 \sin (\theta_1 +\theta_2) + a_1\sin \theta_1 = 9.99999 \approx 10
\end{matrix}
\end{aligned}\]

A simple way to relate the end effector to the base for a serial chain
manipulator is to see each link as a transformation of the base
coordinate system. This is the approach suggested by Denavit and
Hartenberg which is addressed in the next section.

\hypertarget{dual-two-link-parallel-manipulator}{%
\subsection{Dual Two Link Parallel
Manipulator}\label{dual-two-link-parallel-manipulator}}

The Delta configuration is not just found in \emph{Pick and Place}
machines but has also become popular with the 3D printing community.
This style of printer is fast and accurate. Just to get started, we look
at a two dimensional analog shown in \texttt{Fig:paralleltwolink}. The
top (red) is fixed and is of length \(L_0\). The two links on either
side shown in dark blue are connected by servos (in green). These links
are of length \(L_1\). The angles are measured from the dotted line (as
0 degrees) to straight down (90 degrees), see
\texttt{Fig:paralleltwolink2}. At the other end of the dark blue links
is a free rotational joint (pivot). That connects the two light blue
links which are joined together at the bottom with a rotational joint.

Unlike the previous two link manipulator, it is not completely obvious
what the workspace looks like (although you might guess something
elliptical). The configuration space is the space of all possible
angles. This is limited by the red base in theory and by the servos in
practice. Since 360\(^\circ\) motion for the servos is not possible, the
configuration space is a simple square \([\theta_m , \theta_M]^2\) where
\(\theta_m\), \(\theta_M\) are the minimum and maximum servo angles
respectively.

Define the coordinate system as \(x\) is positive right and \(y\) is
positive down. The origin is placed in the center of the red base link.
The question is to figure out the position of the end effector at
\((x,y)\) as a function of \(\theta_1\) and \(\theta_2\) with fixed link
lengths \(L_0\), \(L_1\), \(L_2\), Figure~\texttt{Fig:paralleltwolink2}.
As with the serial chain manipulator, this is an exercise in
trigonometry.

The forward kinematics will provide \((x,y)\) as a function of
\((\theta_1, \theta_2)\). The derivation is left as an exercise and so
the point \((x,y)\) is given by

\[(x,y) = \left( \frac{a+c}{2} + \frac{v (b-d)}{u} , \frac{b+d}{2} + \frac{v (c-a)}{u} \right)\]

Where

\[(a,b) = (-L_1 \cos(\theta_1) - L_0/2 , L_1 \sin(\theta_1) )\]

\[(c,d) = (L_1 \cos(\theta_2) + L_0/2 , L_1 \sin(\theta_2) )\]

and \(u = \sqrt{(a-c)^2 + (b-d)^2}\), \(v  = \sqrt{L_2^2 - u^2/4}\).

If you guessed that the workspace was an ellipse like the author did,
that would be wrong. If you guessed some type of warped rectangle, then
you have great intuition. \texttt{Fig:paralleltwolinkWS} shows the
workspace for the configuration domain \([0, \pi/2]^2\). The figure
graphs \(y\) positive going upwards and for the manipulator \(y\)
positive goes down (so a vertical flip is required to match up). The
workspace can be created by running a program that traces out all the
possible arm angles and plots the resulting end effector position (not
all points, but a dense sample of points will do just fine). Sample code
to plot this workspace is given in \texttt{lst:computeconfigdomain}.

The inverse kinematics will give you \((\theta_1, \theta_2)\) as a
function of \((x,y)\). This is another exercise in trigonometry. For
\((x,y)\) given, we obtain

\[\theta_1  = \pi - \beta - \eta , \quad \quad \theta_2 = \pi - \alpha - \gamma\]

where

\[\| G \| = \sqrt{(x-L_0/2)^2 + y^2},  \quad\quad \| H\| = \sqrt{(x+L_0/2)^2 + y^2}\]

\[\alpha = \cos^{-1} \frac{G^2 + L_0^2 - H^2 }{2GL_0}, \quad \quad \beta = \cos^{-1} \frac{H^2 + L_0^2 - G^2 }{2HL_0}\]

\[\gamma = \cos^{-1} \frac{G^2 + L_1^2 - L_2^2 }{2GL_1},\quad \quad \eta =  \cos^{-1} \frac{H^2 + L_1^2 - L_2^2 }{2HL_1}\]

\texttt{lst:IKParallelTwoLink} illustrates using the inverse kinematic
formulas for a specific pair of \((x,y)\) values.

For the general serial manipulator, given joint angles and actuator
lengths can one compute the end effector position? Yes. Is there a
standard way to approach the problem? Yes. The next section provides
some structure to address and automate the process.
