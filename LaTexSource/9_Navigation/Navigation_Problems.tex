\hypertarget{problems}{%
\section{Problems}\label{problems}}

\begin{enumerate}
\tightlist
\item
  Using Python with Matplotlib, code the basic motion algorithm.

  \begin{enumerate}
  \def\labelenumii{\alph{enumii}.}
  \tightlist
  \item
    Demonstrate your approach with one obstacle.
  \item
    Demonstrate with several obstacles.
  \end{enumerate}
\item
  Using Python with Matplotlib and the basic motion algorithm, place a
  set of obstacles that cause the robot to cycle and not find the goal.
  In other words, build a robot trap.
\item
  Write a Python algorithm to perform boundary following on a grid
  domain.
\item
  Write a boundary following routine for the DD robot in Gazebo using
  the Lidar. Use a video screen capture program to record the results.
\item
  Assume that you have a finite number of convex solid obstacles (solid
  means you are not starting inside). Prove or provide a
  counter-example.

  \begin{enumerate}
  \def\labelenumii{\alph{enumii}.}
  \tightlist
  \item
    Will Bug 1 succeed in navigating from any start point to any goal
    point?
  \item
    Will Bug 2 succeed in navigating from any start point to any goal
    point?
  \item
    Will Tangent Bug succeed in navigating from any start point to any
    goal point?
  \end{enumerate}
\item
  Sketch equations \texttt{LidarRangeEq} and \texttt{ObsConstrEq}.
\item
  Assume that you have a grid map of the type found in the left image of
  \texttt{coarsemap} which was stored in an array. If the start point
  was an interior cell, implement the Bug 1 algorithm to find the
  sequence of cells which describe an escape path if one exists.
\item
  Is it possible to have a single non-convex obstacle trap Bug 1 or Bug
  2?
\item
  Assume that you have a grid domain and the obstacles are represented
  in the grid map. Write a Python program to implement:

  \begin{enumerate}
  \def\labelenumii{\alph{enumii}.}
  \tightlist
  \item
    Bug 1
  \item
    Bug 2
  \item
    Bug 3
  \item
    Tangent Bug
  \end{enumerate}
\item
  Write a Python program to implement the Wavefront algorithm.

  \begin{enumerate}
  \tightlist
  \item
    Demonstrate on a map with multiple obstacles.
  \item
    Compare to the \(A^*\) approach in the previous exercise.
  \end{enumerate}
\end{enumerate}
