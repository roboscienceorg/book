\hypertarget{basic-motion-planning}{%
\section{Basic Motion Planning}\label{basic-motion-planning}}

\hypertarget{simple-planning}{%
\subsection{Simple Planning}\label{simple-planning}}

When controlling the robot without feedback, open loop control, we
preplan the route and then code up a list of motion instructions. For
differential drive robots, the easiest routes to drive are combinations
of lines and circles, \texttt{fig:simplecurvedpath}. If you have a rough
idea of the route, place some points along the route and connect with
line (or circle) segments. Along those segments, the differential drive
has constant wheel speed. In practice this is difficult since one cannot
have instant jumps in wheel velocity. This makes accurate turns
challenging. If stopping and turning in place on the route is
acceptable, paths with just straight lines are the easiest to develop,
\texttt{fig:simplestraightpath}. Then is is just a matter of starting
with the correct orientation and driving for a given amount of time.

\begin{quote}
Path with arcs

Path \emph{without} arcs
\end{quote}

There is a clear problem with open loop control (preprogrammed on any
path without sensor feedback). Any variation in the physical system can
cause drift. This drift accumulates over time and at some point the
robot is not driving the intended course. The other problem is that the
path is tuned to a specific obstacle field. We must know the obstacles
and their locations prior to moving. A more advanced algorithm would be
able to take a goal point and using knowledge of the current robot
location, drive itself to the goal. The basic motion algorithm attempts
this next step.\footnote{This algorithm is slightly more general in that
  it does not need the goal location, but just the direction to the goal
  during the process.}

\hypertarget{basic-motion-algorithm}{%
\subsection{\texorpdfstring{\texttt{Basic\ Motion\ Algorithm}}{Basic Motion Algorithm}}\label{basic-motion-algorithm}}

Assuming we have a simple obstacle map, how should we proceed? Try the
following thought experiment. Pretend that you are in a dark room with
tall boxes. Also pretend that you can hear a phone ringing and you can
tell what direction it is. How would you navigate to the phone? Figuring
that I can feel my way, I would start walking towards the phone. I keep
going as long as there are no obstructions in my way. When I meet an
obstacle, without sight (or a map) I can't make any sophisticated
routing decisions. So, I decide to turn right a bit and head that way.
If that is blocked, then I turn right a bit again. I can continue
turning right until the path is clear. Now I should take a few steps in
this direction to pass the obstacle. Hopefully I am clear and I can turn
back to my original heading. I head in this direction until I run into
another obstacle and so I just repeat my simple obstacle avoidance
approach.

\begin{quote}
Set heading towards goal\\
\textbf{while} Not arrived at goal \textbf{do}\\
\hspace*{0.333em}\hspace*{0.333em}*\emph{while}* No obstacle in front
\textbf{do}\\
\hspace*{0.333em}\hspace*{0.333em}\hspace*{0.333em}\hspace*{0.333em}Move
forward\\
\hspace*{0.333em}\hspace*{0.333em}end while\\
\hspace*{0.333em}\hspace*{0.333em}count = 0\\
\hspace*{0.333em}\hspace*{0.333em}*\emph{while}* count \textless= N
\textbf{do}\\
\hspace*{0.333em}\hspace*{0.333em}\hspace*{0.333em}\hspace*{0.333em}*\emph{while}*
Obstacle in front \textbf{do}\\
\hspace*{0.333em}\hspace*{0.333em}\hspace*{0.333em}\hspace*{0.333em}\hspace*{0.333em}\hspace*{0.333em}Turn
right\\
\hspace*{0.333em}\hspace*{0.333em}\hspace*{0.333em}\hspace*{0.333em}*\emph{end
while}*\\
\hspace*{0.333em}\hspace*{0.333em}\hspace*{0.333em}\hspace*{0.333em}Move
forward\\
\hspace*{0.333em}\hspace*{0.333em}\hspace*{0.333em}\hspace*{0.333em}incr
count\\
\hspace*{0.333em}\hspace*{0.333em}*\emph{end while}*\\
\hspace*{0.333em}\hspace*{0.333em}Set heading towards goal\\
\textbf{end while}

The direct path to the goal.

Path using the Basic Motion algorithm.
\end{quote}

\texttt{turtlebasicmotion\_b} illustrates the idea. This algorithm is
not completely specified. The amount of right turn and the distance
traveled in the move forward steps is not prescribed above. Assuming
values can be determined, will this approach work? We expect success
when faced with convex obstacles but not necessarily for non-convex
obstacles, \texttt{simple1motionproblem}. Using
\texttt{simple1motionproblem} as a guide, we can construct a collection
of convex obstacles which still foil the algorithm; this is expressed in
\texttt{simple2motionproblem}. The robot bounces from obstacle to
obstacle like a pinball and is wrapped around. Leaving the last obstacle
the robot reaches the cutoff distance and then switches back to the
``motion to goal" state. However, this sets up a cycle. So, the answer
to the question ``does this work" is not for all cases.

\begin{quote}
Getting trapped in a non-convex solid object.

A collection of convex objects can mimic a non-convex obstacle.
\end{quote}

In the Chapter~on Motion Planning, we will fully explore the challenge
of motion planning in an environment with obstacles. It is easy to see
how the thought experiment above can fail and more robust approaches are
needed. Before we jump into motion planning, we want to understand what
view of the world we can get from sensors. This is necessary so we know
what kind of assumptions can be made when developing our algorithms.

\textbf{Footnotes}
