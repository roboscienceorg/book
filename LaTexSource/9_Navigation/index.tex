\hypertarget{Chap:Navigation}{%
\section{Navigation}\label{Chap:Navigation}}

It can be argued that the single most important aspect of a robot is its
ability to move. Roughly motion is a necessary condition but not a
sufficient condition. Motion itself is not complicated. Requiring only
servos and motors, motion is easily accomplished. The complexity arises
through the interaction of the environment. In this chapter we explore
how robots move in the plane and navigate around simple landscapes.
Although ground robots have to address three dimensional environments,
the restriction to the plane simplifies the mathematics and the
algorithms allowing us to focus on concepts and not the complexities of
the extra dimension.

Motion planning is an entire field of study, we will highlight some
aspects here. The solution to the planning problem routes from an
initial configuration, start location and pose, to a final
configuration, end location and pose or goal.

The basic path planning problem refers determining a path in
configuration space such that the robot does not collide with any
obstacles and the path is consistent with the vehicle constraints.

MotionPlanning ExplorationandNavigation Implementation Mazes Wavefront
Guidance Navigation\_Problems
