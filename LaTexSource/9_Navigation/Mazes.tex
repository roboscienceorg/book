\hypertarget{mazes}{%
\section{Mazes}\label{mazes}}

A very common demonstration of mobile robot problem solving is a maze
escape. Even though a maze escape routine is standard fair for a data
structures course, it is still impressive to see it implemented with a
robot in a maze.

\begin{quote}
A very simple maze.

A more complicated maze.

Solution path through a maze.
\end{quote}

\texttt{mazesol} shows a solution path through a maze. The random mouse
algorithm is one approach to finding a route. The algorithm has the
``mouse'' travel straight until a wall is encountered. Then the
``mouse'' randomly selects a new direction to follow. This approach is a
form of random search which eventually finds a route, although rather
slowly.

\hypertarget{wall-following}{%
\subsection{Wall Following}\label{wall-following}}

The best known method to traverse a maze is the wall following method.
The idea is to place your left or right hand on the wall as you traverse
the maze. If the maze is simply connected, the method is proven to
provide a path out of the maze. By looking at \texttt{mazesol}, the
solution path partitions the maze. A simply connected maze is
partitioned into two objects which are deformable to a disk. To see
this, focus on the right (or in the figure the lower) part of the
separated maze. Tracing the path, \texttt{mazesolwall}, we record our
motion through the maze. This path can be extracted,
\texttt{mazesolcircle2} to see that it is indeed a circle. The topology
as not changed.

\begin{quote}
Wall following (right hand) to solve the maze.

Connecting the outside to make a circle.

Wall path extracted from the maze.

Moving the nodes on the path to show the circle.
\end{quote}

Since the path is a circle, then the algorithm will transport the robot
between any two points on the circle. Not having a simply connected maze
or having interior starting/finishing points can break this method
-which does not mean it will necessarily fail.

\begin{quote}
A maze for which wall following can fail.
\end{quote}

The \texttt{Pledge\ algorithm} is designed to address the problem of
exiting a maze which has non-simply connected components. This algorithm
does not work in reverse, meaning that it can escape a maze, but not
enter one.

\begin{quote}
\textbf{Input} A point robot with a tactile sensor\\
\textbf{Output} A path to the \(q_{\text{goal}}\) or a conclusion no
such path exists.\\
Set arbitrary heading.\\
\textbf{while} No obstacle in front \textbf{do}\\
\hspace*{0.333em}\hspace*{0.333em}*\emph{repeat}*\\
\hspace*{0.333em}\hspace*{0.333em}\hspace*{0.333em}\hspace*{0.333em}Move
forward\\
\hspace*{0.333em}\hspace*{0.333em}*\emph{end while}*\\
Select right or left side and place that side against the obstacle.\\
\textbf{while} Note original heading and sum of turns not zero
\textbf{do}\\
\hspace*{0.333em}\hspace*{0.333em}*\emph{repeat}*\\
\hspace*{0.333em}\hspace*{0.333em}\hspace*{0.333em}\hspace*{0.333em}Move
along obstacle while keeping "hand" on obstacle\\
\hspace*{0.333em}\hspace*{0.333em}\hspace*{0.333em}\hspace*{0.333em}Sum
turn angles\\
\textbf{end while}

The Pledge Algorithm.
\end{quote}

The final escape algorithm presented here is Trémaux's Algorithm. This
is a form of a recursive backtracker. From Wikipedia:

\begin{quote}
Trémaux's algorithm, invented by Charles Pierre Trémaux, is an efficient
method to find the way out of a maze that requires drawing lines on the
floor to mark a path, and is guaranteed to work for all mazes that have
well-defined passages. A path is either unvisited, marked once or marked
twice. Every time a direction is chosen it is marked by drawing a line
on the floor (from junction to junction). In the beginning a random
direction is chosen (if there is more than one). On arriving at a
junction that has not been visited before (no other marks), pick a
random direction (and mark the path). When arriving at a marked junction
and if your current path is marked only once then turn around and walk
back (and mark the path a second time). If this is not the case, pick
the direction with the fewest marks (and mark it, as always). When you
finally reach the solution, paths marked exactly once will indicate a
direct way back to the start. If there is no exit, this method will take
you back to the start where all paths are marked twice. In this case
each path is walked down exactly twice, once in each direction. The
resulting walk is called a bidirectional double-tracing.
\end{quote}

In most maze solving applications, the maze is represented by a graph.
If you have seen some basic graph search algorithms you will recognize
this as a type of Depth First Search (DFS). For the robot however, there
is more than the DFS maze solving code. There is also the details of
navigating corridors and turns. Using only bump sensors this can be a
challenge, one we will address with ranging sensors later in this
chapter. However, without good sensors, using the algorithms like
Trémaux's algorithm might not work out. Without the ability to drop and
sense breadcrumbs, the recursive backtracker will fail. One way to
approach this problem is to create a map of the maze as you work your
way through it. Acting on the map means you are working on existing
trails and this is just another way of marking the domain.

The robot is running on a more complicated lanscape than the just
operating in the maze. Working on a solution to the maze in the Pledge
Algorithm or Trémaux's algorithm is simply working along the abstracted
paths. We are neglecting all the issues relevant to a robot such as
driving straight down the corridor, detecting walls, keeping distance
from walls, navigating turns, etc. All of this low level navigation is
ignored in the maze algorithms above and they focus on the higher level
aspect of maze escape. This makes sense in that separating the levels
helps to separate tasks leading to better code design.

To reduce the complexity we separate the maps for the robot, the
landscape map, which will have a precision set by the sensors and the
map or graph required by the maze, maze map. The maze map can use a grid
with larger cells. Large cells would mean lower precision but smaller
arrays. However, this is not a problem since the low level routines are
doing the positioning on the high resolution map leaving the high level
routines to navigate.

The maze map can be thought of as a low resolution version of the
landscape map. Each cell can still be an occupancy map, but with large
cells. In this case it is useful to take the cell as large as possible
so that corridors or walls are one cell wide. Using the centers of
unoccupied cells, these are nodes. Adjacent free cells can have their
center nodes connected. This builds a graph representation, see
\texttt{coarsemap}. So, now we have a high resolution grid map and the
corresponding graph representation of free space. This concept will be
used later in more advanced path planniing algorithms. For now we employ
a simple path planner.

\begin{quote}
The coarsening of the grid map for a maze and the construction of the
graph representation. Left side image is a maze on a finer grid. The
right side image is a coarser grid with graph drawn.
\end{quote}

One of the simpliest planners is the flood fill approach. Begin at the
endpoint and run a flood fill algorithm. If the flood fill paints the
starting point then a path has been discovered. You can run the flood
fill algorithm on the landscape map, the reduced maze map or the maze
graph. For illustration, we focus on the second one.

There is a fundamental difference between exploring the domain and a
route, and having a map available to discover a route. If the entire
domain is known and the question is simply to find the route, there are
routing tools available. The route can be found before exploration. We
will see later that flood fill approaches can help even in partially
explored (or mapped) domains.

The maze is a high regular and artificial structure. We don't have
anything like them in nature and few things in our day to day
surroundings really resemble a maze. So, why discuss them? The maze has
setup some fundamental approaches which we will employ next. First, we
see that it makes sense to approach an obstacle, like a wall, and then
follow the obstacle. This is the ``place a hand on the wall" idea. We
see that that approach is not sufficient from more complicated mazes and
we also need to know when and where to break free of the obstacle. We
have learned that seeing the domain in terms of a graph is useful in
that we can apply algorithms designed for graphs, such as a depth first
search. We see that certain solutions are comprehensive in how they
solve the problem and others are not. The maze is then the launching
point for planners which live in unstructured worlds.
