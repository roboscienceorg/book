\hypertarget{guidance}{%
\section{Guidance}\label{guidance}}

First a few definitions are in order. According to Wikipedia, GNC,
Guidance, Navigation and Control:

\begin{itemize}
\tightlist
\item
  Guidance refers to the determination of the desired path of travel
  (the "trajectory") from the vehicle's current location to a designated
  target, as well as desired changes in velocity, rotation and
  acceleration for following that path.
\item
  Navigation refers to the determination, at a given time, of the
  vehicle's location and velocity (the "state vector") as well as its
  attitude.
\item
  Control refers to the manipulation of the forces, by way of steering
  controls, thrusters, etc., needed to execute guidance commands whilst
  maintaining vehicle stability.
\end{itemize}

Planners can find paths through free space. They may be planning based
on certain requirements such as minimal path, minimal curvature or
maximum distance to obstacle. Some but not all will respect the
kinematic constraints. Computationally, it may be rather expensive to
have the planner compute the path at a high resolution. The resolution
may be sufficiently coarse that it provides online performance in terms
of the planning, but opens the door for excessive drift. One may
interpolate between distant points using a polynomial interpolant. Cubic
splines are a very common interpolant which allows for endpoint values
and slope to be selected. Cubics will also provide smooth paths which
minimize curvature.

It should be noted that we are not trying to find \(x(t)\) and \(y(t)\)
to plug into the inverse kinematics for some robot. We will use the
parametric equations to create a finer grid of points which is in turn
handed to the speed and heading controller. Between the fine grid
points, the controller is driving and we are not using the inverse
kinematics.

\hypertarget{cubic-spline-example}{%
\subsection{Cubic Spline Example}\label{cubic-spline-example}}

Assume you want the spline that connects the points (1,-1) with (3,4).
Also assume that the derivative at (1,-1) is given by \(<1,-3>\) and at
(3,4) is given by \(<0,2>\). We can take \(t_0=0\) and \(t_1 = 1\). This
gives \(z = t\), \(\dot{z} = 1\), \(a = 1 - 2 = -1\), \(b = 2\),
\(c = -8\), \(d = 3\). This gives us the two splines for the parametric
description of the curve:

\[x(t) = (1-t) + 3t + t(1-t)[-1(1-t) + 2t]  = -3 t^3+4 t^2+t+1\]

\[y(t) = -(1-t) + 4t + t(1-t)[-4(1-t)+3t] =  -11 t^3+19 t^2-3 t-1\]

\[\dot{x} = -9t^2+8t+1, \quad \ddot{x} =   -18t+8\]

\[\dot{y} =   -33t^2 +38t -3, \quad \ddot{y} =  -66t+38\]

See \texttt{cubicsplinefigure} for a plot.

\begin{verbatim}
t0, t1 = 0, 1
x0, y0 = 1, -1
x1, y1 = 3, 4
xd0 , yd0 = 1, -3
xd1 = 0
yd1 = 2
dt = (t1-t0)
dx = (x1-x0)
dy = (y1-y0)
a = xd0*dt- dx
b = -xd1*dt+dx
c = yd0*dt-dy
d = -yd1*dt+dy
t = np.linspace(t0,t1,100)
dotz = 1.0/dt
z = (dotz)*(t-t0)
x = (1-z)*x0 + z*x1+z*(1-z)*(a*(1-z)+b*z)
y = (1-z)*y0 + z*y1+z*(1-z)*(c*(1-z)+d*z)
ptx = np.array([x0,x1])
pty = np.array([y0,y1])

plt.figure()
plt.xlim(0,4)
plt.ylim(-2,5)
plt.plot(ptx,pty, 'ro',x,y,'g-')
plt.legend(['Data', 'Interpolant'],loc='best')
plt.title('Cubic Spline')
plt.show()







Graph of the spline for example  `cubicsplineexample`.
\end{verbatim}

\hypertarget{example}{%
\subsubsection{Example}\label{example}}

Assume that your planner has provided the following points (0,0),
(10,50), (30,20). Also assume that you start with zero derivative at
(0,0), would like to pass through (10,50) with slope \(m=1\) and have
slope \(m=-1\) at (30,20). You would like to create set of points on the
curve separated by a distance of roughly 1 and not 10 or 50. How can you
do this? The solution is to create two cubic splines which will match
slopes at the three points. First we convert the problem into a
parametric problem: \((t,x,y)\): (0,0,0), (10,10,50) and \((t,x,y)\):
(0,10,50), (20,30,20) This was an arbitrary choice for the time values.
Working on the first segment and the spline formulas, for \(t_0 = 0\),
\((x,y) = (0,0)\) and \((\dot{x}, \dot{y} ) = (1,0)\) and for
\(t_1 = 10\), \((x,y) = (10,50)\) and \((\dot{x}, \dot{y} ) = (1,1)\).
From the data we then obtain for the first segment
\(z= 0.1t, a = 0, b = 0, c = -50, d = 40\), and on the next segment
\(z= 0.05t, a = 0, b = 0, c = 50, d = -10\).

\begin{quote}
The two cubic splines from the three data points.

Sampling the two splines to get guidance data.
\end{quote}

The plot, \texttt{fig:cubicsplineexample1} is produced by the following
code with setting the plot command to lines, g-. The following code as
is produces \texttt{fig:cubicsplineexample2}.

\begin{verbatim}
import numpy as np
import pylab as plt

def spline(t0,t1, x0, x1, y0, y1, xd0 , yd0, xd1, yd1, N):
  dt = (t1-t0)
  dx = (x1-x0)
  dy = (y1-y0)
  a = xd0*dt- dx
  b = -xd1*dt+dx
  c = yd0*dt-dy
  d = -yd1*dt+dy
  t = np.linspace(t0,t1,N)
  dotz = 1.0/dt
  z = (dotz)*(t-t0)
  x = (1-z)*x0 + z*x1+z*(1-z)*(a*(1-z)+b*z)
  y = (1-z)*y0 + z*y1+z*(1-z)*(c*(1-z)+d*z)
  ptx = np.array([x0,x1])
  pty = np.array([y0,y1])
  return x, y, ptx, pty

N = 20
t0, t1 = 0, 10
x0, y0 = 0, 0
x1, y1 = 10, 50
xd0 , yd0 = 1, 0
xd1, yd1 = 1, 1
xc1, yc1, ptx1, pty1 = spline(t0,t1, x0, x1, y0, y1, xd0 , yd0, xd1, yd1, N)

t0, t1 = 0, 20
x0, y0 = 10,50
x1, y1 = 30, 20
xd0 , yd0 = 1, 1
xd1, yd1 = 1, -1
xc2, yc2, ptx2, pty2 = spline(t0,t1, x0, x1, y0, y1, xd0 , yd0, xd1, yd1, N)

plt.figure()
plt.xlim(-5,35)
plt.ylim(-5,65)
plt.plot(ptx1,pty1, 'ro')
plt.plot(ptx2,pty2, 'ro')
plt.plot(xc1,yc1,'g.')
plt.plot(xc2,yc2,'g.')
plt.legend(['Data', 'Interpolant'],loc='best')
plt.title('Cubic Spline')
plt.savefig("cubicexample2.pdf")
plt.show()
\end{verbatim}
