\hypertarget{problems}{%
\section{Problems}\label{problems}}

\begin{enumerate}
\item
  Let \(z_1 = 284\), \(z_2 = 257\), \(z_3 = 295\), be measurements from
  sensors which have normally distributed errors with standard
  deviations \(\sigma_1 = 10\), \(\sigma_2 = 20\), \(\sigma_3 = 15\),
  respectively. What is the best estimate for the measured state?
\item
  You are given three distance sensors which all measure the same
  distance. To determine the accuracy of each you run repeated
  measurements of an object which you setup so that the sensor will
  return 2 meters. Running 40 measurements you obtain the data listed
  below. Next you measure the distance of an object ahead of the robot.
  Sensor A gives 2.4577696, sensor B gives 1.8967743, and sensor C gives
  2.1352561. Combine these readings to make a more accurate estimate of
  the object distance. Hint: you need to figure out the distributions
  and the data according to the variances.

\begin{verbatim}
2.2411549   1.8286673    2.3295015
2.3366108   1.9243295    2.4867167
1.9687234   1.8972737    2.5489412
2.1240351   2.0961534    1.9834876
2.3984044   1.7985819    2.6153805
2.3523899   1.8377782    1.8132444
2.1074266   1.8358201    2.5563951
2.4711542   1.8875839    2.2031291
2.1998000   1.8116113    2.3117542
2.2710086   1.8701890    2.3495262
2.3530473   1.7646824    2.0109293
2.4391559   1.9499153    2.5030771
2.2066306   1.9243432    2.3561112
2.3000099   1.8309696    2.3097754
2.2235766   1.8453219    2.2940692
2.1396901   1.8390955    2.1904604
2.0929719   1.7978329    2.5693897
2.3154159   1.8217245    1.9332188
2.3716302   1.9558670    2.3002433
2.2611420   1.8654487    2.5508342
2.1415088   1.7836290    2.6884786
2.2088487   1.9245743    2.5037028
2.2714614   1.8918415    2.7112663
2.3345816   1.8275421    2.1656644
2.3052296   1.8494488    2.1940472
2.1600041   1.7632971    2.2703708
2.0630943   1.8396972    2.6488544
2.0997821   1.8412331    2.1828831
2.3037175   1.7761007    2.2959535
2.4536524   1.8542271    2.0446945
2.3909478   1.8649815    2.7852822
2.1195966   1.9533324    2.5700007
2.0205112   1.8857815    2.1113650
2.1708006   1.7115595    2.1215336
2.0800748   1.9403332    2.3126032
2.3332722   1.8530670    2.4687277
2.0826115   1.8279041    2.6104026
2.2652480   1.9058054    2.3165716
2.3734464   1.9632258    2.0907554
2.0563260   1.9367908    2.2130578
\end{verbatim}
\item
  Perform a numerical study on the three sensor problem. The claim is
  that if you fuse the three sensor measurements using the formula
  derived in the text, the fused value (the estimate) is a better
  estimate than that any of the measurement values even if one
  measurement comes from a sensor with a very low error (small standard
  deviation). Generate sample values from three distributions with the
  same mean (select mean = 5) but very different sigmas (sigma1 = 0.05,
  sigma2 = 0.25, sigma3 = 0.5). Run 1000 experiments and compute the
  percentage for which this is true.
\item
  Let \(x = (x_1,x_2)\), \(y=(y_1,y_2)\) and define
  \(d(x,y) = \| x - y\|_P\) where \(\| x \|_P^2 =\, x^TPx\) for

  \[\begin{aligned}
  P = \left( \begin{array}{cc} 3 & 0.1 \\ 0.1 & 1
  \end{array}\right).
  \end{aligned}\]

  Find the closest point on the line \(x_2 = 10 - 5x_1\) to the origin
  with respect to \(d(x,y)\).
\item
  Model determination

  \begin{enumerate}
  \item
    Assume that you run an experiment on a single step. Starting from
    \((x_0, y_0, z_0) = (1, 1, 0.5)\), you obtain x, y, w:

\begin{verbatim}
2.608832, 5.055857, 6.189379
2.925827, 5.256055, 6.377555
2.741887, 5.012025, 6.225253
2.808115, 5.277323, 6.412870
2.604396, 4.942732, 6.143021
2.715381, 5.048058, 6.151169
2.785934, 5.153957, 6.457948
2.731107, 5.157646, 6.312867
2.741480, 5.052214, 6.327102
2.738335, 5.172248, 6.372636
2.790870, 5.152972, 6.270782
2.690942, 4.867113, 6.448155
2.788157, 4.831810, 6.151857
3.005297, 5.476095, 6.538915
2.778656, 5.085782, 6.314246
2.759511, 5.271102, 6.469469
2.633871, 4.915128, 6.243359
2.845448, 5.256687, 6.464442
2.736627, 5.146030, 6.300301
2.767497, 5.250046, 6.464192
2.860662, 4.980395, 6.294793
2.878436, 5.082964, 6.374364
2.825564, 5.114201, 6.288422
2.818848, 4.974110, 6.158882
2.844205, 5.102877, 6.354154
\end{verbatim}

    This data is repeated experiments and NOT iteration data. The means
    that each row is generated by starting from the initial condition
    and taking on step of your machine. Determine the parameters, a, b,
    c, and covariance V for the kinematic model with zero mean Gaussian
    noise based on dynamics:

    \[x_k = x_{k-1} + (x_{k-1}^2 + y_{k-1}^2)\cos(w_{k-1}) +a,\]

    \[y_k = y_{k-1} + (x_{k-1}^2 + y_{k-1}^2)\sin(w_{k-1})+b ,\]

    \[w_k = w_{k-1} + (x_{k-1}^2 + y_{k-1}^2 + w_{k-1}^2)^{1/2}+c.\]

    Approach this by computing the mean of each column. Using the means
    you can estimate a,b,c. Then using the covariance estimation given
    in the notes, you can find the covariance matrix.
  \item
    Assume that you have zero mean Gaussian data. Find a standard
    deviation that produces data where you observe that 20\% of the time
    you have three correct digits (meaning three zeros). This is not
    unique. Can you also find a sigma that gives you the previous
    observation but also 80\% of the time you see two correct (or zero)
    digits. Can you write an observational model for this?
  \end{enumerate}
\end{enumerate}

\hypertarget{problems-1}{%
\section{Problems}\label{problems-1}}

\begin{enumerate}
\item
  Basic Kalman Filter. Let

  \[\begin{aligned}
  x = \begin{bmatrix}x_1 \\ x_2\end{bmatrix}, \quad F = \begin{bmatrix} 0 &0.1 \\-0.02 &0.2\end{bmatrix}, \quad G_k u_k= \begin{bmatrix} 0\\ 2*\sin(k/25)\end{bmatrix},
  \end{aligned}\]

  \[\begin{aligned}
  H = \begin{bmatrix} 1& 0 \end{bmatrix},
  \quad V = \begin{bmatrix} 0.05^2&0\\0.& 0.05^2\end{bmatrix}, \quad W = 0.25^2,
  \end{aligned}\]

  \[\begin{aligned}
  x(0) = \begin{bmatrix} 0.025\\0.1\end{bmatrix}, \quad P(0) = \begin{bmatrix}0 & 0\\ 0&0\end{bmatrix}.
  \end{aligned}\]

  Apply the Kalman Filter process to compute 100 iterations and plot
  them. Hint: run the simulation to create your observation data \(z\)
  and then run your Kalman Filter.
\item
  Assume that one has three different measurements for the location of
  some object. The three measurements with the covariances are

  \[\begin{aligned}
  (10.5, 18.2), \quad \left(\begin{array}{cc} 0.1 & 0.01 \\ 0.01 & 0.15
    \end{array}\right); \quad
  (10.75, 18.0), \quad \left(\begin{array}{cc} 0.05 & 0.005 \\ 0.005 & 0.05
      \end{array}\right);
  \end{aligned}\]

  \[\begin{aligned}
  (9.9, 19.1), \quad \left(\begin{array}{cc} 0.2 & 0.05 \\ 0.05 & 0.25
  \end{array}\right).
  \end{aligned}\]

  Fuse this data into one measurement and provide an estimate of the
  covariance.
\item
  Run a simulation on

  \[\begin{aligned}
  \begin{array}{l}\dot{x} = y \\\dot{y} = -\cos(x) + 0.5\sin(t)\end{array}
  \end{aligned}\]

  adding noise to the \(x\) and \(y\) components (with variance = 0.2 on
  each). Let \(\Delta t = 0.1\). Assume that you can observe the first
  variable, \(x\), with variance \(0.25\). Record the observations.
  Write a program to run the EKF on the observed data and compare the
  state estimate to the original values.
\item
  Differential Drive - EKF. The motion equations for a differential
  drive robot are given below. Assume that the wheels are 5cm in radius
  and the wheelbase is 12cm. Recall that the kinematics for this is (r =
  radius, L = wheelbase):

  \[\begin{aligned}
  \begin{array}{l}
   \dot{x} = \frac{r}{2} (\dot{\phi_1}+\dot{\phi_2})\cos(\theta) \\[5mm]
  \dot{y} = \frac{r}{2} (\dot{\phi_1}+\dot{\phi_2})\sin(\theta) \\[5mm]
  \dot{\theta} = \frac{r}{2L} (\dot{\phi_1}-\dot{\phi_2})
  \end{array}
  \end{aligned}\]

  Select \(\Delta t = 0.2\) (time increment) and convert to discrete
  equations. After conversion, assume the covariance of the state
  transition is \(V\). Also assume that you have a local GPS system that
  gives \((x,y)\) data subject to Gaussian noise with covariance \(W\).
  The units on the noise are given in cm. If you want to use meters then
  you will need to divide your noise by 100.

  \begin{enumerate}
  \def\labelenumii{\alph{enumii}.}
  \item
    Starting at \(t=0\), \(x=0\), \(y=0\), \(\theta=0\), predict
    location when wheel velocities are:

\begin{verbatim}
t=0 -> 5:  omega1 = omega2 = 3 (rads/time),
t=5 -> 6:  omega1 = - omega2 = 1,
t=6 -> 10: omega1 = omega2 = 3,
t=10 -> 11:  - omega1 = omega2 = 1,
t=11 -> 16: omega1 =  omega2 = 3,
\end{verbatim}

    assuming that you have Gaussian noise in the process that is
    described by:

    \[\begin{aligned}
    `V = \begin{bmatrix}.05 &  .02 & 0.01\\.02& .05& 0.01\\ 0.01& 0.01& .1\end{bmatrix}`
    \end{aligned}\]
  \item
    Write out the formulas for the Extended Kalman Filter.
  \item
    Apply an Extended Kalman filter to the motion simulation above to
    track the location of the vehicle. Observations can be simulated by
    using previous simulation data as actual data, i.e. use this as the
    observed data (\(z_k\)). Parameters:

    \[\begin{aligned}
    x_{0|0} = (0,0,0), \quad V = \begin{bmatrix}.05 &  .02 & 0.01\\.02& .05& 0.01\\ 0.01& 0.01& .1\end{bmatrix},
    \end{aligned}\]

    \[\begin{aligned}
    W= \begin{bmatrix} .08& .02 \\.02&  .07\end{bmatrix}, \quad P_{0|0} = \begin{bmatrix}2 &0& 0\\0 &1& 0\\0& 0& 0.5\end{bmatrix}.
    \end{aligned}\]
  \item
    Output the x-y locations on a 0.5 sec grid and compare in a plot.
  \item
    The covariance matrix P gives the uncertainly ellipse for the
    location of the robot. Plot 5 ellipses along the path. This ellipse
    has major and minor axes given by the eigenvectors of P and the axes
    lengths are given by the associated eigenvalues. Matplotlib can plot
    an ellipse,
    \href{https://matplotlib.org/api/_as_gen/matplotlib.patches.Ellipse.html\#matplotlib.patches.Ellipse}{click
    here.}
  \end{enumerate}
\item
  Assume that you have a differential drive robot located in a lab with
  special landmarks placed in various locations around the lab. Also
  assume that your robot has a forward looking stereo vision system
  which can determine the distance and angle off the forward direction
  (relative to the robot) of the landmark. You don't know the locations
  of the landmarks and the stereo system can only see the landmarks if
  the angle off of the front is less than 75\(^{\circ}\).

  \begin{enumerate}
  \def\labelenumii{\alph{enumii}.}
  \tightlist
  \item
    Write the equations for the apriori EKF step (\(f_k\)) for some
    process noise covariance \(V_k\).
  \item
    Assuming that the error of the angular measurement is 2 degrees in
    standard deviation - when you can observe the landmark, and the
    distance measurement error is 5 percent; what is the observation
    formula (\(h_k\)) and the error \(W_k\)?
  \item
    What are the linearizations of \(f\) and \(h\)?
  \item
    What are the aposteriori formulas? Don't forget about the
    conversions from the robot (sensor) coordinates to the global or map
    coordinates.
  \item
    Write out the EKF process to track the location of the robot and the
    discovered landmarks. You should assume that you start at (0,0,0).
  \end{enumerate}
\end{enumerate}
