\hypertarget{the-impact-of-robotics---human-cost}{%
\section{The Impact of Robotics - Human
Cost}\label{the-impact-of-robotics---human-cost}}

\hypertarget{the-three-laws}{%
\subsection{The Three Laws}\label{the-three-laws}}

The Three Laws, quoted from the "Handbook of Robotics, 56th Edition,
2058 A.D.", are:\footnote{Asimov, Isaac (1950). "Runaround". I, Robot
  (The Isaac Asimov Collection ed.). New York City: Doubleday. p. 40.
  ISBN 978-0-385-42304-5. "This is an exact transcription of the laws.
  They also appear in the front of the book, and in both places there is
  no "to" in the 2nd law."}

\begin{description}
\item[\textbf{First Law}]
A robot may not injure a human being or, through inaction, allow a human
being to come to harm.
\item[\textbf{Second Law}]
A robot must obey the orders given it by human beings except where such
orders would conflict with the First Law.
\item[\textbf{Third Law}]
A robot must protect its own existence as long as such protection does
not conflict with the First or Second Law.
\end{description}

More to come on this topic ...

\hypertarget{the-core-issue}{%
\subsection{The core issue}\label{the-core-issue}}

Although robots in some form have been around for decades, robotics is
in many ways still an emerging technology - especially robots with the
newer AI systems included. Even with the optimistic predictions made
about the positive economic impacts, robotics has the potential to
displace workers and cause unemployment. We will briefly touch on a few
thoughts given by economists who study new technology. This is a
politically charged conversation and there are plenty of opinions. And
as a disclaimer, since this is a robotics text, probably sees more
benefits than risks with robotics.

We start with the concern of many Americans. "I am worried that I will
be replaced by a robot and then be un-employed." This is a valid
concern. There are plenty of examples of industrial jobs being replaced
by automation. It is often met with unsupported optimism that this new
age of robotics will provide a multitude of benefits for which we cannot
foresee. All of this is happening against a backdrop discussion of the
dangers found in modern AI. Given that this tech can render a percentage
of the workforce unemployed, or that there is profound concern of the
dangers of super intelligent AIs built into mobile robits -why would we
allow robots? It is a complicated question for which there is not a one
size fits all answer.

We start by describing two basic forms of technology entry: Enabling
Technologies and Replacing Technologies. Enabling technology is one that
assists a worker to be faster, more accurate, lower cost work. When view
at the system level, it is possible this tech will replace a person, but
overall allows for lower cost production. Replacing technologies do just
that. The worker is replaced by the new technology.

In real applications, elements of both will appear. The idea behind
enabling technologies is that the production line becomes more
efficient, profits increase and value increases. In the case where
people lose jobs, the idea is that the economic impact is sufficiently
strong to bring everyone up. Meaning the increased revenues then
generate other opportunities and those that had lost there jobs will
find good jobs elsewhere due to the economic impact. This is to be
contrasted to replacing technologies which remove individuals and do not
have an impact strong enough to counter act the economic effects of job
loss. Those individuals may find lower wage work or may be unemployed
which leads to dimished economic growth.

History has examples of both types of market entry. Replacing
technologies are seen to lower overall productivity and leave economic
picture worse off. This is the current trend. It does have historical
precedent and we have witnessed long periods of decline before. This is
when new tech enters, jobs are lost but new opportunity has yet to be
found.

It is hard to say where robotics lies currently - athough you will hear
plenty of experts take a stand. Either way, we can make some choices as
we move forward. We can make choices on what directions to increase
automation so that we create more opportunity to offset job loss.

An example of this is the following. Assume that an employer can reduce
costs for a stage of a process by replacing a person with automation.
But no other aspect of the process changes. So number of unit produced
does not change although there might be a drop in price. Most likley
this is an example of a replacing technology. Stopping here would lead
to an overall community productivity loss. The opportunity would be to
look at bottlenecks in other parts of the production line. If automation
can be added, then now it may be possible to increase the production
rate. A drop in price and an ability to source more product hopefully
means the ability to enter an additional market and look at scaling up.

The opportunity now is that if the factory scales (new or larger
markets), it can add new jobs based on the scaling. Additional
marketing, sales, supply chain people can be hired. So even though the
new automation has replaced a job just like the replacing technology
did, it acts as an enabling technology for the company. Of course this
is just a thought experiment and for actual situations, the truth may be
a combination of the different enabling tech, replacing tech and
business specific details. The point is that it is important on a larger
scale to look for enabling technologies which can open new markets and
effectively continue to grow overall productivity.

So, can we answer whether or not robotics will be good or bad for
society. Yes, we can answer. We don't mean we can provide a yes or no
now. We mean that we as a society can guide the process to achieve
either answer. So, our conclusion is that the impact of robotics is up
to us. Whether or not this is true for AI is altogether another unknown.
