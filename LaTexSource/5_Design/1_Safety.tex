\hypertarget{working-with-humans}{%
\section{Working with humans}\label{working-with-humans}}

\begin{itemize}
\item
  \begin{description}
  \item[Robert Williams, an American Auto Worker.]
  Mr. Williams worked at a Ford Motor Company factory in Flat Rock,
  Michigan. He was working on January 25, 1979 with a parts-retrieval
  system that moved material from one part of the factory to another.
  When the robot began running slowly, Williams reportedly climbed into
  the storage rack to retrieve parts manually. He was struck in the head
  by the arm of a 1-ton production-line robot as he was gathering parts
  and killed instantly. He would go down in history as the first
  recorded human death by a robot. William's family successfully sued
  the manufacturers of the robot, Litton Industries, and was awarded
  \$10 million dollars. The court concluded that there were insufficient
  safety measures in place to prevent such an accident from happening.
  \end{description}
\item
  \begin{description}
  \item[Kenji Urada, a Japanese maintenance engineer.]
  Urada worked at the Akashi Kawasaki Heavy Industries plant. On July 4,
  1981, Urada was checking on a malfunctioning robot. He leaped over the
  protective fence and accidentally hit the on-switch, resulting in the
  robot pushing him into a grinding machine with its hydraulic arm and
  crushing him to death. Mr. Urada is the second individual to be listed
  as a death by a robot.
  \end{description}
\item
  \begin{description}
  \item[Wanda Holbrook, an American technician.]
  In July 2015, Wanda Holbrook, a maintenance technician performing
  routine duties on an assembly line at Ventra Ionia Main, an auto-parts
  maker in Ionia, Michigan, was ``trapped by robotic machinery'' and
  crushed to death.
  \end{description}
\end{itemize}

When OSHA was founded in 1971, there was an estimated 14,000 job related
fatalities every year. This is roughly 38 deaths per day. In 2017, the
number has dropped to around 12 per day. The vast majority of these
deaths are impact related. The deaths are from falling or impact (struck
by vehicles or other machinery). Next in line are workplace violence,
electrocutions and drowning. Careful design, planning and implementation
of the workspace can prevent injuries and fatalities as well as save
considerable finances.

Large robots have become common in industrial environments and are now
starting to penetrate other markets. Since 1971, OSHA has documented
over 300,000 work related US fatalities. The fatality number for
industrial robotics is much lower at roughly 30 deaths for a 30 year
period, see ~\texttt{tab:deathstats}. {[}However, this is not a fair
comparison since the overall number of workers around industrial robots
is much less that the general work population.{]} A good argument can be
made that a number of robotic systems place the robot in the higher risk
situation and they have most likely saved more lives than were lost. The
point here, though, is that shipments of robots are currently
exponentially increasing and as these machines move out of heavy
industrial settings, the potential for human injury is exponentially
increasing.

Industrial robots deployed in the automotive sector are powerful
machines. They can strike a human with great force causing fatal
injuries. Even smaller or lower powered systems can cause significant
injury. The power-up process can produce unpredictable positioning and
movement. This has prompted a series of guidelines for the setup and use
of industrial robots. These systems are placed in cages or in blocked
off areas. Fenced regions are setup so that opening the fence door shuts
down power. Isolation gates are common practice to keep people safe. A
standard power-down procedure is required for physical access to the
robot and robot workspace.

An examination of OSHA records and supported by studies in Sweden and
Japan show that accidents don't occur during the normal operation of the
robot. During normal operation, training and safety barriers protect the
people working around the machines. Accidents occur during programming,
program touch-up or refinement, maintenance, repair, testing, setup, or
adjustment. Problems that arise occur when something unexpected has
happened. No training procedures for the current problem may exist and
the work staff is forced outside their training and expertise. The
situation may be confusing with multiple distractions. Human error
occurs often which has resulted in terrible accidents.

The reports show multiple instances of individuals circumventing safety
systems in place. By jumping fences or crossing safety barriers,
individuals placed themselves at grave risk. Accidents occur during
maintenance when systems are activated while individuals are inside the
robot workspace. Poor decisions to save time, by troubleshooting the
system without proper shutdown caused numerous fatalities and injuries.
Businesses that place extreme pressure to keep on schedule or not stop
the line, setup the culture of skipping normative practices leading to
unsafe decisions.

\hypertarget{robotics-industries-association}{%
\subsection{Robotics Industries
Association}\label{robotics-industries-association}}

Founded in 1974, \texttt{Robotics\ Industries\ Association},
\texttt{RIA}, is the only trade group in North America organized
specifically to serve the robotics industry. Member companies include
leading robot manufacturers, users, system integrators, component
suppliers, research groups, and consulting firms.
\url{https://www.robotics.org/}

Safety standards by the RIA are used as the industry standard. The
national standard ISO 10218 is based on the RIA standard R15.06. This
standard covers hazard identification, risk assessment, actuation
control, speed control, stopping control, operation modes, axis limiting
and all other aspects of robot design and operation.

Once the decision is made to bring in robots, a full hazard
identification and analysis is required. The standard identifies the
following aspects:

\begin{enumerate}
\tightlist
\item
  the intended operations at the robot, including teaching, maintenance,
  setting and cleaning;
\item
  unexpected start-up;
\item
  access by personnel from all directions;
\item
  reasonably foreseeable misuse of the robot;
\item
  the effect of failure in the control system; and
\item
  where necessary, the hazards associated with the specific robot
  application.
\end{enumerate}

Industrial robots need to be physically separated from people and the
standard defines the needed infrastructure. These include covering
gears, links, toolheads and electrical systems with panels and when not
possible placing in physical barriers between human work areas and the
robotics hardware. Robots need to have emergency stops and accessible
power-off panels. The physical barriers should automatically stop the
robot when anyone enters the workspace. The system needs to have speed
controls and workspace limit controls. Good software and good interfaces
are needed.

There is no substitute for good training and good policy. Many of the
accidents could have been avoided if workers followed the access rules.
Some accidents arose due to insufficient barriers, markers or space. All
of these threats can be addressed by a careful hazard analysis.
