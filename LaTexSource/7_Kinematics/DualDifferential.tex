\hypertarget{dual-differential-drive}{%
\section{Dual Differential Drive}\label{dual-differential-drive}}

This section derives the kinematics for a robot with a single axle. This
will be used to extend the differential drive to the dual differential
drive. All results are with respect to the local robot coordinate
system, with \(y\) the forward direction, \(z\) up, and \(x\) defined
according to the right hand rule. The total length of the axle is given
by \(2L\), the robot angle by \(\theta\), and the angle of the axle with
respect to the robot given by \(\alpha\), with \(\alpha=0\) aligning the
axle with the \(x\) axis \texttt{fig:angles}. The points at the end of
the axle are denoted by \(A\) and \(B\), with \(A\) corresponding to the
point in the positive \(x\) direction when \(\alpha=0\).

\begin{quote}
The axis angles.
\end{quote}

Simple planar kinematics gives the following relationships between the
velocities at points \(A\) and \(B\) and the robot motion. Let \(x,y\)
denote the center of the axle.

\[v_{A_x} = \dot{x}-L\dot{\alpha}\sin\alpha, \quad
v_{A_y} = \dot{y}+L\dot{\alpha}\cos\alpha\]

Incorporating the non-holonomic constraint on wheel velocity directions
yields

\[V_A\sin\alpha = L\dot{\alpha}\sin\alpha-\dot{x}, \quad
V_A\cos\alpha = \dot{y}+L\dot{\alpha}\cos\alpha\]

where \(V_A\) is the magnitude of the axle tip velocity. Similarly, for
point \(B\)

\[V_B\sin\alpha = -L\dot{\alpha}\sin\alpha-\dot{x}, \quad
V_B\cos\alpha = \dot{y}-L\dot{\alpha}\cos\alpha\]

Combining the equations for points \(A\) and \(B\) results in

\[\dot{y} = \frac{V_A+V_B}{2}\cos\alpha, \quad
\dot{x} = -\frac{V_A+V_B}{2}\sin\alpha, \quad
\dot{\alpha} = \frac{V_A-V_B}{2L}\]

The major difference with this current derivation and our previous
version in the Terms Chapter~ is that the coordinate system is rotated
by \(90^\circ\) compared to what we use.

The analysis now can be easily extended to the case of two axles. Let
the pivots for each of the two axles be separated from the robot
centroid by distance \(d\) in the \(y\) direction. Let \(A\) and \(B\)
denote the velocities of wheel for the axle offset in the positive \(y\)
direction from the centroid and \(C\) and \(D\) denote the velocities of
wheel for the axle offset in the negative \(y\) direction from the
centroid. The angle of the front axle with respect to the robot is given
by \(\alpha\), whereas the angle of the rear axle with respect to the
robot is given by \(\beta\). Then

\[\begin{aligned}
\begin{array}{l} V_A\sin\alpha = L\dot{\alpha}\sin\alpha-\dot{x}+d\dot{\theta}, \quad
V_A\cos\alpha = \dot{y}+L\dot{\alpha}\cos\alpha \\[4mm]
V_B\sin\alpha = -L\dot{\alpha}\sin\alpha-\dot{x}+d\dot{\theta}, \quad
V_B\cos\alpha = \dot{y}-L\dot{\alpha}\cos\alpha \end{array}
\end{aligned}\]

for the front axle and

\[\begin{aligned}
\begin{array}{l}  V_C\sin\beta = L\dot{\beta}\sin\beta-\dot{x}-d\dot{\theta}, \quad
V_C\cos\beta = \dot{y}+L\dot{\beta}\cos\beta \\[4mm]
V_D\sin\beta = -L\dot{\beta}\sin\beta-\dot{x}-d\dot{\theta}, \quad
V_D\cos\beta = \dot{y}-L\dot{\beta}\cos\beta\end{array}
\end{aligned}\]

for the rear axle.

Combining equations for the dual differential drive case results in

\[\dot{y} = \frac{V_A+V_B}{2}\cos\alpha=\frac{V_C+V_D}{2}\cos\beta\]

Note that this equation places a constraint on the relationship between
front and rear axle velocities.

\[\begin{aligned}
\begin{array}{l}
\displaystyle \dot{\theta} = \frac{(V_A+V_B)\sin\alpha-(V_C+V_D)\sin\beta}{4d}\\[4mm]
\displaystyle \dot{x} = -\frac{(V_a+V_B)\sin\alpha+(V_C+V_D)\sin\beta}{4}\\[4mm]
\displaystyle \dot{\alpha} = \frac{V_A-V_B}{2L}, \quad
\dot{\beta} = \frac{V_C-V_D}{2L}\end{array}
\end{aligned}\]

Implementation of the forward kinematics is easily done and can be
simulated for sample wheel speeds without use of the brake.
\texttt{fig:DDDpath}, shows the resulting path for sample wheel inputs
which demonstrate the ability to steer the craft. The wheel speeds for
this figure are

\[\begin{aligned}
\begin{array}{l}
V_A, V_B =  5t - t^2 + 1.5 \mp \sin(t), \quad 0 \leq t \leq 5 \\[3mm]
V_C, V_D = (5t - t^2)\cos(\alpha)/\cos(\beta) \pm \sin(t) ,   \quad 0 \leq t \leq 5 .
\end{array}
\end{aligned}\]\[Path for the DDD system demonstrating the ability to steer and
control the vehicle with free axle pivots.\]

\hypertarget{four-axle-robot-or-the-four-wheel-steer-robot}{%
\section{Four Axle Robot or the Four Wheel Steer
Robot}\label{four-axle-robot-or-the-four-wheel-steer-robot}}

The case of a four axle robot is very similar to the dual differential
drive case. The angles of the four axles are \(\alpha\), \(\beta\),
\(\gamma\), and \(\delta\), with \(\alpha\) representing the angle of
the axle in the first quadrant, \(\beta\) the angle of the axle in the
second quadrant, \(\gamma\) the angle of the axle in the fourth
quadrant, and \(\delta\) the angle of the axle in the third quadrant.
Let the hinge point be located by vector \(\vec{r}\) with components of
magnitude \(r_x\) and \(r_y\) with respect to the centroid of the robot,
and have the wheel located at distance \(L\) from the hinge. Then the
velocities of the ends of the axles are given below. The constraints for
the front two axles are:

\[\begin{aligned}
\begin{array}{l}
V_{A_x} =\dot{x}-r_y\dot{\theta}-\dot{\alpha}L\sin\alpha = -V_A\sin\alpha, \\[4mm]
V_{A_y} = \dot{y}+r_x\dot{\theta}+\dot{\alpha}L\cos\alpha = V_A\cos\alpha , \\[4mm]
V_{B_x} =\dot{x}-r_y\dot{\theta}+\dot{\beta}L\sin\beta = -V_B\sin\beta, \\[4mm]
V_{B_y} = \dot{y}-r_x\dot{\theta}-\dot{\beta}L\cos\beta = V_B\cos\beta , \end{array}
\end{aligned}\]

and the constraints for the rear two axles are:

\[\begin{aligned}
\begin{array}{l}
V_{C_x} =\dot{x}+r_y\dot{\theta}+\dot{\gamma}L\sin\gamma = -V_C\sin\gamma, \\[4mm]
V_{C_y} = \dot{y}-r_x\dot{\theta}-\dot{\gamma}L\cos\gamma= V_C\cos\gamma , \\[4mm]
V_{D_x} =\dot{x}+r_y\dot{\theta}-\dot{\delta}L\sin\delta = -V_C\sin\delta, \\[4mm]
V_{D_y} = \dot{y}+r_x\dot{\theta}+\dot{\delta}L\cos\delta= V_C\cos\delta  . \end{array}
\end{aligned}\]

These equations reduce to the DDD case when the offset is removed, i.e.,
when pivots are located in the center of the robot. The consequence is
that the constraint these equations present is \(\alpha=\beta\) and
\(\gamma = \delta\). For any other angular relationships the wheels'
kinematic constraints would conflict and the robot would be locked in
place. In the general case, we must have a relation
\(\alpha=\beta + \epsilon_1\) and \(\gamma = \delta+ \epsilon_2\) where
\(\epsilon_1\), \(\epsilon_2\) are the corrections due to the offset.

However, clearly there are admissible motions, such as the case in which

\[\begin{aligned}
\begin{array}{l} V_{A_y} = V_{B_y} = V_{C_y} = V_{D_y} = \dot{y},\\[4mm]V_{A_x} = V_{B_x} = V_{C_x} = V_{D_x} = 0, \\[4mm]\dot{\theta} = \alpha = \beta = \gamma = \delta = \dot{x} = 0.\end{array}
\end{aligned}\]

In other words, a vehicle that already has forward motion could maintain
it with all brakes unlocked. Given the constraint that the angles must
remain equal, the kinematics of the FWS robot are identical to those of
the DDD robot as expected.

The system that emerges is one where the split axles are connected to
the center of the robot as shown in \texttt{fig:DDDFWS}. The locking
mechanism will lock the axles in line, but leave them free to pivot with
respect the frame. This produces a robot which has DDD motion normally.
When the pivot brakes are released, then the axles can separate and the
wheels move to a configuration that allows in place rotation.

\begin{quote}
Hybrid between the DDD and FWS designs. This places the pivots at the
center allowing different axle angles. This design also holds costs by
only using two brakes.
\end{quote}

So, based on the kinematics, we see that linear motion is possible for
the both vehicles when the pivot brakes are locked or free. The DDD
vehicle can also turn without locks on the pivots. The kinematic
constraint induced by the body connection between front and rear axles
places constraints on wheel motion (as expected). Violating these will
cause wheel slip and slide. You can think of DDD motion as simply two
differential drive robots moving in tandem.

The FWS system is more complicated and the dynamics do allow unlocked
pivots during a turn as long as not all are unlocked. So, dynamic turns
can be performed by acting on axles sequentially. One may employ motion
sequences such as

\begin{enumerate}
\tightlist
\item
  Unlock rear axle pivots
\item
  Change rear wheel velocities
\item
  Lock rear axle pivots
\item
  Unlock front axle pivots
\item
  Change front wheel velocities
\item
  Lock front axle pivots
\end{enumerate}

to turn the robot without performing a complete stop. This configuration
works very much like an Ackerman drive other than the ability to stop
and rotate in place. A simulation is shown of the DDD-FWS hybrid in
\texttt{fig:FWSpath}.

\begin{quote}
Path for the DDD-FWS hybrid system demonstrating the ability to steer
and control the vehicle with free axle pivots. The system stops halfway
and resets pose.
\end{quote}
