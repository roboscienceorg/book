\hypertarget{mathematical-notation}{%
\section{Mathematical Notation}\label{mathematical-notation}}

This text will use italicized letters for the variables in formulas:
\(x\). We will not distinguish scalar (single) variables from vector
(array) variables. Scalar and vector quantities will normally be lower
case (unless we need to be consistent with a well known formula). When
possible, elements of a vector will be indexed using a subscript
\(x_k\). Matrices will be indicated using italicized uppercase, \(M\).
Discrete time steps will be indicated by superscripts, \(x^k\).

\texttt{Notation} related to robotics:

\begin{itemize}
\tightlist
\item
  Real line: \({\mathbb R}\)
\item
  Plane: \({\mathbb R}^2\)
\item
  Space: \({\mathbb R}^3\)
\item
  Workspace: \({\cal W}\)
\item
  Workspace Obstacles: \({\cal W}{\cal O}_i\)
\item
  Free space: \({\cal W}\setminus \bigcup_i {\cal W}{\cal O}_i\)
\item
  Point in space: \(x \in {\mathbb R}^n\)
  \(x = (x_1, x_2, x_3, \dots, x_n)\)
\item
  Vector: \(\vec{v} \in {\mathbb R}^n\),
  \(v = [v_1, v_2, \dots , v_n]^T\).
\item
  Bounded workspace: \({\cal W} \subset B_r(x)\)
  \(\equiv \{ y \in {\mathbb R}^n | d(x,y) < r\}\) for some
  \(0 < r < \infty\).
\end{itemize}

We will use several forms of notation for derivatives. Let \(f(x)\) be a
differentiable function, both Newton's notation, \(f'(x)\), and
Leibniz's notation, \(df/dx\), will be employed. Specifically for
derivatives of functions of time, we will also use the superscript dot
notation, \(\dot{g}\). The dot product and cross product of two vectors
will be denoted by \(x \cdot y\) and \(x \times y\) respectively. Matrix
vector and matrix matrix products will be denoted by \(Ax\) and \(AB\).
The normal matrix multiplication is implied. Rarely will we need to do
element-wise matrix and vector operations. Those will be written out
since they don't occur often enough to warrant custom notation.

There is a temptation for some authors to present material in the most
abstract setting possible. It can be argued that this provides the most
general case (widest application). In addition, one should remove the
specifics of any application so the abstract formulation represents the
core concept at its purest form. In many cases this is true. At times it
is also true that the author is trying to impress his or her audience
with their mathematical talent using an assault of symbols. At times I
am sure you will feel this way, but we have made significant effort to
keep the mathematical level in the first two years of an engineering
curriculum (Calculus, Differential Equations, Linear Algebra,
Probability).

\hypertarget{parametric-form}{%
\subsection{\texorpdfstring{\texttt{Parametric}
Form}{Parametric Form}}\label{parametric-form}}

Say you want to traverse a path \(C\), shown in Figure~
\texttt{Fig:intro-path}

\begin{quote}
A path for an explicitly defined function.

A path for a parametric function.
\end{quote}

The path \(C\) often will come from some function description of the
curve \(y = f(x)\). This type of description will work for many paths,
but fails for a great number of interesting paths like circles:
\(x^2 + y^2 = 1\). We want the robot to wander around in the plane
crossing our own path which certainly is not the graph of a function.
So, we must move to a parametric description of the path (actually a
piecewise parametric description).

The first step is to write in \texttt{parametric\ form}: \(x(t)\),
\(y(t)\). Example: convert \(y=x^2\) to parametric

\[\mbox{Let } x = t  \quad \to \quad y = x^2 = t^2\]

Note that there are an infinite number of choices. Let

\begin{quote}
\[\begin{aligned}
\begin{array}{l}
x = 2t  \quad \to \quad y = x^2 = 4t^2 \\
x = e^t  \quad  \to \quad y = x^2 = e^{2t} \\
x = \tan(t) \quad \to \quad y = x^2 = \tan^2(t)
\end{array}
\end{aligned}\]
\end{quote}

and so forth.

All the parametric forms provide the same curve, same shape, same
geometry. They vary in the speed. Think of the function form telling you
the shape, like the shape of a road, but not the velocity. The
parametric form gives you both path shape and velocity. We will assume
that you can find parametric functions \(x = \phi(t)\) and
\(y = \psi(t)\) such that the graph is \(y=f(x)\) which generates the
path \(C\) of interest.

\hypertarget{example-functions}{%
\subsubsection{Example Functions}\label{example-functions}}

Some examples of parametric forms may help in getting good at writing
these down.

\begin{description}
\item[Line]
\(x(t) = t\), \(y(t) = mt + b\), where \(m\) is the slope and \(b\) is
the intercept.
\item[Circle]
\(x(t) = R \cos(t) + h\), \(y(t) = R \sin(t) + k\), where the radius is
\(R\) and the center is \((h,k)\).
\item[Ellipse]
\(x(t) = A \cos(t) + h\), \(y(t) = B \sin(t) + k\), where \(A\) and
\(B\) describe the major and minor axes and the center is \((h,k)\).
\item[Lissajous]
\(x(t) = A\sin(at)\), \(y(t) = B \sin(bt)\) (\texttt{Fig:intro-path2}
\(A=1\), \(B=1\), \(a=3\), \(b=4\)). Infinity: \(A=1\), \(B=0.25\),
\(a=1\), \(b=2\)
\item[Root]
\(x(t) =  t^2\), \(y(t) = t\).
\item[Heart]
\(x(t) = 16\sin^3(t)\),
\(y(t) = 13\cos(t) - 5\cos(2t) -2\cos(3t) - \cos(4t)\)
\end{description}

\hypertarget{calculus}{%
\subsection{Calculus}\label{calculus}}

Robotics requires good calculus skills (along with linear algebra,
differential equations and probablity).

The derivative of a function \(x(t)\) is

\[\displaystyle \frac{dx}{dt}\]

This text will follow the standard notation for derivatives:

\[f'(x) \equiv \displaystyle \frac{df}{dx}\]

\[\dot{x} \equiv \displaystyle \frac{dx}{dt}\]

The chain rule will be used very often

\[\displaystyle \frac{d}{dt} f(g(t))  = f'(g(t)) \frac{dg}{dt}\]

For example

\[\displaystyle \frac{d}{dt} \sin(\theta(t))  = \cos(\theta(t)) \dot{\theta}\]

Since the digital systems use discrete data, we often must use a
discrete approximation of the derivative. This is also known as a
\texttt{finite\ difference}. The following approximations will be used:

\[\frac{dx}{dt} \approx \frac{x(t+\Delta t) - x(t)}{\Delta t}\]

If our data lies on a grid: \(t_k\) where \(t_k - t_{k-1} = \Delta t\),
or \(t_k \equiv k\Delta t\) and let \(x(t_k) \to x_k\) we have:

\[\frac{dx}{dt} \approx \frac{x(t_k+\Delta t) - x(t_k)}{\Delta t} = \frac{x_{k+1} - x_k}{\Delta t}\]

This is known as a \texttt{forward\ difference}. A
\texttt{backwards\ difference} is

\[\frac{dx}{dt} \approx \frac{x_{k} - x_{k-1}}{\Delta t}\]

For completeness, a \texttt{central\ difference} is given by

\[\frac{dx}{dt} \approx \frac{x_{k+1} - x_{k-1}}{2\Delta t}\]

We will also need a second derivative. In this case we only present the
central difference.

\[\frac{d^2x}{dt^2} \approx \frac{x_{k+1} -2x_k + x_{k-1}}{\Delta t^2}\]

This is often used to approximate differential equations producing a
difference equation. This is \texttt{Euler\textquotesingle{}s\ Method}.
We approximate

\[\frac{dx}{dt} = f(x)\]

by

\[\frac{x_{k+1} - x_k}{\Delta t} \approx \frac{dx}{dt} = f(x)\]

or

\[x_{k+1} - x_k = \Delta t f(x_k) \quad\Rightarrow\quad x_{k+1} = x_k + \Delta t f(x_k)\]

\hypertarget{vectors-matrices-and-linear-systems}{%
\subsection{Vectors, Matrices and Linear
Systems}\label{vectors-matrices-and-linear-systems}}

A \texttt{vector} is a list of numbers. It can be used to represent
physical quantities like force and direction. It can be expressed as

\[\vec{x} = \left< x_1, x_2, x_3, \dots , x_n \right>.\]

The notation for a point in n-dimensional space and a n-dimensional
vector are similar: \(\vec{x}\in \mathbb{R}^n\):
\(\vec{x} = (x_1, x_2, ... x_n)\), and also written as

\[\begin{aligned}
\vec{x} = \left(\begin{array}{c} x_1 \\ x_2 \\ \vdots
\\ x_n \end{array}\right).
\end{aligned}\]

If the context is understood, the small arrow above the variable is left
off, so \(\vec{x}\) becomes \(x\). The basic datatype used in scientific
computing is the array. Arrays are used to store points, vectors,
matrices and other mathematical constructs. The basic operations defined
on vectors are listed below. Let \(c\in \mathbb{R}\) and \(x,y \in
\mathbb{R}^n\), then

\begin{itemize}
\tightlist
\item
  Sum: \(x+y = \{ x_1 + y_1, x_2 + y_2, \dots, x_n + y_n\}\)
\item
  Scalar multiplication: \(cx = \{ cx_1, cx_2, \dots , cx_n\}\)
\item
  Inner product (related to angle): \(x \cdot y = \sum_{i=1}^n x_iy_i\)
\item
  Norm (length): \(\| x \| = \sqrt{\sum_{i=1}^n x_i^2}\)
\item
  Norm as multiplication: \(\| x \|^2 = x^T x\)
\end{itemize}

We will make use of matrix algebra and will follow the normal
conventions. Let \(A, B \in \mathbb{R}^{n\times n}\),

\[\begin{aligned}
A =
\left( \begin{array}{ccc}a_{11}&\dots&a_{1n}\\ \dots & \dots & \dots
\\ a_{n1} & \dots & a_{nn}\end{array}\right), \quad B =
\left( \begin{array}{ccc}b_{11}&\dots&b_{1n}\\ \dots & \dots & \dots
\\ b_{n1} & \dots & b_{nn}\end{array}\right).
\end{aligned}\]

Matrix addition and multiplication are defined in the standard manner as

\begin{itemize}
\tightlist
\item
  \(A+B=\left( \begin{array}{ccc}a_{11}+b_{11}&\dots&a_{1n}+b_{1n}\\ \dots & \dots & \dots \\ a_{n1}+b_{n1} & \dots & a_{nn}+b_{nn}\end{array}\right)\)
\item
  \(AB =\left( \begin{array}{ccc}c_{11}&\dots&c_{1n}\\ \dots & \dots & \dots\\ c_{n1} & \dots & c_{nn}\end{array}\right)\),
\end{itemize}

where the entries are \(c_{ij} = \sum_k a_{ik}b_{kj}\)

Matrix vector multiplication occurs often and is given by

\begin{itemize}
\tightlist
\item
  \(Ax = \left( \begin{array}{ccc}a_{11}&\dots&a_{1n}\\ \dots & \dots & \dots\\ a_{n1} & \dots & a_{nn}\end{array}\right)\left(\begin{array}{c} x_1 \\ x_2 \\ \vdots\\ x_n \end{array}\right) =   \left(\begin{array}{c} \sum_k a_{1k}x_k \\ \sum_k a_{2k}x_k \\ \vdots\\ \sum_k a_{nk}x_k \end{array}\right)\)
\end{itemize}

The identity element and the matrix transpose are given by

\begin{itemize}
\tightlist
\item
  \(I=\left( \begin{array}{ccccc}1&0&\dots&0&0\\ 0&1&\dots&0&0\\ \vdots&\vdots & \ddots & \vdots & \vdots\\ 0& 0 & \dots& 1 & 0  \\ 0& 0 &  \dots &0& 1  \end{array}\right)\)
\item
  Transpose: \(A^T\): \(\{ a_{ij}\}^T = \{ a_{ji}\}\) Example: If \(A =
  \left( \begin{array}{ccc}1 & 2 & 3 \\ 4 & 5 & 6
  \\ 7 & 8 & 9\end{array}\right)\) then \(A^T =
  \left( \begin{array}{ccc}1 & 4 & 7 \\ 2 & 5 & 8
  \\ 3 & 6 & 9\end{array}\right)\)
\end{itemize}

Some additional matrix terms and properties:

\begin{itemize}
\tightlist
\item
  The matrix determinant is indicated by det(\(A\))
\item
  The transpose formula is given by \((AB)^T=B^TA^T\)
\item
  The determinant formula is given by det(\(AB\)) = det(\(A\))det(\(B\))
\item
  A symmetric matrix is defined by \(A^T = A\)
\item
  A symmetric positive definite matrix satisfies \(x^T A x >0\) for
  \(x \neq 0\).
\end{itemize}
