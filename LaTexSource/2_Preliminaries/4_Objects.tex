\hypertarget{representation-of-objects-in-space}{%
\section{Representation of objects in
space}\label{representation-of-objects-in-space}}

Before we can derive the forward kinematics for a serial chain
manipulator, we need to be able to describe in a very general manner,
the changes in position or orientation of a solid object in space.

As indicated before, we represent a point or a vector in space (focused
on three dimensions for now) via

\[\begin{aligned}
p = \begin{pmatrix}
a \\
b \\
c \end{pmatrix}
\end{aligned}\]

It is common to extend this notation to four components by appending a
scale factor \(w>0\):

\[\begin{aligned}
p = \begin{pmatrix}
a \\
b \\
c \\
w \end{pmatrix}, \quad
x = a/w, \quad y = b/w, \quad z = c/w
\end{aligned}\]

The length of the vector is given by

\[\| p\| = \sqrt{x^2 + y^2 + z^2}\]

The factor \(w\) will scale the length of the vector as \(w\) changes
from 0 to infinity. When \(w=1\), the length is unchanged. For \(w=0\),
the length of the scaled vector would be infinite. In this extended
notation, it is commonly used to represent a direction vector. We will
find this notation useful later when we introduce homogeneous
coordinates.

\hypertarget{example}{%
\subsection{Example}\label{example}}

What is the length of \([1,-2,3,4]\)?

\[\sqrt{(1/4)^2 + (-2/4)^2 + (3/4)^2} = 1/16 + 1/4 + 9/16 = 14/16 = 7/8\]

So, we begin by reviewing how we represent a rigid body in space. For
this chapter and the next, robotic arms will be constructed from rigid
elements. To represent a point in space, we only need three variables
which is three degrees of freedom. Solid objects will also have an
orientation in space which is tracked by another three variable giving
six degrees of freedom. The orientation of the object will be tracked by
a matrix description called a frame. A frame is a collection of mutually
orthonormal vectors which act as a local coordinate system which can be
used describe an object from a different perspective or to orient a
robotic arm.

A full discussion of change of basis or change of coordinate systems can
be found in any text on linear algebra. We will restrict our discussion
to the elements directly applicable to what we need for modeling robot
arms. For here we are just focused on the change of frames (which are
made up of orthogonal basis vectors) in three dimensional space. We
start with a simple frame, Figure \texttt{Fig:frame}. This means that
frame G has three basis vectors for the three directions: \(A, B, C\).

\begin{quote}
Simple coordinate frame G.
\end{quote}

Assume that you have a vector \(U\) which is described in frame L,
Figure \texttt{Fig:frame1}:

\begin{quote}
Vector \(U\) in the frame.
\end{quote}

We can represent \(U\) as components in each of the coordinate
directions via

\[U = [ (U_A) , (U_B) , (U_C)  ]\]

where \(U_A, U_B, U_C\) are the projections or lengths in the basis
directions \(A, B, C\). See Figure \texttt{Fig:frame2}

\begin{quote}
The projection of \(U\) onto basis element \(A\).
\end{quote}

Assume that \(U\) is unit length and that you have two additional
vectors \(V, W\) which form a orthonormal frame, call it L.\footnote{Technically
  saying "assume orthonormal frame" for us would be redundant since we
  defined frames as made up from orthonormal sets of vectors.} Any
vector, say \(Q\) can be represented in either coordinate system. Since
we can express each vector \(U, V, W\) in the frame G. Doing so give us
a way to relate these two coordinate systems mathematically. Figure
\texttt{Fig:frames} gives the idea (although we have not yet discussed
translations, only rotations, but it is nice to separate for
readability).

\begin{quote}
Mapping from one coordinate frame to another.
\end{quote}

The way to relate vectors in Frame L to Frame G is to multiply the
vector (in Frame L) by an orthogonal rotation matrix

\[\begin{aligned}
R = \begin{pmatrix}
U_A & V_A & W_A  \\
U_B & V_B & W_B \\
U_C & V_C & W_C  \end{pmatrix}
\end{aligned}\]

An easy check will show this works. The vector \([1,0,0]\) in the Frame
L is \(U\). We can compute

\[\begin{aligned}
R \begin{pmatrix} 1 \\ 0 \\ 0\end{pmatrix}
=
\begin{pmatrix}
U_A & V_A & W_A  \\
U_B & V_B & W_B \\
U_C & V_C & W_C  \end{pmatrix} \begin{pmatrix} 1 \\ 0 \\ 0\end{pmatrix}
=
\begin{pmatrix}
U_A   \\
U_B  \\
U_C  \end{pmatrix}
= U \mbox{  in Frame G}
\end{aligned}\]

Note that in general transformations (given by non-singular matrices)
\(M\) can generate scalings, rotations, reflections, shears. and are
called transformation matrices. Also, these are linear transformations
and so they do not translate the vectors (since \(R0 = 0\)).

So, how does this relate to robotics? We will dive into the details of
robotic arms in the next chapter. For now, suffice it to say that one
will need to track the tool end of a robotic arm (e.g. where a drill bit
might be located). The direction that the tool tip faces is the approach
direction or the principle direction. We will use the vector \(a\) to
indicate the unit vector pointing in the approach direction. A second
orthogonal direction to \(a\) can be found and will be called \(n\). A
third direction, \(o\), selected using the cross-product
\(o = a \times n\). Note that some texts will use \(x = a\), \(y = n\)
and \(z = o\). At this point we can apply the transformations given
above. We will abuse the notation a bit and have \(x, y, z\) be the
directions of the world coordinate system or global frame. Load the
three vectors column-wise into a matrix

\[\begin{aligned}
R = \begin{pmatrix}
n_x & o_x & a_x  \\
n_y & o_y & a_y \\
n_z & o_z & a_z  \end{pmatrix}
\end{aligned}\]

and since these are mutually orthogonal vectors, we can see that this
acts like a coordinate system and what we are doing is the mathematical
operation of a change of coordinates or change of reference frame.

Let \(c = [c_1,c_2,c_3]\)

\[c' = Rc = c_1  n + c_2 o + c_3 a\]

and as indicated above, \(R\) transforms from one coordinate system to
another.

To perform a translation we need to augment by a displacement vector,
\(D\).

\[c' = Mc = Rc + D\]

This coordinate transformation and translation is known as an affine
map, \(M\). Although the affine map works well as a way to shift
coordinate systems, the linear transformation property (\(L(0) = 0\))
will turn out to be important and so to gain rotations, scalings as well
as the translation, but keeping the linearity property, we inflate our
matrix and introduce homogeneous coordinates and homogeneous
transformation matrices.

\hypertarget{homogeneous-coordinates-and-transforms}{%
\subsubsection{Homogeneous Coordinates and
Transforms}\label{homogeneous-coordinates-and-transforms}}

The homogeneous transforms act on four component vectors. We extend the
vectors by adding a fourth element. Homogeneous coordinates are defined
by appending a "1" at the bottom of a normal 3 component position vector
giving

\[\begin{aligned}
\xi = \begin{pmatrix}x \\ y \\ z \\ 1 \end{pmatrix}
\end{aligned}\]

Allows for general transforms: \(\xi' = A\xi\), which are linear
transforms. In most of our applications, we will be interested in a
rotation and then a translation. Shear and reflection are not an issue
here since these changes in coordinates will apply to rigid robot
hardware which (for now) does not experience reflection and shear.

We can represent a rigid body in space by giving the body a frame and
then representing that frame in space. The rotation and translation of
the frame can be combined into a single transformation matrix.
Specifically, the translation will be appended as a final column in the
matrix and a unit basis vector is added to the last row giving us

\[\begin{aligned}
T =  \begin{pmatrix}
         n_x & o_x & a_x & p_x \\
         n_y & o_y & a_y & p_y\\
         n_z & o_z & a_z & p_z \\
         0  &  0  &  0 & 1 \end{pmatrix}.
\end{aligned}\]

This turns out to be a rotation followed by a translation. To get a feel
of these operations, we will look at translations and rotations
separately.

In the setup, we saw the displacement or translation as an additive
operation.

\[\begin{aligned}
T + D' =
\begin{pmatrix}
         n_x & o_x & a_x & p_x \\
         n_y & o_y & a_y & p_y\\
         n_z & o_z & a_z & p_z \\
         0  &  0  &  0 & 1 \end{pmatrix}
 +
 \begin{pmatrix}0& 0 & 0 & t_1 \\
          0 & 0 & 0 & t_2\\ 0 &0 & 0 & t_3 \\
          0& 0& 0& 0 \end{pmatrix}
=
\begin{pmatrix}
         n_x & o_x & a_x & p_x + t_1\\
         n_y & o_y & a_y & p_y + t_2\\
         n_z & o_z & a_z & p_z + t_3\\
         0  &  0  &  0 & 1 \end{pmatrix}
\end{aligned}\]

However, we can write this as a matrix multiplication by combining the
4x4 identity with the displacement matrix and so the pure translation
matrix can be formed by

\[\begin{aligned}
D = I + D' = \begin{pmatrix}1 & 0 & 0 & t_1 \\
         0 & 1 & 0 & t_2\\ 0 &0 & 1 & t_3 \\
         0& 0& 0& 1 \end{pmatrix}
\end{aligned}\]

You will note that it has the property that if you apply a translation
to the frame:

\[\begin{aligned}
DR =
\begin{pmatrix}1 & 0 & 0 & t_1 \\
         0 & 1 & 0 & t_2\\ 0 &0 & 1 & t_3 \\
         0& 0& 0& 1 \end{pmatrix}
\begin{pmatrix}
         n_x & o_x & a_x & p_x \\
         n_y & o_y & a_y & p_y\\
         n_z & o_z & a_z & p_z \\
         0  &  0  &  0 & 1 \end{pmatrix}
=
\begin{pmatrix}
         n_x & o_x & a_x & p_x + t_1\\
         n_y & o_y & a_y & p_y + t_2\\
         n_z & o_z & a_z & p_z + t_3\\
         0  &  0  &  0 & 1 \end{pmatrix}
\end{aligned}\]

A very simple example for this case, translate the point \(p=[5,12,13]\)
by \(v=<3,4,5>\). We can point to \(p\) using the matrix T:

\[\begin{aligned}
T =    \begin{pmatrix}1 & 0 & 0 & 5 \\ 0 & 1 & 0 & 12\\ 0 &0 & 1 & 13 \\ 0& 0& 0& 1 \end{pmatrix}
\end{aligned}\]

and then add the displacement \(v\):

\[\begin{aligned}
T_v T =  \begin{pmatrix}1 & 0 & 0 & 3 \\ 0 & 1 & 0 & 4\\ 0 &0 & 1 & 5 \\ 0& 0& 0& 1 \end{pmatrix}
\begin{pmatrix}1 & 0 & 0 & 5 \\ 0 & 1 & 0 & 12\\ 0 &0 & 1 & 13 \\ 0& 0& 0& 1 \end{pmatrix}
= \begin{pmatrix}1 & 0 & 0 & 8 \\ 0 & 1 & 0 & 16\\ 0 &0 & 1 & 18 \\ 0& 0& 0& 1 \end{pmatrix}
\end{aligned}\]

Let \(R\) be a rotation matrix. Rotation can be expressed by

\[\begin{aligned}
R = \begin{pmatrix}
         n_x & o_x & a_x & 0 \\
         n_y & o_y & a_y & 0\\
         n_z & o_z & a_z & 0 \\
         0  &  0  &  0 & 1 \end{pmatrix}
\end{aligned}\]

It is useful to review the basic rotations about the three axes:

\begin{itemize}
\item
  About \(z\)

  \[\begin{aligned}
  R_z = \begin{pmatrix}\cos\theta & -\sin\theta & 0 & 0 \\
           \sin\theta & \cos\theta & 0 & 0\\ 0 &0 & 1 & 0 \\
           0& 0& 0& 1 \end{pmatrix}
  \end{aligned}\]
\item
  About \(x\)

  \[\begin{aligned}
  R_x = \begin{pmatrix}1 & 0 & 0 & 0 \\ 0 & \cos\theta & -\sin\theta & 0  \\
           0& \sin\theta & \cos\theta & 0 \\
           0& 0& 0& 1 \end{pmatrix}
  \end{aligned}\]
\item
  About \(y\)

  \[\begin{aligned}
  R_y = \begin{pmatrix}\cos\theta & 0 & -\sin\theta & 0  \\ 0 & 1 & 0 & 0\\
           \sin\theta &0& \cos\theta & 0 \\
           0& 0& 0& 1 \end{pmatrix}
  \end{aligned}\]
\end{itemize}

It is not hard to show that \(R^{-1} = R^T\). We can also verify that
replacing \(\theta\) with \(-\theta\) is the reverse rotation and gives
the same thing as \(R^{-1}\) and \(R^T\).

Example, rotate the point \([1,2,3]\) by 30 degrees about the y-axis.
Let \(v\) point to \([1,2,3]\).

\[\begin{aligned}
w = R_yv = \begin{pmatrix}\cos 30^\circ & 0 & -\sin 30^\circ & 0  \\ 0 & 1 & 0 & 0\\
         \sin 30^\circ &0& \cos 30^\circ & 0 \\  0& 0& 0& 1 \end{pmatrix}v
    = \begin{pmatrix}\sqrt{3}/2 & 0 & -1/2 & 0  \\ 0 & 1 & 0 & 0\\
             1/2 &0& \sqrt{3}/2& 0 \\  0& 0& 0& 1 \end{pmatrix} \begin{pmatrix} 1 \\ 2 \\ 3\\ 1\end{pmatrix}
    = \begin{pmatrix} \sqrt{3}/2 - 3/2 \\ 2 \\ 1/2 + 3\sqrt{3}/2\\ 1\end{pmatrix}
\end{aligned}\]

These operations can be chained together and this is the basis for the
matrix T we began with. We then see the matrix T as the representation
of the orientation and position of some frame that describes solid body.

\[\begin{aligned}
T  =
\begin{pmatrix}
R  & d \\
0 & 1\\
\end{pmatrix}
\end{aligned}\]

We will multiply this matrix often and this should be done blockwise.
Let \(u = (x,y,z,1)\), \(w = (x,y,z)\)

\[\begin{aligned}
Tu = \begin{pmatrix} R & d \\ 0 & 1 \end{pmatrix} u = \begin{pmatrix} Rw + d \\ 1 \end{pmatrix}
\end{aligned}\]

For example a rotation about the \(z\) axis and then a translation of
\((t_1, t_2, t_3 )\) would have the following tansformation matrix.

\[\begin{aligned}
\xi' =
\begin{pmatrix}
\cos\theta & -\sin\theta & 0 & t_1 \\
\sin\theta & \cos\theta & 0 & t_2\\
0 &0 & 1 & t_3 \\
0& 0& 0& 1
\end{pmatrix}  \xi
\end{aligned}\]

As an aside, we can chain as many of these as we would like. Assume that
you are given the following motions: Rotate about the x-axis 30 degrees,
translate in y by 3cm, and rotate about the z axis 45 degrees. Find the
coordinate transformation.

\[\begin{aligned}
R_1 = \begin{pmatrix}1 & 0 & 0 & 0 \\ 0 & \cos 30 & -\sin 30 & 0  \\
         0& \sin 30 & \cos 30 & 0 \\
         0& 0& 0& 1 \end{pmatrix},  \quad R_2 =
         \begin{pmatrix}\cos 45 & -\sin 45 & 0 & 0 \\
         \sin 45 & \cos 45 & 0 & 0\\ 0 &0 & 1 & 0 \\
         0& 0& 0& 1 \end{pmatrix}
\end{aligned}\]

\[\begin{aligned}
D = \begin{pmatrix}1 & 0 & 0 & 0 \\
         0 & 1 & 0 & 3\\ 0 &0 & 1 & 0 \\
         0& 0& 0& 1 \end{pmatrix}
\end{aligned}\]

Then the transformation is \(M = R_2DR_1\)

\[\begin{aligned}
= \begin{pmatrix}\cos 45 & -\sin 45 & 0 & 0 \\
         \sin 45 & \cos 45 & 0 & 0\\ 0 &0 & 1 & 0 \\
         0& 0& 0& 1 \end{pmatrix}
         \begin{pmatrix}1 & 0 & 0 & 0 \\
         0 & 1 & 0 & 3\\ 0 &0 & 1 & 0 \\
         0& 0& 0& 1 \end{pmatrix}
         \begin{pmatrix}1 & 0 & 0 & 0 \\ 0 & \cos 30 & -\sin 30 & 0  \\
         0& \sin 30 & \cos 30 & 0 \\
         0& 0& 0& 1 \end{pmatrix}
\end{aligned}\]

\[\begin{aligned}
=
\begin{pmatrix}\cos 45 & -\sin 45 & 0 & 0 \\
         \sin 45 & \cos 45 & 0 & 0\\ 0 &0 & 1 & 0 \\
         0& 0& 0& 1 \end{pmatrix}
\begin{pmatrix}1 & 0 & 0 & 0 \\
         0 & \cos 30 & -\sin 30 & 3\\ 0 &\sin 30 & \cos 30 & 0 \\
         0& 0& 0& 1 \end{pmatrix}
\end{aligned}\]

\[\begin{aligned}
=
\begin{pmatrix}
         \cos 45 & -\sin 45 \cos 30 & -\sin 45 \sin 30 & -3\sin 45 \\
         \sin 45 & \cos 45 \cos 30 & -\cos 45 \sin 30 & 3\cos 45\\
         0       & \sin 30 & \cos 30 & 0 \\
         0       & 0& 0& 1 \end{pmatrix}
\end{aligned}\]

Going forward we will just have T represent the rotation and
displacement pair, and then chain those. It is also useful to have the
inverse of the transformation. How does one invert the transformations?
For us this is simplified since we are restricted to rotations and
translations which are easily inverted. Rotation matrices are orthogonal
and so

\[R^{-1} = R^T\]

For example, the inverse of the 60 degree rotation mentioned above:

\[\begin{aligned}
\begin{pmatrix}1 & 0 & 0 & 0 \\ 0 & \cos 60 & -\sin 60 & 0  \\
         0& \sin 60 & \cos 60 & 0 \\
         0& 0& 0& 1 \end{pmatrix}^{-1} =
         \begin{pmatrix}1 & 0 & 0 & 0 \\ 0 & \cos 60 & \sin 60 & 0  \\
         0& -\sin 60 & \cos 60 & 0 \\
         0& 0& 0& 1 \end{pmatrix}
\end{aligned}\]

Translation matrices are simple as well. One just negates the
translation components.

Thus:

\[\begin{aligned}
\begin{pmatrix}1 & 0 & 0 & a \\ 0 & 1 & 0 & b  \\
         0& 0 & 1 & c \\
         0& 0& 0& 1 \end{pmatrix}^{-1} =
         \begin{pmatrix}1 & 0 & 0 & -a \\ 0 & 1 & 0 & -b  \\
         0& 0 & 1 & -c \\
         0& 0& 0& 1 \end{pmatrix}
\end{aligned}\]

Thus we can just undo the transformations individually.

You may guess that the inverse of the combined transformation must
include the transpose of the rotation and the negative of the
displacement. By trial and error you can find it. Here is the result:

\[\begin{aligned}
T^{-1} =  \begin{pmatrix}
         n_x & n_y & n_z & -p\cdot n \\
         o_x & o_y & o_z & -p\cdot o\\
         a_x & a_y & a_z & -p\cdot a \\
         0  &  0  &  0 & 1 \end{pmatrix}.
\end{aligned}\]

\hypertarget{successive-transformations}{%
\subsubsection{Successive
transformations}\label{successive-transformations}}

Once you can relate one frame (coordinate system) to another, we can
chain these to relate additional coordinate systems. Each new frame is
related to the previous frame by a transformation.

\begin{quote}
Successive changes of frames
\end{quote}

Successive motion can be computed by matrix multiplication. This is done
by multiplication from left to right: \(T_1 T_2 T_3\). Any type of
transformation will work here. We can mix rotations and translations.
For example, let \(R\) be a rotation and \(D\) be a translation. Then

\begin{quote}
\[T = DR\]
\end{quote}

is the matrix that describes the rotation by \(R\) followed by
translation by \(D\).

\[\begin{aligned}
\begin{pmatrix}
n_x & o_x & a_x & p_x \\
n_y & o_y & a_y & p_y\\
n_z & o_z & a_z & p_z \\
0& 0& 0& 1 \end{pmatrix}
=
\begin{pmatrix}1 & 0 & 0 & p_x \\
0 & 1 & 0 & p_y\\
0 &0 & 1 & p_z \\
0& 0& 0& 1 \end{pmatrix}
\begin{pmatrix}
n_x & o_x & a_x & 0 \\
n_y & o_y & a_y & 0 \\
n_z & o_z & a_z & 0 \\
0& 0& 0& 1 \end{pmatrix}
\end{aligned}\]

It is useful to have a feel for the difference in the postmultiplication
we are doing and the premultiplication you may have seen in other
context's such as the LU factorization.

The core idea illustrated in Figure \texttt{Fig:frames2} is that
starting with \(T_1\) and successive postmultiplication by
\(T_2, T_3, ...\) is layer by layer creating the transformation that
will take an object described in the final frame (coordinate system) and
represent it in the first frame (coordinate system). It rotates the
frames from first frame to final frame. When you are transforming a
vector from position to position (rotating, translating, etc) then you
would perform a sequence of premultiplications. It is a difference in
view whether you are holding the outer frame and moving a vector verses
having a fixed vector and moving the reference frame. Both are valuable
ways to look at these transformations. To see this, we illustrate with
specific points.

Begin with a point \(x\) in space. An application of a transformation,
\(T_1\), with respect to the global frame carries this point to a new
point \(x'\):

\[x' = T_1x\]

We can think of the new point \(x'\) as movement of the original point
\(x\). This can be repeated. Apply another transformation \(T_2\) to the
new point \(x'\):

\[x" = T_2x' = T_2(x') = T_2(T_1x) = T_2T_1x\]

Note that each transform was done with respect to the fixed frame.

Again, begin with a point \(x\) in space. If we view the transformation,
\(T\) from the perspective of the point (which will be fixed), then it
appears that the "fixed" frame is moving AND that the motion is in the
\emph{opposite} direction of the fixed frame transformation. Opposite
here would be the inverse transformation: \(T^{-1}\). Thus combined
transformations from the point's ``point of view'':

\begin{quote}
\[T^{-1} = T_2^{-1}T_1^{-1}, \quad \mbox{or}\quad T = \left(T_2^{-1} T_1^{-1} \right)^{-1}\]
\end{quote}

\[T = T_1T_2\]

This places the list of operations in reverse order. Successive
transformations relative to the global frame are left multiplied:

\[T = T_n T_{n-1} \dots T_1 T_0\]

For example, take a rotation about \(z\) of 30 degrees, \(R_1\),
followed by a rotation about \(x\) by 60 degrees, \(R_2\):

\[\begin{aligned}
R = R_2R_1= \begin{pmatrix}1 & 0 & 0 & 0 \\ 0 & \cos 60 & -\sin 60 & 0  \\
         0& \sin 60 & \cos 60 & 0 \\
         0& 0& 0& 1 \end{pmatrix}\begin{pmatrix}\cos 30 & -\sin 30 & 0 & 0 \\
         \sin 30 & \cos 30 & 0 & 0\\ 0 &0 & 1 & 0 \\
         0& 0& 0& 1 \end{pmatrix}
\end{aligned}\]

Successive transformations relative to the moving frame are right
multiplied:

\[T = T_0 T_{1} \dots T_{n-1} T_n\]

For example, take a rotation about x by 45 degrees, \(R\), followed by a
translation in z by 4 cm, \(T\):

\[\begin{aligned}
M = TR= \begin{pmatrix}1 & 0 & 0 & 0 \\ 0 & \cos 60 & -\sin 60 & 0  \\
         0& \sin 60 & \cos 60 & 0 \\
         0& 0& 0& 1 \end{pmatrix}\begin{pmatrix}1 & 0 & 0 & 0 \\
         0 & 1 & 0 & 0\\ 0 &0 & 1 & 4 \\
         0& 0& 0& 1 \end{pmatrix}
\end{aligned}\]

The formula for inverting products of transformation matrices is given
by

\[T^{-1} = \left( T_n T_{n-1} \dots T_1 T_0 \right)^{-1}
  = T_0^{-1} T_{1}^{-1} \dots T_{n-1}^{-1} T_n^{-1}\]

\hypertarget{rpy-angles-and-euler-angles}{%
\subsubsection{RPY Angles and Euler
Angles}\label{rpy-angles-and-euler-angles}}

Roll-Pitch-Yaw (RPY) angles provide the position and orientation of a
craft by using a translation to body center and then three rotation
matrices for craft pose.

\begin{itemize}
\tightlist
\item
  Rotation about \(a\) (z axis) - Roll
\item
  Rotation about \(o\) (y axis) - Pitch
\item
  Rotation about \(n\) (x axis) - Yaw
\end{itemize}

\[M = R_nR_oR_aT\]

Euler angles provide the position and orientation of a craft by using a
translation to body center and then three rotation matrices for craft
pose. However - reference is with respect to the body, not the world
coordinates.

\begin{itemize}
\tightlist
\item
  Rotation about \(a\) (z axis) - Roll
\item
  Rotation about \(o\) (y axis) - Pitch
\item
  Rotation about \(a\) - Roll
\end{itemize}

\[M = R_aR_oR_aT\]

\textbf{Footnotes}
