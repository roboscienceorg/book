\hypertarget{student-notes}{%
\section{Student Notes}\label{student-notes}}

These notes are intended for both formal students and casual readers. In
2009, SDSMT Computer Science offered a Mobile Robotics course for the
first time. As mentioned above, we used \emph{Introduction to Autonomous
Mobile Robots} by Siegwart and Nourbakhsh~\texttt{Siegwart:2004:IAM} .
This book has evolved from course notes. There is a significant lack of
textbooks on robotics for those individuals who approach the subject
with a computing background and interest. This is not suprising since
the bulk of the effort in robotics over the last 50 years has been with
the design and control of the mechanical and electrical systems.

It is only recently that the embedded computing and the supporting
software has been powerful enough to attempt to deal with the complexity
found in unstructured environments outside manufacturing. The landscape
is changing and with the advent of self-driving cars and other
autonomous systems, the need for computer scientists in robotics is
accelerating. This book attempts to balance the competing forces of
presenting a complete thorough coverage with the normal constraints on
time and resources. Robotics covers many different disciplines. Some
specific techniques can require a great deal of mathematics or other
specialized skill. Our goal is to get you up and running as quickly as
possible without sacrificing the core concepts needed to progress later.
We will iterate through the subject preceeding deeper on each pass.

With this approach we can get you running with a robotic system without
months of theory. But we will come back and build on what you have
learned. A bit like an Agile method applied to your learning. In past
versions of the course, we have separated material. So, we would do all
of the mechanical content, then all of the electrical content, then
algorithms, tools and so forth. This approach was well criticized on
Medium.com with a burrito analogy. Imagine having the burrito with all
of the beans on one end, then a section of only meat, a section of
cheese and the other end was the sauce and seasoning. Not very
appealing. Our goal is, when it makes sense, to "mix it up". Not only
does this avoid boredom, it should help with understanding how to
integrate the concepts.

We have developed the chapters to be as independent as possible. So, one
could read the chapters out of order. However, we have a progression
that has developed over a decade presenting the concepts and we think
this will be the most effective for those who want a more complete
understanding of the subject. In addition, there is an iterative
approach to trying to get you up and running quickly, so jumping around
may extend the learning time. So, we suggest that you consider reading
the chapters in the order presented.

A common question is "what background is really needed?" We will be
using algorithms and software to solve problems. So, you need to know
how to program. Language is not important since once you know one
language it is very easy to learn another. This book uses Python to
demonstrate concepts. Python is really easy to learn and use. For our
purposes it is plenty fast enough and is the choice for robotics
prototyping. The algorithms will employ certain data structures which if
you don't have a course in data structures can be understood after a
good second programming course.

There are times when we need to model the robot's electrical or
mechanical systems. These models are based in calculus and physics. You
should have a year of calculus and a year of physics. Ideally you would
have seen a course in probability and a course in linear algebra. This
is not the case for everyone and so we have the essential math covered
in the appendix.
