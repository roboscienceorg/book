\hypertarget{about-this-text}{%
\section{About this text}\label{about-this-text}}

\texttt{Robotics} is an engineering discipline stemming from the fusion
of science and art. One of the most interesting and exciting aspects of
robotics is how many different fields it touches. At some institutions,
for example Worcester Polytechnic Institute (WPI), it is now a
undergraduate degree program. Many courses can be taught within the
field to cover the different aspects just like what we do with
Electrical or Mechanical Engineering. So any attempt at a survey course
is challenging. This book is not a really a great survey of the field.
For such a large field, a survey would be a long list of titles,
authors, researchers and so on. There would be no room to "get down into
the weeds". We want the student (reader) to also get in the detail and
acquire real skills. To do that, we must focus. So, the book is oriented
around mobile robots. You will see that our coverage of manipulators is
very thin. Fortunately, there are fabulous texts on manipulators
available.

Our goal is to get the reader to the point where they can build a simple
mobile robot which can interact with the environment. This requires
covering basics of sensing, computation and motion. Our target will be
the creation of autonomous vehicles.

There are many issues to resolve with this book. Many small ones and
some large ones. The obvious problems relate to content, coverage,
background and other academic concerns. Smaller ones relate to
formatting, specific examples and images.

\hypertarget{copyright-concern}{%
\subsection{Copyright concern}\label{copyright-concern}}

This book came from course notes. The notes were restricted to class and
as such subject to \emph{Fair Use} guidelines. Materials from the web
such as images were used as is. Going to publication (either print or
online) brings a different set of rules. We have gone to great effort to
make sure that all of the content is legally used. Meaning that
externally derived content is in the public domain or has a "use with
attribute" style license.

Note

If you see something that violates a use agreement, please contact us
immediately:
\href{mailto:jeff.mcgough@sdsmt.edu}{\nolinkurl{jeff.mcgough@sdsmt.edu}}

\hypertarget{conversion-issues}{%
\subsection{Conversion Issues}\label{conversion-issues}}

The original text was written in LaTeX. We have converted over to
restructuredText using a fabulous tool called Pandoc. Pandoc is good,
but not perfect. We had a very complicated latex document which Pandoc
was not able to completely convert. Although we have spent many hours
with the conversion, there are aspects we have missed. The labels and
references were a particular problem and many are still broken. Let us
know about format problems (broken links, etc).

\hypertarget{document-content}{%
\subsection{Document Content}\label{document-content}}

This is a living document. There are many sections of the original notes
that need to be converted over to to rst AND there are many sections
that need to be written. Robotics is a active field with new technology
daily. New stuff needs to be written and some old tools need to be
updated. We hope to get a bunch of this done by 9/1/18.

\hypertarget{robotics-education-community}{%
\subsection{Robotics Education
Community}\label{robotics-education-community}}

Our goal is to build an open source robotics education community. If you
want to help edit, add sections, add homework problems, rework sections,
etc, please contact us. If you have comments or concerns, again please
contact us.
