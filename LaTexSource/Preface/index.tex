\hypertarget{Chap:Preface}{%
\section{Preface}\label{Chap:Preface}}

Historically, the academic study of robotics has been limited to the
Mechanical Engineering or Electrical Engineering departments. This makes
perfect sense if the target field is manufacturing. Often requiring
significant power, reliable repetition or precise positioning, these
departments focused on machine design, controls and targeted
instrumentation. Leaving the factory floor, however, requires a great
deal more in generalized sensing and machine ability. It demands
powerful computing and far more advanced algorithms. To address this
need, the Computer Science department at the South Dakota School of
Mines and Technology decided to include Robotics as part of the Computer
Science curriculum. We developed a course devoted to robotics. The
original course began with the text \emph{Introduction to Autonomous
Mobile Robots} by Siegwart and Nourbakhsh~\texttt{Siegwart:2004:IAM} .
It is a really good survey text. The issue that arose was that we found
ourselves having to fill in details, add content, provide more current
examples, provide information on frameworks and enhance focus on
software. The slides, notes and handouts ended up growing into this
text.

The word ``robotics'' encompasses many different meanings and fields.
This is very different from a subject like Calculus, for example, for
which all the books written in the last two centuries cover the exact
same material - nearly lining up in chapter and section numbers. Not so
with robotics. This subject touches on all of the engineering
disciplines, mathematics, computer science (if this was not on your
engineering list) and several of the sciences. It can be aimed at
children, hobbyists, engineers, researchers, managers, evil scientists
and many more. Finding a textbook that fell at the right level and
contained the specific subjects that we wanted turned out to be a
difficult task. There are many, many very great books on the subject,
but none that addressed the needs of juniors and seniors studying
computer science.

This text exists to present the particular subjects we wanted to cover
and relate them at the level that is appropriate for our students. Our
course is aimed at computer science students who have had three
semesters of Calculus, a semester of differential equations, a semester
of linear algebra, exposure to computer hardware, and significant
experience with software development. This book will take a computer
science perspective and a software focus. The ordering of the material
is based on the typical interests and goals for someone who is a senior
in computer science wanting to learn about controlling mechatronic
systems.

\hypertarget{acknowledgements}{%
\subsection{Acknowledgements}\label{acknowledgements}}

This text would not exist if it were not for the support of many
individuals. The author extends his thanks for the support by the South
Dakota School of Mines and Technology, Mathematics and Computer Science
Department, specifically, Dr. Kyle Riley, Dept Head. Several individuals
helped significantly in their day to day efforts to support software and
chapter development: Scott Logan, Caleb Jamison, Chris Smith, Remington
Bullis, Joe Lillo, Lisa Woody and Kali Regenold. I would also like to
thank Stephanie Athow, Kelsey Bellew, Paul Blasi, Julian Brackins, John
Brink, Andrew Carpenter, Michael Cerv, Marshall Gaucher, Yun Gwon,
Lawrence Hoffman, Travis Larson, Scott Samson, Derek Stotz, Donovan
Torgerson and Kyle Macmillan for feedback and editing assistance.

students.rst instructors.rst
