\hypertarget{course-instructor-notes}{%
\section{Course Instructor Notes}\label{course-instructor-notes}}

This course began in 2009 as mixed senior undergraduate and first year
graduate student offering. Very quickly it became clear that most of the
challenge was in the breadth of material and skill set of the students.
Since then I have proceeded to make every mistake possible with this
class. Fortunately at a small school we have forgiving students. This
book, specifically the content and organization, is a result of the
decade of running the class and correcting those errors.

The goal of this text is several fold. First, there are limited
offerings in this space. You can find K-12 materials on robots,
especially with the Lego system, easily on the internet. There is a ton
of hobbyist sites (Arduino, Raspberry Pi, etc) that are fabulous. The
problem is that they lack the completeness and rigor needed for a
college course. Since robotics is a research area, finding graduate
level references is not hard, but we have the opposite problem of that
being too difficult or not sufficiently general in scope. {[}And
honestly, even some undergraduate texts outside your personal area of
expertise can feel like a graduate text.{]}

The second goal was to create a text aimed at students with more of a
computer science background and not Mechanical/Electrical Engineering
background as with most of the college level texts available. In
addition, we are working to expose these students to a diverse
curriculum since it is just not reasonable to get three or four Bachelor
degrees to be a well schooled roboticist.

So, this text attempts to survey the field, provide sufficient depth for
the student and cover current industry tools, all while keeping the
course engaging to students. We have worked to balance theory with
practice since it seems to help the learning and retention process.
There is some mixing of topics to keep student interest. Earlier
versions of the course were partitioned over fields and application.
Students liked having material mixed and then progressing in level. So,
because of this, the ordering of chapters might not be as expected. We
have worked hard to get the student up and running in virtual as well as
real robots as soon as practical. Some theoretical (important) material
is left to later in the text. One benefit is that when students try the
simple approaches, and those fail, they more easily understand the more
advanced and mathematical approaches later.

It is easier to limit or focus the students in the class. However, a
diversity of students makes for a better course. The problem then is
that you cannot assume all students have the same background. In the
version we offer at SDSMT, we have Data Structures and Differential
Equations as background. However, we want students from Mechanical and
Electrical Engineering to participate and at SDSMT they normally don't
take Data Structures. Likewise, many Computer Science students will not
have Differential Equations in their background. We normally meet and
make sure they understand what elements that they might need to make up.
Then sign them in. This works well for us.

We run a one semester course. There is more material here than a three
or four credit semester course (15 week) can address. Now that the text
has been open sourced and placed online, we expect the text to grow.
There are a couple of directions we can proceed. One is to fork the text
repo and build multiple textbooks for different applications (such as
student levels or backgrounds, or goals). Another is having a modular
system which the instructor selects sections/chapters and have Sphinx
build a custom text. If you have thoughts on this or would like a custom
text built now, please contact Jeff McGough,
\href{mailto:jeff.mcgough@sdsmt.edu}{\nolinkurl{jeff.mcgough@sdsmt.edu}}
.
